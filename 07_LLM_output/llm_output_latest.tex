
\documentclass[a4paper,12pt]{article}
\usepackage[utf8]{inputenc}
\usepackage[T1]{fontenc}
\usepackage[ngerman]{babel}
\usepackage{amsmath, amssymb, mathtools}
\usepackage{geometry}
\geometry{margin=2.5cm}

\begin{document}

\section*{Kapitel 6}
\section*{Matrizen \& Lineare Abbildungen}
\subsection*{6.1 Matrix-Algebra}
\subsection*{6.1.1 Definition}
Wir machen folgende Definition.\\
Definition 6.1 Matrix\\
Seien $m, n \in \mathbb{N}^{+}$. Eine reelle $m \times n$-Matrix $A$ ist eine Zahlentabelle mit $m$ Zeilen und $n$ Spalten der Form

\[
A=\left[\begin{array}{llll}
A^{1}{ }_{1} & A^{1}{ }_{2} & \ldots & A^{1}{ }_{n}  \tag{6.1}\\
A^{2}{ }_{1} & A^{2}{ }_{2} & \ldots & A^{2}{ }_{n} \\
\vdots & \vdots & \vdots & \vdots \\
A^{m}{ }_{1} & A^{m}{ }_{2} & \ldots & A^{m}{ }_{n}
\end{array}\right],
\]

wobei alle $A^{i}{ }_{j} \in \mathbb{R}$.\\
Bemerkungen:\\
i) Die Zahlen $m$ und $n$ heissen Dimensionen der Matrix $A$.\\
ii) Die reellen Zahlen $A^{i}{ }_{j}$ heissen Komponenten der Matrix $A$. Eine $m \times n$-Matrix besteht offensichtlich aus $m \cdot n$ Komponenten.\\
iii) Wir werden später sehen, dass es sinnvoll ist, den Zeilen-Index $i$ oben und den SpaltenIndex $j$ unten zu schreiben.\\
iv) Für die Menge aller reellen $m \times n$-Matrizen gibt es verschiedene Bezeichnungen. Wir verwenden

\begin{equation*}
\mathbb{M}(m, n, \mathbb{R})=\mathbb{R}^{m \times n} . \tag{6.2}
\end{equation*}

v) Wir betrachten die Matrix

\[
M=\left[\begin{array}{lll}
1 & 2 & 3  \tag{6.3}\\
4 & 5 & 6
\end{array}\right] .
\]

\end{document}