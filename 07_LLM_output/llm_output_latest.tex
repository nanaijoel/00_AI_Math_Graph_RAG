
\documentclass[a4paper,12pt]{article}
\usepackage[utf8]{inputenc}
\usepackage[T1]{fontenc}
\usepackage[ngerman]{babel}
\usepackage{amsmath, amssymb, mathtools}
\usepackage{geometry}
\geometry{margin=2.5cm}

\begin{document}
\noindent
In den verwendeten Quellen, speziell auf Seite 46, wird der Nabla-Operator $\boldsymbol{\nabla}$ in n-dimensionalen Raum $\mathbb{R}^n$ als der Vektor der partiellen Ableitungen definiert:

\[
\boldsymbol{\nabla} := \begin{bmatrix}
\partial_1 \\
\vdots \\
\partial_n
\end{bmatrix}.
\]

Dies ermöglicht, Operationen wie Divergenz und Rotation auf Vektorfelder anzuwenden. Der Nabla-Operator $\boldsymbol{\nabla}$ kann benutzt werden, um verschiedene wichtige Differentialoperationen durchzuführen:

\begin{itemize}
    \item \textbf{Gradient eines Skalarfeldes $f$}: 
    \[
    \boldsymbol{\nabla} f = \begin{bmatrix}
    \frac{\partial f}{\partial x_1} \\
    \vdots \\
    \frac{\partial f}{\partial x_n}
    \end{bmatrix},
    \]
    wobei $f: \mathbb{R}^n \to \mathbb{R}$.
    
    \item \textbf{Divergenz eines Vektorfeldes $\mathbf{v}$}:
    \[
    \text{div}(\mathbf{v}) = \boldsymbol{\nabla} \cdot \mathbf{v} = \sum_{i=1}^n \frac{\partial v_i}{\partial x_i},
    \]
    wobei $\mathbf{v}: \mathbb{R}^n \to \mathbb{R}^n$.
    
    \item \textbf{Rotation eines Vektorfeldes $\mathbf{v}$ in $\mathbb{R}^3$}:
    \[
    \text{rot}(\mathbf{v}) = \boldsymbol{\nabla} \times \mathbf{v},
    \]
    spezifisch für drei Dimensionen.
\end{itemize}

Zusätzlich zu diesen Anwendungen wird der Nabla-Operator in der theoretischen Physik und in der Ingenieurwissenschaft häufig verwendet, um fundamentale Gesetze wie die Maxwell-Gleichungen zu formulieren.
\end{document}