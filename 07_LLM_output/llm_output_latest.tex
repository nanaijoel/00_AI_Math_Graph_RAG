
\documentclass[a4paper,12pt]{article}
\usepackage[utf8]{inputenc}
\usepackage[T1]{fontenc}
\usepackage[ngerman]{babel}
\usepackage{amsmath, amssymb, mathtools}
\usepackage{geometry}
\geometry{margin=2.5cm}

\begin{document}

\noindent
\textbf{Definition des Gradienten} (Definition 2.19):
Der Gradient einer differenzierbaren reellwertigen Funktion $f: \mathbb{R}^n \to \mathbb{R}$ ist das Vektorfeld $\boldsymbol{\nabla} f$ definiert durch:
\[
\boldsymbol{\nabla} f := \begin{bmatrix}
    f_{,1} \\
    f_{,2} \\
    \vdots \\
    f_{,n}
\end{bmatrix}.
\]
Hierbei sind $f_{,i}$ die partiellen Ableitungen von $f$ bezüglich der Koordinaten.

\noindent
\textbf{Bemerkungen:}
\begin{itemize}
  \item Der Gradient transformiert ein Skalarfeld in ein Vektorfeld.
  \item Geometrisch repräsentiert $\boldsymbol{\nabla} f$ an jedem Punkt die Richtung des steilsten Anstiegs von $f$.
\end{itemize}

\noindent
\textbf{Beispiele für den Gradienten:}
\begin{itemize}
  \item $f(x, y) = x^2 y^2$ hat den Gradienten $\boldsymbol{\nabla} f = \begin{bmatrix} 2xy^2 \\ 2x^2y \end{bmatrix}$.
  \item $f(x, y, z) = x^2 y + z$ hat den Gradienten $\boldsymbol{\nabla} f = \begin{bmatrix} 2xy \\ x^2 \\ 1 \end{bmatrix}$.
\end{itemize}

\noindent
\textbf{Rechenregeln für Gradienten} (Sätze 2.12 und 2.13):
\begin{itemize}
  \item \textbf{Linearität:} $\boldsymbol{\nabla} (a f + b g) = a \boldsymbol{\nabla} f + b \boldsymbol{\nabla} g$ für skalare Felder $f, g$ und Konstanten $a, b$.
  \item \textbf{Produktregel:} $\boldsymbol{\nabla} (f g) = (\boldsymbol{\nabla} f) g + f \boldsymbol{\nabla} g$.
  \item \textbf{Kettenregel:}
    \begin{itemize}
      \item Für $f(x) = g(h(x))$: $\boldsymbol{\nabla} f = g'(h(x)) \boldsymbol{\nabla} h$.
      \item Für $f(x) = g(\mathbf{h}(x))$: $f'(x) = \langle \boldsymbol{\nabla} g(\mathbf{h}(x)), \mathbf{h}'(x) \rangle$.
    \end{itemize}
\end{itemize}

\end{document}