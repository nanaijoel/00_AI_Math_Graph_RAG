
\documentclass[a4paper,12pt]{article}
\usepackage[utf8]{inputenc}
\usepackage[T1]{fontenc}
\usepackage[ngerman]{babel}
\usepackage{amsmath, amssymb, mathtools}
\usepackage{geometry}
\geometry{margin=2.5cm}

\begin{document}

\section*{Definition eines homogenen Vektorfelds}
Ein Vektorfeld $\mathbf{v}: A \rightarrow B$, definiert über Teilmengen $A, B \subseteq \mathbb{R}^n$ mit $n \in \mathbb{N}^+ \backslash \{1\}$, heißt \textit{homogen}, wenn ein Vektor $\mathbf{w} \in B$ existiert, so dass für alle Punkte $\mathbf{p} \in A$ gilt:
\[ \mathbf{v}(\mathbf{p}) = \mathbf{w}. \]
Dies bedeutet, dass das Vektorfeld $\mathbf{v}$ an jedem Punkt in $A$ den gleichen Vektor $\mathbf{w}$ annimmt.

\end{document}