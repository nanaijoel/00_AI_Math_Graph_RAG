\documentclass[10pt]{article}
\usepackage[ngerman]{babel}
\usepackage[utf8]{inputenc}
\usepackage[T1]{fontenc}
\usepackage{amsmath}
\usepackage{amsfonts}
\usepackage{amssymb}
\usepackage[version=4]{mhchem}
\usepackage{stmaryrd}
\usepackage{bbold}

\begin{document}
\subsection*{2.7.2 Zweite Richtungsableitung}
Wir betrachten den folgenden Satz.\\
Satz 2.26 Hesse-Formel in nD\\
Seien $n \in \mathbb{N}^{+}, f: \mathbb{R}^{n} \rightarrow \mathbb{R}$ eine zweifach differentierbare Funktion mit Hesse-Matrix $H:=\boldsymbol{\nabla}^{2} f$ und $\hat{\mathbf{v}}, \hat{\mathbf{w}} \in \mathbb{R}^{n}$ zwei Einheitsvektoren, dann gilt


\begin{equation*}
\nabla_{\hat{\mathbf{w}} \hat{\mathbf{v}}}^{2} f=\nabla_{\hat{\mathbf{w}}}\left(\nabla_{\hat{\mathbf{v}}} f\right)=\langle\hat{\mathbf{w}}, H \cdot \hat{\mathbf{v}}\rangle . \tag{2.200}
\end{equation*}


Beweis: Siehe Übungen.\\
Bemerkungen:\\
i) Für $\hat{\mathbf{w}}=\hat{\mathbf{v}}$ folgt


\begin{equation*}
\nabla_{\hat{\mathbf{v}} \hat{\mathbf{v}}}^{2} f=\langle\hat{\mathbf{v}}, H \cdot \hat{\mathbf{v}}\rangle . \tag{2.201}
\end{equation*}


Dies ist eine quadratische Form in $\hat{\mathbf{v}}$ mit $H$ als Matrix.\\
ii) Wegen der Symmetrien von $H$ und des Skalar-Produkts kommutieren die zweiten Richtungsableiten, d.h. es gilt


\begin{equation*}
\nabla_{\hat{\mathrm{v}} \hat{\mathrm{w}}}^{2} f=\nabla_{\hat{\mathrm{w}} \hat{\mathrm{v}}}^{2} f . \tag{2.202}
\end{equation*}


iii) Für die Standard-Einheitsvektoren $\hat{\mathbf{e}}_{1}, \ldots, \hat{\mathbf{e}}_{n} \in \mathbb{R}^{n}$ entlang der Koordinatenachsen gilt


\begin{equation*}
\nabla_{\hat{e}_{\mu} \hat{e}_{\nu}}^{2} f=f_{, \nu, \mu}=H_{\nu \mu} . \tag{2.203}
\end{equation*}


iv) Die zweiten Richtungsableitungen verschwinden genau dann für alle Richtungen, wenn gilt $H=0$.

\subsection*{2.7.3 Lokale Extrema}
Wir betrachten die folgende Definition.\\
Definition 2.27 Kritische Stelle\\
Seien $n \in \mathbb{N}^{+}$und $f: \mathbb{R}^{n} \rightarrow \mathbb{R}$ differentierbar, dann heisst ein Punkt $P \in \mathbb{R}^{n}$ kritische Stelle von $f$, wenn gilt


\begin{equation*}
\boldsymbol{\nabla} f(P)=0 . \tag{2.204}
\end{equation*}


Bemerkungen:\\
i) Zur Bestimmung der kritischen Stellen von $f$ muss das Gleichungssystem (2.204) gelöst werden.\\
ii) An einer kritischen Stelle einer Funktion kann sich ein Tief-Punkt, ein Hoch-Punkt oder ein Sattel-Punkt befinden.\\
iii) Für $n=2$ verläuft die Tangentialebene an den Graphen von $f$ an einer kritischen Stelle horizontal.


\end{document}