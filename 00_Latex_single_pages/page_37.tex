\documentclass[10pt]{article}
\usepackage[ngerman]{babel}
\usepackage[utf8]{inputenc}
\usepackage[T1]{fontenc}
\usepackage{graphicx}
\usepackage[export]{adjustbox}
\graphicspath{ {./images/} }
\usepackage{amsmath}
\usepackage{amsfonts}
\usepackage{amssymb}
\usepackage[version=4]{mhchem}
\usepackage{stmaryrd}
\usepackage{bbold}

\begin{document}
\subsection*{2.4.3 Flussintegrale}
Wir betrachten eine parametrisierte Fläche im Bereich eines Vektorfeldes. Die Situation ist in der folgenden Skizze dargestellt.\\
\includegraphics[max width=\textwidth, center]{2025_05_07_fd5ff795e29286569815g-1}

Wir betrachten die folgende Definition.

\section*{Definition 2.17 Fluss eines Vektorfeldes}
Seien $M$ eine parametrisierte Fläche mit Einheitsnormalen-Vektor $\hat{\mathbf{n}}$ und $\mathbf{v}: \mathbb{R}^{3} \rightarrow \mathbb{R}^{3}$ ein integrierbares Vektorfeld. Der Fluss des Vektorfeldes v durch die Fläche M ist


\begin{equation*}
\Phi:=\int_{M}\langle\mathbf{v}, \hat{\mathbf{n}}\rangle \mathrm{d} A . \tag{2.121}
\end{equation*}


Bemerkungen:\\
i) Die Begriffe Flux, Fluss und Flussintegral sind synonym.\\
ii) Für die Masseinheit erhalten wir


\begin{equation*}
[\Phi]=[\langle\mathbf{v}, \hat{\mathbf{n}}\rangle] \cdot[A]=[\mathbf{v}] \cdot[\hat{\mathbf{n}}] \cdot[A]=[\mathbf{v}] \cdot 1 \cdot[A]=\underline{\underline{\mathbf{v}}] \cdot[A]} . \tag{2.122}
\end{equation*}


iii) Man kann zeigen, dass das Flussintegral $\Phi$ bis auf das Vorzeichen nicht von der Wahl der Parametrisierung sondern nur von der Fläche $M$ abhängt (sofern mehrfache Durchläufe auch mehrfach gerechnet werden).\\
iv) Um ein Flussintegral auszurechnen, müssen die Koordinaten der Punkte entlang der parametrisierten Fläche im Vektorfeld eingesetzt werden. Vollständig ausgeschrieben mit allen Abhängigkeiten ergibt dies


\begin{equation*}
\Phi=\int_{u_{0}}^{u_{\mathrm{E}}} \int_{v_{0}}^{v_{\mathrm{E}}}\langle\mathbf{v}(x(u ; v) ; y(u ; v) ; z(u ; v)), \hat{\mathbf{n}}(u ; v)\rangle \cdot \sqrt{g(u ; v)} \mathrm{d} v \mathrm{~d} u . \tag{2.123}
\end{equation*}


v) In der Literatur findet man für Flussintegrale durch eine Fläche $M$ die Schreibweisen


\begin{equation*}
\Phi=\int_{M}\langle\mathbf{v}, \hat{\mathbf{n}}\rangle \mathrm{d} A=\int_{M} \mathbf{v} \cdot \mathrm{~d} \mathbf{A} . \tag{2.124}
\end{equation*}



\end{document}