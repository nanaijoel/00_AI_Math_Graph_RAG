\documentclass[10pt]{article}
\usepackage[ngerman]{babel}
\usepackage[utf8]{inputenc}
\usepackage[T1]{fontenc}
\usepackage{amsmath}
\usepackage{amsfonts}
\usepackage{amssymb}
\usepackage[version=4]{mhchem}
\usepackage{stmaryrd}
\usepackage{bbold}

\begin{document}
Bemerkungen:\\
i) Jeder Gradient ist wirbelfrei.\\
ii) Jede Rotation ist quellenfrei.\\
iii) In kartesischen Koordinaten ist $\Delta \mathbf{v}$ komponentenweise zu berechnen.\\
iv) Die Terme $\boldsymbol{\nabla}_{\mathbf{w}} \mathbf{v}$ und $\boldsymbol{\nabla}_{\mathbf{v}} \mathbf{w}$ bezeichnen sogenannte Richtungsableitungen (siehe nächste Abschnitte.)

\subsection*{2.5.6.2 Nabla-Operator}
Wir betrachten die folgende Definition.\\
Definition 2.25 Nabla-Operator in nD\\
Sei $n \in \mathbb{N}^{+}$, dann ist der Nabla-Operator in $\mathbb{R}^{n}$ der Differentialoperator

\[
\boldsymbol{\nabla}:=\left[\begin{array}{c}
\partial_{1}  \tag{2.159}\\
\vdots \\
\partial_{n}
\end{array}\right] .
\]

Bemerkungen:\\
i) Der Nabla-Operator ist ein abstrakter Differentialoperator, welcher erst bei seiner Anwendung auf eine Funktion bzw. ein Vektorfeld eine sinnvolle mathematische Grösse ergibt.\\
ii) Durch den Nabla-Operator können die Divergenz in nD und die Rotation in 3D durch Vektor-Operationen ausgedrückt werden. Insbesondere in älterer Literatur findet man die Schreibweisen


\begin{equation*}
\operatorname{div}(\mathbf{v})=\langle\boldsymbol{\nabla}, \mathbf{v}\rangle=\boldsymbol{\nabla} \cdot \mathbf{v} \tag{2.160}
\end{equation*}


$\operatorname{rot}(\mathbf{v})=\boldsymbol{\nabla} \times \mathbf{v}$.

\subsection*{2.5.6.3 Anwendungen}
Viele wichtige Formeln in der Physik werden durch Divergenz bzw. Rotation von Vektorfeldern ausgedrückt.

\begin{itemize}
  \item Strömungsdynamik: Beschreibt v das Geschwindigkeitsvektorfeld eines inkompressiblen Mediums (z.B. Wasser), dann ist es quellenfrei, d.h.\\
$\operatorname{div}(\mathbf{v})=0$.
  \item Elektrodynamik: Die Maxwell-Gleichungen beschreiben jeweils Divergenz und Rotation des E-Feldes und B-Feldes. Es gilt
\end{itemize}

\[
\begin{array}{l|l}
\operatorname{div}(\mathbf{E})=\frac{1}{\varepsilon_{0}} \cdot \rho & \operatorname{rot}(\mathbf{E})=-\dot{\mathbf{B}}  \tag{2.163}\\
\operatorname{div}(\mathbf{B})=0 & \operatorname{rot}(\mathbf{B})=\varepsilon_{0} \cdot \mu_{0} \cdot \dot{\mathbf{E}}+\mu_{0} \cdot \mathbf{J} .
\end{array}
\]

Der Ladungs-Erhaltungssatz kann ausgedrückt werden durch die Kontinuitätsgleichung\\
$\dot{\rho}+\operatorname{div}(\mathbf{J})=0$.


\end{document}