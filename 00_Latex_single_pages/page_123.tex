\documentclass[10pt]{article}
\usepackage[ngerman]{babel}
\usepackage[utf8]{inputenc}
\usepackage[T1]{fontenc}
\usepackage{amsmath}
\usepackage{amsfonts}
\usepackage{amssymb}
\usepackage[version=4]{mhchem}
\usepackage{stmaryrd}
\usepackage{bbold}

\begin{document}
ii) Gilt $\operatorname{det}(A)=0$, dann ist die quadratische Matrix $A$ singulär, d.h. sie hat keine Inverse und die zugehörige lineare Abbildung a ist nicht bijektiv.

Wir werden später sehen, dass der Determinante auch eine geometrische Bedeutung zukommt. In diesem Abschnitt erwähnen wir dazu nur den folgenden Satz über Determinanten von orthogonalen Matrizen.

Satz 6.19 Determinante einer orthogonalen Matrix\\
Seien $n \in \mathbb{N}^{+}$und $A \in \mathrm{O}(n)$, dann gilt $\operatorname{det}(A) \in\{-1,1\}$.\\
Beweis: Für eine orthogonale Matrix $A$ muss gelten


\begin{align*}
\operatorname{det}(A) & \left.=\operatorname{det}\left(A^{T}\right)=\operatorname{det}\left(A^{-1}\right)=\frac{1}{\operatorname{det}(A)} \quad \right\rvert\, \cdot \operatorname{det}(A)  \tag{6.126}\\
\Leftrightarrow \quad(\operatorname{det}(A))^{2} & =1 \tag{6.127}
\end{align*}


Daraus folgt


\begin{equation*}
\underline{\underline{\operatorname{det}( }(A) \in\{-1,1\} .} \tag{6.128}
\end{equation*}


Damit haben den Satz bewiesen.\\
Bemerkungen:\\
i) Die Umkehrung des Satzes 6.19 gilt nicht. Dazu betrachten wir z.B. die Matrix

\[
A=\left[\begin{array}{cc}
\frac{1}{2} & 0  \tag{6.129}\\
0 & 2
\end{array}\right] .
\]

Wir berechnen leicht, dass zwar


\begin{equation*}
\operatorname{det}(A)=\frac{1}{2} \cdot 2-0 \cdot 0=1-0=1, \tag{6.130}
\end{equation*}


aber deswegen ist $A$ noch lange nicht orthogonal, denn es gilt

\[
A^{-1}=\left[\begin{array}{cc}
2 & 0  \tag{6.131}\\
0 & \frac{1}{2}
\end{array}\right] \neq\left[\begin{array}{cc}
\frac{1}{2} & 0 \\
0 & 2
\end{array}\right]=A^{T} .
\]

ii) Die Determinante teilt die Menge $\mathrm{O}(n)$ in zwei Teile auf. Man definiert


\begin{equation*}
\mathrm{O}^{ \pm}(n):=\{A \in \mathrm{O}(n) \mid \operatorname{det}(A)= \pm 1\} . \tag{6.132}
\end{equation*}


Ferner bezeichnet man


\begin{equation*}
\mathrm{SO}(n):=\mathrm{O}^{+}(n) \tag{6.133}
\end{equation*}


als spezielle orthogonale Gruppe in $n$ Dimensionen.\\
iii) Man kann zeigen, dass $\mathrm{O}^{-}(n)$ alle Spiegelungsmatrizen und $\mathrm{O}^{+}(n)$ alle Rotationsmatrizen enthält.


\end{document}