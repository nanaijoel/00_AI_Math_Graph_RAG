\documentclass[10pt]{article}
\usepackage[ngerman]{babel}
\usepackage[utf8]{inputenc}
\usepackage[T1]{fontenc}
\usepackage{amsmath}
\usepackage{amsfonts}
\usepackage{amssymb}
\usepackage[version=4]{mhchem}
\usepackage{stmaryrd}
\usepackage{bbold}

\begin{document}
\subsection*{2.3.2 Dreifach-Integrale}
Die Theorie der Integrale in 3D ist ganz analog zur Theorie in 2D. Exemplarisch betrachten wir dazu den folgenden Satz.

Satz 2.7 Fubini-Satz für Quader\\
Seien $x_{0}, x_{\mathrm{E}}, y_{0}, y_{\mathrm{E}}, z_{0}, z_{\mathrm{E}} \in \mathbb{R}$ mit $x_{0}<x_{\mathrm{E}}, y_{0}<y_{\mathrm{E}}$ und $z_{0}<z_{\mathrm{E}}, f: \mathbb{R}^{3} \rightarrow \mathbb{R}$ eine integrierbare Funktion sowie $Q$ der Quader


\begin{equation*}
Q:=\left[x_{0}, x_{\mathrm{E}}\right] \times\left[y_{0}, y_{\mathrm{E}}\right] \times\left[z_{0}, z_{\mathrm{E}}\right] . \tag{2.90}
\end{equation*}


Dann gilt


\begin{equation*}
\int_{Q} f \mathrm{~d} V=\int_{z_{0}}^{z_{\mathrm{E}}} \int_{y_{0}}^{y_{\mathrm{E}}} \int_{x_{0}}^{x_{\mathrm{E}}} f(x ; y ; z) \mathrm{d} x \mathrm{~d} y \mathrm{~d} z . \tag{2.91}
\end{equation*}


Bemerkungen:\\
i) Durch den Fubini-Satz kann ein Integral über ein Gebiet in 3D auf die Berechnung von 3 verschachtelten Integralen in 1D zurückgeführt werden und umgekehrt.\\
ii) Bei der Anwendung des Fubini-Satzes werden die Dreifach-Integrale in der Reihenfolge von innen nach aussen berechnet.\\
iii) Bei der Integration über einen Quader kann die Integrationsreihenfolge vertauscht werden.

Integrale über Gebiete in 3D erhält man aus einem Archimedes-Cauchy-Riemann-Approximationsprozess in 3D gemäss folgenden Schritten.

S1 Lokal: Der Beitrag zur Grösse $I$ eines kleinen Volumenstücks $\delta V$ im $x-y$ - $z$-Raum ist


\begin{equation*}
\underline{\delta I} \approx \ldots \approx f(x ; y ; z) \cdot \delta V . \tag{2.92}
\end{equation*}


S2 Global: Durch Integration über das Gebiet $G$ können wir die gesamte Grösse $I$ berechnen. Wir erhalten


\begin{equation*}
\underline{\underline{I}}=\int_{G} f \mathrm{~d} V=\underline{\underline{\ldots}} \tag{2.93}
\end{equation*}


Anwendungen:

\begin{itemize}
  \item Masse: Wir betrachten einen Körpers $K$ mit Dichte $\rho: \mathbb{R}^{3} \rightarrow \mathbb{R}$.
\end{itemize}

S1 Lokal: Die Masse eines kleinen Volumenstücks $\delta V$ im $x-y$ - $z$-Raum ist


\begin{equation*}
\underline{\delta m} \approx \rho \cdot \delta V=\rho(x ; y ; z) \cdot \delta V . \tag{2.94}
\end{equation*}


S2 Global: Durch Integration über den Körper K können wir seine gesamte Masse berechnen. Wir erhalten


\begin{equation*}
m=\int_{K} \rho \mathrm{~d} V . \tag{2.95}
\end{equation*}



\end{document}