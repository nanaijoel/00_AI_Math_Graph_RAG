\documentclass[10pt]{article}
\usepackage[ngerman]{babel}
\usepackage[utf8]{inputenc}
\usepackage[T1]{fontenc}
\usepackage{amsmath}
\usepackage{amsfonts}
\usepackage{amssymb}
\usepackage[version=4]{mhchem}
\usepackage{stmaryrd}
\usepackage{bbold}

\begin{document}
\subsection*{5.4 Potenzen}
\subsection*{5.4.1 Euler-Formel}
Wir betrachten den folgenden Satz.\\
Satz 5.4 Euler-Formel\\
Sei $\varphi \in \mathbb{R}$, dann gilt


\begin{equation*}
\mathrm{e}^{\mathrm{i} \cdot \varphi}=\operatorname{cis}(\varphi)=\cos (\varphi)+\mathrm{i} \cdot \sin (\varphi) . \tag{5.28}
\end{equation*}


Beweis: Übung mit Hilfe von Maclaurin-Entwicklungen.\\
Wir betrachten den folgenden Satz.\\
Satz 5.5 Umkehrungen der Euler-Formel\\
Sei $\varphi \in \mathbb{R}$, dann gilt folgendes.\\
(a) $\sin (\varphi)=\frac{\mathrm{e}^{\mathrm{i} \cdot \varphi}-\mathrm{e}^{-\mathrm{i} \cdot \varphi}}{2 \cdot \mathrm{i}}=-\mathrm{i} \cdot \sinh (\mathrm{i} \cdot \varphi)$\\
(b) $\cos (\varphi)=\frac{\mathrm{e}^{\mathrm{i} \cdot \varphi}+\mathrm{e}^{-\mathrm{i} \cdot \varphi}}{2}=\cosh (\mathrm{i} \cdot \varphi)$

Beweis: Übung.\\
Bemerkungen:\\
i) Die Euler-Formel beschreibt die fundamentalen Zusammenhänge zwischen der natürlichen Exponentialfunktion und den trigonometrischen bzw. hyperbolischen Funktionen.\\
ii) Aus der Euler-Formel lässt sich eine einfache algebraische Beziehung zwischen den fundamentalen Zahlen in $\{0,1, \mathrm{e}, \pi, \mathrm{i}\}$ herleiten. Es gilt


\begin{equation*}
\mathrm{e}^{\mathrm{i} \cdot \pi}+1=0 . \tag{5.29}
\end{equation*}


\subsection*{5.4.2 Exponentielle Form}
Wir betrachten den folgenden Satz.

\section*{Satz 5.6 Exponentielle Form}
Seien $x, y \in \mathbb{R}, r \in \mathbb{R}^{+}$und $\varphi \in[0,2 \pi[$ oder $\varphi \in]-\pi, \pi]$ dann gibt es ein eindeutiges $z \in \mathbb{C} \backslash\{0\}$ mit


\begin{equation*}
z=x+y \cdot \mathrm{i}=r \cdot \mathrm{e}^{\mathrm{i} \cdot \varphi} . \tag{5.30}
\end{equation*}


Ferner gelten die folgenden Umrechnungsformeln.\\
(a) $x=r \cdot \cos (\varphi) \wedge y=r \cdot \sin (\varphi)$\\
(b) $r=|z|=\sqrt{x^{2}+y^{2}} \wedge \varphi=\arg (z)$


\end{document}