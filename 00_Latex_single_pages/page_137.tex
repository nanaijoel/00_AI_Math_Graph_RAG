\documentclass[10pt]{article}
\usepackage[ngerman]{babel}
\usepackage[utf8]{inputenc}
\usepackage[T1]{fontenc}
\usepackage{amsmath}
\usepackage{amsfonts}
\usepackage{amssymb}
\usepackage[version=4]{mhchem}
\usepackage{stmaryrd}
\usepackage{bbold}

\begin{document}
\subsection*{7.1.3 Unterräume}
Eine Teilmenge eines Vektorraums kann selbst wieder ein Vektorraum sein. Dazu machen wir die folgende Definition.

\section*{Definition 7.6 Unterraum}
Sei $(V, \mathbb{K},+, \cdot)$ ein Vektorraum. Eine Teilmenge $W \subseteq V$ heisst $\operatorname{Unterraum}$ von $V$, falls $(W, \mathbb{K},+, \cdot)$ ebenfalls ein Vektorraum ist.

\section*{Bemerkungen:}
i) In der Literatur werden Unterräume auch Teilräume genannt.\\
ii) Um kompakt auszudrücken, dass $W \subseteq V$ nicht nur eine Teilmenge sondern ein Unterraum von $V$ ist, verwendet man die Schreibweise


\begin{equation*}
W \leq V \tag{7.30}
\end{equation*}


iii) Jeder Unterraum eines Vektorraums $V$ muss mindestens 0 enthalten.\\
iv) Jeder Vektorraum hat zumindest sich selbst und den trivialen Vektorraum als Unterraum. Für jeden Vektorraum $V$ gilt also


\begin{equation*}
\{0\} \leq V \quad \text { und } \quad V \leq V \tag{7.31}
\end{equation*}


Mit Hilfe des folgenden Satzes kann sehr einfach getestet werden, ob eine Teilmenge eines Vektorraums ein Unterraum ist.

Satz 7.5 Test auf Unterraum\\
Sei $(V, \mathbb{K},+, \cdot)$ ein Vektorraum und $W \subseteq V$. Dann gilt


\begin{equation*}
W \leq V \Leftrightarrow \operatorname{span}(W)=W \tag{7.32}
\end{equation*}


Bemerkungen:\\
i) Eine Teilmenge eines Vektorraums ist also genau dann ein Unterraum, wenn sie abgeschlossen ist unter der Bildung von Linearkombinationen.\\
ii) Um nachzuweisen, das $W \leq V$ muss man also zeigen, dass


\begin{equation*}
\mathbf{w}_{1}, \mathbf{w}_{2} \in W, x_{1}, x_{2} \in \mathbb{K} \Rightarrow x_{1} \cdot \mathbf{w}_{1}+x_{2} \cdot \mathbf{w}_{2} \in W . \tag{7.33}
\end{equation*}


Beispiele:

\begin{itemize}
  \item Für jedes $m, n \in \mathbb{N}^{+}$mit $m \leq n$ gilt $\mathbb{K}^{m} \leq \mathbb{K}^{n}$.
  \item Für jedes $m, n \in \mathbb{N}^{+}$mit $m \leq n$ gilt $\mathcal{P}_{m}(\mathbb{R}) \leq \mathcal{P}_{n}(\mathbb{R})$.
  \item Jede Gerade in der Ebene kann als Unterraum der Ebene aufgefasst werden.
  \item Jede Gerade oder Ebene im Raum kann als Unterraum des Raumes aufgefasst werden.
\end{itemize}

\end{document}