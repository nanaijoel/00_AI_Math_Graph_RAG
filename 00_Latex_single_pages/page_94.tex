\documentclass[10pt]{article}
\usepackage[ngerman]{babel}
\usepackage[utf8]{inputenc}
\usepackage[T1]{fontenc}
\usepackage{amsmath}
\usepackage{amsfonts}
\usepackage{amssymb}
\usepackage[version=4]{mhchem}
\usepackage{stmaryrd}
\usepackage{bbold}

\begin{document}
\begin{itemize}
  \item $\left[\begin{array}{ll}1 & 2\end{array}\right] \cdot\left[\begin{array}{l}3 \\ 4\end{array}\right]=[1 \cdot 3+2 \cdot 4]=[11]$
  \item $\left[\begin{array}{l}1 \\ 2\end{array}\right] \cdot\left[\begin{array}{ll}3 & 4\end{array}\right]=\left[\begin{array}{cc}1 \cdot 3 & 1 \cdot 4 \\ 2 \cdot 3 & 2 \cdot 4\end{array}\right]=\left[\begin{array}{ll}3 & 4 \\ 6 & 8\end{array}\right]$
\end{itemize}

\subsection*{6.1.2.5 Rechnenregeln}
Für reelle Matrizen gelten einige Rechenregeln.\\
Satz 6.1 Rechenregeln für Matrizen\\
Es seien $A, B, C$ reelle Matrizen und $a, b \in \mathbb{R}$. Sofern alle Operationen gemäss den Dimensionen definiert sind, gelten die folgenden Rechenregeln.\\
(a) $A+B=B+A$\\
(g) $A \cdot(B+C)=A \cdot B+A \cdot C$\\
(b) $(A+B)+C=A+(B+C)$\\
(h) $(A+B) \cdot C=A \cdot C+B \cdot C$\\
(c) $a \cdot(A+B)=a \cdot A+a \cdot B$\\
(i) $\left(A^{T}\right)^{T}=A$\\
(d) $(a+b) \cdot A=a \cdot A+b \cdot A$\\
(j) $(A+B)^{T}=A^{T}+B^{T}$\\
(e) $(a \cdot A) \cdot B=a \cdot(A \cdot B)=A \cdot(a \cdot B)$\\
(k) $(a \cdot A)^{T}=a \cdot A^{T}$\\
(f) $(A \cdot B) \cdot C=A \cdot(B \cdot C)$\\
(I) $(A \cdot B)^{T}=B^{T} \cdot A^{T}$

Bemerkungen:\\
i) Die Matrix-Addition ist also assoziativ und kommutativ und es gilt das Distributivgesetz sowohl bei der Multiplikation mit einem Skalar als auch mit einer reellen Matrix.\\
ii) Die Matrix-Multiplikation ist im allgemeinen nicht kommutativ. Es gibt Matrizen $A$ und $B$, für welche gilt $A \cdot B=B \cdot A$ aber auch solche für die wir $A \cdot B \neq B \cdot A$ finden.\\
iii) Das Transponierte eines Matrix-Produkts ist das umgekehrte Matrix-Produkt der Transponierten Faktoren.


\end{document}