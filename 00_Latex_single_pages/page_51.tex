\documentclass[10pt]{article}
\usepackage[ngerman]{babel}
\usepackage[utf8]{inputenc}
\usepackage[T1]{fontenc}
\usepackage{amsmath}
\usepackage{amsfonts}
\usepackage{amssymb}
\usepackage[version=4]{mhchem}
\usepackage{stmaryrd}
\usepackage{bbold}

\begin{document}
\subsection*{2.6.4 Zerlegungssatz}
Wir betrachten den folgenden Satz.\\
Satz 2.23 Zerlegungssatz für Vektorfelder in 3D\\
Jedes differentierbare Vektorfeld $\mathbf{v}: \mathbb{R}^{3} \rightarrow \mathbb{R}^{3}$ lässt sich zerlegen in eine Summe aus einem wirbelfreien Vektorfeld $\mathbf{q}: \mathbb{R}^{3} \rightarrow \mathbb{R}^{3}$, einem quellenfreien Vektorfeld $\mathbf{w}: \mathbb{R}^{3} \rightarrow \mathbb{R}^{3}$ und einem homogenen Vektorfeld $\mathbf{h}: \mathbb{R}^{3} \rightarrow \mathbb{R}^{3}$ gemäss


\begin{equation*}
\mathbf{v}=\mathbf{w}+\mathbf{q}+\mathbf{h} . \tag{2.186}
\end{equation*}


Bemerkungen:\\
i) Bei einer solchen Zerlegung ist $\mathbf{w}$ ein reines Wirbelfeld und $\mathbf{q}$ ein reines Quellenfeld.\\
ii) Für jedes differentierbare Vektorfeld gibt es unendlich viele Möglichkeiten eine Zerlegung der Form (2.186) zu wählen.\\
iii) Gemäss den Potential-Sätzen hat $\mathbf{q}$ ein Skalarpotential $\phi$ und $\mathbf{w}$ ein Vektorpotential A, so dass gilt


\begin{equation*}
\mathbf{q}=\boldsymbol{\nabla} \phi \quad \text { und } \quad \mathbf{w}=\operatorname{rot}(\mathbf{A}) . \tag{2.187}
\end{equation*}


Mit Hilfe dieser Potentiale lässt sich die Zerlegung (2.186) schreiben gemäss


\begin{equation*}
\underline{\underline{\mathbf{v}}}=\mathbf{w}+\mathbf{q}+\mathbf{h}=\operatorname{rot}(\mathbf{A})+\boldsymbol{\nabla} \phi+\mathbf{h} . \tag{2.188}
\end{equation*}



\end{document}