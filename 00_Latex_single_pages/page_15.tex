\documentclass[10pt]{article}
\usepackage[ngerman]{babel}
\usepackage[utf8]{inputenc}
\usepackage[T1]{fontenc}
\usepackage{amsmath}
\usepackage{amsfonts}
\usepackage{amssymb}
\usepackage[version=4]{mhchem}
\usepackage{stmaryrd}
\usepackage{bbold}

\begin{document}
\subsection*{2.1.2.2 Eigenschaften}
Wir betrachten folgende Definition.

\section*{Definition 2.4 Einheitsvektorfeld}
Seien $n \in \mathbb{N}^{+} \backslash\{1\}, A, B \subseteq \mathbb{R}^{n}$, dann ist ein Vektorfeld $\mathbf{v}: A \rightarrow B$ ein Einheitsvektorfeld, falls für alle $p \in A$ gilt


\begin{equation*}
\mathbf{v}(p)=\hat{\mathbf{v}}(p) . \tag{2.34}
\end{equation*}


Bemerkungen:\\
i) Bei einem Einheitsvektorfeld wird an jedem Punkt $p \in A$ ein Einheitsvektor "angehängt".\\
ii) Wenn ein Vektorfeld als Einheitsvektorfeld identifiziert ist, dann wird es in der Notation sinnvollerweise durch einen Hut gekennzeichnet.\\
iii) Wie alle Einheitsvektoren, tragen auch die Vektoren eines Einheitsvektorfeldes keine Masseinheit.\\
iv) Die Bezeichnungen Einheitsvektorfeld und Richtungsvektorfeld sind synonym.

Wir betrachten folgende Definition.\\
Definition 2.5 Homogenes Vektorfeld\\
Seien $n \in \mathbb{N}^{+} \backslash\{1\}, A, B \subseteq \mathbb{R}^{n}$, dann ist ein Vektorfeld $\mathbf{v}: A \rightarrow B$ homogen, falls es ein $\mathbf{w} \in B$ gibt, so dass für alle $p \in A$ gilt


\begin{equation*}
\mathbf{v}(p)=\mathbf{w} . \tag{2.35}
\end{equation*}


Bemerkungen:\\
i) Ein homogenes Vektorfeld ist einfach ein konstantes Vektorfeld.\\
ii) Bei einem homogenen Vektorfeld wird an jedem Punkt $p \in A$ der gleiche Vektor w "angehängt".

\subsection*{2.1.2.3 Visualisierung}
Siehe Demonstration von Quiver-Plots.


\end{document}