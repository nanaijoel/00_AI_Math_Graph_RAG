\documentclass[10pt]{article}
\usepackage[ngerman]{babel}
\usepackage[utf8]{inputenc}
\usepackage[T1]{fontenc}
\usepackage{amsmath}
\usepackage{amsfonts}
\usepackage{amssymb}
\usepackage[version=4]{mhchem}
\usepackage{stmaryrd}
\usepackage{bbold}

\begin{document}
\subsection*{6.5.3 Charakteristisches Polynom}
Wir machen folgende Definition.\\
Definition 6.25 Charakteristisches Polynom\\
Seien $n \in \mathbb{N}^{+}$und $A \in \mathbb{M}(n, n, \mathbb{R})$. Die Funktion


\begin{align*}
p_{A}: & \mathbb{R}
\end{align*} \rightarrow \mathbb{R}, ~\left(\lambda \mapsto p_{A}(\lambda):=\operatorname{det}(\lambda \cdot \mathbb{1}-A)\right.


heisst charakteristisches Polynom von $A$.\\
Bemerkungen:\\
i) Der Funktionsterm von $p_{A}$ wirkt zunächst irritierend. Setzt man jedoch die Komponenten von $\mathbb{1}$ und $A$ ein und rechnet die Determinante aus, dann erhält man tatsächlich ein Polynom in $\lambda$.\\
ii) Offensichtlich gilt


\begin{align*}
p_{0}(\lambda) & =\operatorname{det}(\lambda \cdot \mathbb{1}-0)=\operatorname{det}(\lambda \cdot \mathbb{1})=\lambda^{n} \cdot \operatorname{det}(\mathbb{1})=\lambda^{n} \cdot 1=\lambda^{n}  \tag{6.157}\\
p_{\mathbb{1}}(\lambda) & =\operatorname{det}(\lambda \cdot \mathbb{1}-\mathbb{1})=\operatorname{det}((\lambda-1) \cdot \mathbb{1})=(\lambda-1)^{n} \cdot \operatorname{det}(\mathbb{1})=(\lambda-1)^{n} \cdot 1 \\
& =(\lambda-1)^{n} . \tag{6.158}
\end{align*}


Das charakteristische Polynom hat ein paar wichtige Eigenschaften.\\
Satz 6.21 Eigenschaften des charakteristischen Polynoms\\
Seien $n \in \mathbb{N}^{+}$und $A \in \mathbb{M}(n, n, \mathbb{R})$. Dann ist $p_{A}$ ein Polynom vom Grad $n$ der Form


\begin{equation*}
p_{A}(\lambda)=a_{n} \cdot \lambda^{n}+a_{n-1} \cdot \lambda^{n-1}+\ldots+a_{1} \cdot \lambda+a_{0}, \tag{6.159}
\end{equation*}


wobei in jedem Fall gilt\\
(a) $a_{n}=1$\\
(b) $a_{n-1}=-\operatorname{tr}(A)$\\
(c) $a_{0}=(-1)^{n} \cdot \operatorname{det}(A)$

Beweis: Es gilt\\
$\underline{\underline{a_{0}}}=p_{A}(0)=\operatorname{det}(0 \cdot \mathbb{1}-A)=\operatorname{det}(-A)=\underline{\underline{(-1)^{n}} \cdot \operatorname{det}(A)}$.\\
Damit haben wir die Aussage (c) bewiesen.\\
Bemerkungen:\\
i) Für eine singuläre Matrix gilt $\operatorname{det}(A)=0$ und somit $a_{0}=0$. Das charakteristische Polynom hat dann die Form


\begin{equation*}
p_{A}(\lambda)=\lambda^{n}-\operatorname{tr}(A) \cdot \lambda^{n-1}+\ldots+a_{1} \cdot \lambda=\lambda \cdot\left(\lambda^{n-1}-\operatorname{tr}(A) \cdot \lambda^{n-2}+\ldots+a_{1}\right) . \tag{6.161}
\end{equation*}


ii) Für $n=2$ ist $p_{A}$ eine quadratische Funktion und es folgt


\begin{equation*}
p_{A}(\lambda)=\lambda^{2}-\operatorname{tr}(A) \cdot \lambda+\operatorname{det}(A) . \tag{6.162}
\end{equation*}



\end{document}