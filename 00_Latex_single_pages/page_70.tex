\documentclass[10pt]{article}
\usepackage[ngerman]{babel}
\usepackage[utf8]{inputenc}
\usepackage[T1]{fontenc}
\usepackage{amsmath}
\usepackage{amsfonts}
\usepackage{amssymb}
\usepackage[version=4]{mhchem}
\usepackage{stmaryrd}
\usepackage{bbold}
\usepackage{graphicx}
\usepackage[export]{adjustbox}
\graphicspath{ {./images/} }

\begin{document}
\subsection*{3.2.2 Integration über Polstellen}
Wir betrachten die Fläche zwischen dem Graphen einer Funktion $f: \mathbb{R} \backslash\left\{x_{\mathrm{p}}\right\} \rightarrow \mathbb{R}$ und einer vertikalen Asymptote bei $x_{\mathrm{p}} \in \mathbb{R}$. Eine solche Situation ist im folgenden $x$ - $y$-Diagramm dargestellt.\\
\includegraphics[max width=\textwidth, center]{2025_05_07_bda369c65ba3d1048fcag-1}

Wir betrachten die folgende Definition.\\
Definition 3.3 Uneigentliches Integral über eine Polstelle.\\
Seien $x_{0}, x_{\mathrm{p}}, x_{\mathrm{E}} \in \mathbb{R}$ mit $x_{0} \leq x_{\mathrm{p}} \leq x_{\mathrm{E}}$ und $f: \mathbb{R} \backslash\left\{x_{\mathrm{p}}\right\} \rightarrow \mathbb{R}$ eine integrierbare Funktion. Das uneigentliche Integral von $f$ über die Polstelle $x_{\mathrm{p}}$ ist


\begin{equation*}
\int_{x_{0}}^{x_{\mathrm{E}}} f(x) \mathrm{d} x:=\lim _{r>x_{\mathrm{P}}} \int_{x_{0}}^{r} f(x) \mathrm{d} x+\lim _{s \backslash x_{\mathrm{p}}} \int_{s}^{x_{\mathrm{E}}} f(x) \mathrm{d} x, \tag{3.40}
\end{equation*}


falls beide Grenzwerte konvergieren.\\
Bemerkungen:\\
i) Das uneigentliche Integral wird als Summe von zwei Grenzwerten berechnet und existiert somit nur dann, wenn beide einzeln konvergieren.\\
ii) Falls $x_{\mathrm{p}} \in\left\{x_{0}, x_{\mathrm{E}}\right\}$, dann fällt einer der beiden Grenzwerte weg. Es gilt


\begin{align*}
& x_{\mathrm{p}}=x_{0} \Rightarrow \int_{x_{0}}^{x_{\mathrm{E}}} f(x) \mathrm{d} x=\int_{x_{\mathrm{P}}}^{x_{\mathrm{E}}} f(x) \mathrm{d} x=\lim _{s \backslash x_{\mathrm{P}}} \int_{s}^{x_{\mathrm{E}}} f(x) \mathrm{d} x  \tag{3.41}\\
& x_{\mathrm{p}}=x_{\mathrm{E}} \Rightarrow \int_{x_{0}}^{x_{\mathrm{E}}} f(x) \mathrm{d} x=\int_{x_{0}}^{x_{\mathrm{p}}} f(x) \mathrm{d} x=\lim _{r \not x_{\mathrm{P}}} \int_{x_{0}}^{r} f(x) \mathrm{d} x . \tag{3.42}
\end{align*}


iii) Befinden sich mehr als eine Polstelle im Intervall $\left[x_{0}, x_{\mathrm{E}}\right]$, dann muss das Integral in Teilabschnitte unterteilt und weitere Grenzwerte eingeführt werden.\\
iv) In der Praxis ist aus Integrand und Integrationsgrenzen nicht immer auf den ersten Blick ersichtlich, ob es sich um ein uneigentliches Integral mit Polstellen handelt, wie z.B. bei


\begin{equation*}
I=\int_{0}^{3} \frac{2 x-4}{1-\mathrm{e}^{\frac{2-x}{3}}} \mathrm{~d} x \tag{3.43}
\end{equation*}
 mit einem Pol bei $x=2$.


\end{document}