\documentclass[10pt]{article}
\usepackage[ngerman]{babel}
\usepackage[utf8]{inputenc}
\usepackage[T1]{fontenc}
\usepackage{amsmath}
\usepackage{amsfonts}
\usepackage{amssymb}
\usepackage[version=4]{mhchem}
\usepackage{stmaryrd}
\usepackage{bbold}

\begin{document}
\subsection*{2.6.3 Potential-Sätze}
Wir betrachten den folgenden Satz.\\
Satz 2.21 Skalarpotential in 3D\\
Seien $n \in\{2,3\}$ und $\mathbf{v}: \mathbb{R}^{n} \rightarrow \mathbb{R}^{n}$ ein differentierbares, wirbelfreies Vektorfeld, dann gibt es ein Skalarfeld $\phi: \mathbb{R}^{n} \rightarrow \mathbb{R}$, so dass gilt


\begin{equation*}
\mathbf{v}=\boldsymbol{\nabla} \phi . \tag{2.181}
\end{equation*}


Bemerkungen:\\
i) Jedes wirbelfreie Vektorfeld ist ein Gradientenfeld.\\
ii) Das Skalarfeld $\phi$ heisst Skalarpotential oder Potential des Vektorfeldes v.\\
iii) Das Skalarpotential ist eine Verallgemeinerung der Stammfunktion auf wirbelfreie Vektorfelder in 2D und 3D.\\
iv) Für jedes wirbelfreie Vektorfeld $\mathbf{v}$ gibt es unendlich viele Möglichkeiten ein Skalarpotential zu wählen.

Wir betrachten den folgenden Satz.\\
Satz 2.22 Vektorpotential in 3D\\
Sei $\mathbf{v}: \mathbb{R}^{3} \rightarrow \mathbb{R}^{3}$ ein differentierbares, quellenfreies Vektorfeld, dann gibt es ein Vektorfeld $\mathbf{A}: \mathbb{R}^{3} \rightarrow \mathbb{R}^{3}$, so dass gilt


\begin{equation*}
\mathbf{v}=\operatorname{rot}(\mathbf{A}) . \tag{2.182}
\end{equation*}


Bemerkungen:\\
i) Jedes quellenfrei Vektorfeld ist ein Rotationsfeld.\\
ii) Das Vektorfeld A heisst Vektorpotential des Vektorfeldes v.\\
iii) Das Vektorpotential ist eine Verallgemeinerung der Stammfunktion auf quellenfreie Vektorfelder in 3D.\\
iv) Für jedes quellenfreie Vektorfeld $\mathbf{v}$ gibt es unendlich viele Möglichkeiten ein Vektorpotential zu wählen.

Anwendungen:

\begin{itemize}
  \item Elektrodynamik: Die Maxwell-Gleichungen beschreiben jeweils Divergenz und Rotation des E-Feldes und B-Feldes. Es gilt
\end{itemize}

\[
\begin{array}{l|l}
\operatorname{div}(\mathbf{E})=\frac{1}{\varepsilon_{0}} \cdot \rho & \operatorname{rot}(\mathbf{E})=-\dot{\mathbf{B}}  \tag{2.183}\\
\operatorname{div}(\mathbf{B})=0 & \operatorname{rot}(\mathbf{B})=\varepsilon_{0} \cdot \mu_{0} \cdot \dot{\mathbf{E}}+\mu_{0} \cdot \mathbf{J} .
\end{array}
\]

In einer statischen Situation, d.h. für $\dot{\mathbf{E}}=\dot{\mathbf{B}}=0$ vereinfachen sich diese Gleichungen zu

\[
\begin{array}{l|l}
\operatorname{div}(\mathbf{E})=\frac{1}{\varepsilon_{0}} \cdot \rho & \operatorname{rot}(\mathbf{E})=0  \tag{2.184}\\
\operatorname{div}(\mathbf{B})=0 & \operatorname{rot}(\mathbf{B})=\mu_{0} \cdot \mathbf{J} .
\end{array}
\]

Weil also das E-Feld wirbelfrei und das B-Feld quellenfrei ist, gibt es entsprechend ein Skalarpotential $\phi$ und ein Vektorpotential A, so dass


\begin{equation*}
\mathbf{E}=-\boldsymbol{\nabla} \phi \quad \text { und } \quad \mathbf{B}=\operatorname{rot}(\mathbf{A}) . \tag{2.185}
\end{equation*}



\end{document}