\documentclass[10pt]{article}
\usepackage[ngerman]{babel}
\usepackage[utf8]{inputenc}
\usepackage[T1]{fontenc}
\usepackage{amsmath}
\usepackage{amsfonts}
\usepackage{amssymb}
\usepackage[version=4]{mhchem}
\usepackage{stmaryrd}
\usepackage{bbold}

\begin{document}
\subsection*{6.5.5 Spezielle Eigenwerte \& Eigenvektoren}
Für einige Abbildungsmatrizen kann man die Eigenwerte und Eigenvektoren direkt aus ihren geometrischen Eigenschaften ablesen. Wir machen dazu folgende Bemerkungen.\\
i) Es gilt $\operatorname{Spec}(0)=\{0\}$ und jedes $\mathbf{E} \in \mathbb{R}^{n}$ ist Eigenvektor der Nullmatrix.\\
ii) Es gilt $\operatorname{Spec}(\mathbb{1})=\{1\}$ und jedes $\mathbf{E} \in \mathbb{R}^{n}$ ist Eigenvektor der Einheitsmatrix.\\
iii) $A$ singulär $\Leftrightarrow 0 \in \operatorname{Spec}(A)$\\
iv) $A \in O(n) \Rightarrow \operatorname{Spec}(A) \subseteq\{-1,1\}$\\
v) $A$ eine Spiegelung $\Rightarrow-1 \in \operatorname{Spec}(A)$\\
vi) Für eine Diagonalmatrix gilt:\\
$D=\left[\begin{array}{rrlc}\lambda_{1} & 0 & \ldots & 0 \\ 0 & \lambda_{2} & \ldots & 0 \\ \vdots & \vdots & \ddots & \vdots \\ 0 & 0 & \ldots & \lambda_{n}\end{array}\right] \Rightarrow \operatorname{Spec}(D) \subseteq\left\{\lambda_{1}, \ldots, \lambda_{n}\right\}$.\\
Die Einheitseigenvektoren zu jedem Eigenwert sind gerade die Einheitsvektoren entlang der Achsen, d.h.\\
$\hat{\mathbf{E}}_{k}=\hat{\mathbf{e}}_{k}$.


\end{document}