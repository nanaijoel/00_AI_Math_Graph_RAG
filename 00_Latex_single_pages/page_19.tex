\documentclass[10pt]{article}
\usepackage[ngerman]{babel}
\usepackage[utf8]{inputenc}
\usepackage[T1]{fontenc}
\usepackage{amsmath}
\usepackage{amsfonts}
\usepackage{amssymb}
\usepackage[version=4]{mhchem}
\usepackage{stmaryrd}
\usepackage{graphicx}
\usepackage[export]{adjustbox}
\graphicspath{ {./images/} }
\usepackage{bbold}

\begin{document}
ii) Man kann zeigen, dass die Bogenlänge $\Delta s$ nicht von der Wahl der Parametrisierung sondern nur von der Bahn abhängt (sofern mehrfache Durchläufe auch mehrfach gerechnet werden).\\
iii) In 2D und 3D stimmt die Bogenlänge mit dem klassischen Begriff der Weglänge aus der Elementargeometrie überein.\\
iv) Wählt man für den Kurvenparameter die geometrische Weglänge $\tau=s$ entlang der Bahn ("Kilometrierung"), dann gilt


\begin{align*}
& v(s)=1  \tag{2.51}\\
& \underline{\underline{\Delta s}}=\int_{\tau_{0}}^{\tau_{\mathrm{E}}} v(\tau) \mathrm{d} \tau=\int_{s_{0}}^{s_{\mathrm{E}}} v(s) \mathrm{d} s=\int_{s_{0}}^{s_{\mathrm{E}}} 1 \mathrm{~d} s=\underline{\underline{s_{\mathrm{E}}-s_{0}}} \tag{2.52}
\end{align*}


v) In der Praxis lässt sich das Integral in (2.49) nur in einfachen Fällen analytisch berechnen. Meistens muss auf numerische Integration zurückgegriffen werden.

\subsection*{2.2.2 Linienintegrale}
Wir betrachten eine parametrisierte Kurve im Bereich eines Vektorfeldes. Die Situation ist in der folgenden Skizze dargestellt.\\
\includegraphics[max width=\textwidth, center]{2025_05_07_dc57745355fdb137ae35g-1}

Wir betrachten die folgende Definition.\\
Definition 2.9 Linienintegral\\
Seien $n \in \mathbb{N}^{+}, \tau_{0}, \tau_{\mathrm{E}} \in \mathbb{R}$ mit $\tau_{0}<\tau_{\mathrm{E}}, \mathbf{s}:\left[\tau_{0}, \tau_{\mathrm{E}}\right] \rightarrow \mathbb{R}^{n}$ eine parametrisierte Kurve mit Geschwindigkeitsvektor $\mathbf{v}(\tau)$ und $\mathbf{w}: \mathbb{R}^{n} \rightarrow \mathbb{R}^{n}$ ein Vektorfeld. Das Linienintegral des Vektorfeldes $\mathbf{w}$ entlang der Kurve $\mathbf{s}(\tau)$ ist die reelle Zahl


\begin{equation*}
I:=\int_{\tau_{0}}^{\tau_{\mathrm{E}}}\langle\mathbf{w}, \mathbf{v}\rangle \mathrm{d} \tau . \tag{2.53}
\end{equation*}


Bemerkungen:\\
i) Die Begriffe Linienintegral und Kurvenintegral sind synonym.\\
ii) Für die Masseinheit erhalten wir


\begin{equation*}
\underline{\underline{[I]}}=[\langle\mathbf{w}, \mathbf{v}\rangle] \cdot[\tau]=[\mathbf{w}] \cdot[\mathbf{v}] \cdot[\tau]=[\mathbf{w}] \cdot \frac{[\mathbf{s}]}{[\tau]} \cdot[\tau]=\underline{\underline{[\mathbf{w}]} \cdot[\mathbf{s}]} . \tag{2.54}
\end{equation*}



\end{document}