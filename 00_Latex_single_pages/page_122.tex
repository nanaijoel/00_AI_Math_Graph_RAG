\documentclass[10pt]{article}
\usepackage[ngerman]{babel}
\usepackage[utf8]{inputenc}
\usepackage[T1]{fontenc}
\usepackage{amsmath}
\usepackage{amsfonts}
\usepackage{amssymb}
\usepackage[version=4]{mhchem}
\usepackage{stmaryrd}
\usepackage{bbold}

\begin{document}
iii) Eine quadratische Matrix $U$ heisst unimodular, falls gilt $\operatorname{det}(U)=1$. Für eine beliebige quadratische Matrix $A$ folgt dann\\
$\operatorname{det}(U \cdot A)=\operatorname{det}(U) \cdot \operatorname{det}(A)=1 \cdot \operatorname{det}(A)=\operatorname{det}(A)$.\\
Ferner ist das Produkt von zwei unimodularen Matrizen wieder unimodular.\\
Gewisse einfache Modifikationen an einer Matrix führen auch zu einfachen Veränderungen ihrer Determinante.

Satz 6.16 Modifikationsregeln der Determinante\\
Sei $n \in \mathbb{N}^{+}, A \in \mathbb{M}(n, n, \mathbb{R})$ und $a \in \mathbb{R}$. Dann gelten folgende Modifikationsregeln.\\
(a) Zeilentausch $\Rightarrow \operatorname{det}(A) \mapsto-\operatorname{det}(A)$.\\
(b) Spaltentausch $\Rightarrow \operatorname{det}(A) \mapsto-\operatorname{det}(A)$.\\
(c) Multiplikation einer Zeile mit $a \Rightarrow \operatorname{det}(A) \mapsto a \cdot \operatorname{det}(A)$.\\
(d) Multiplikation einer Spalte mit $a \Rightarrow \operatorname{det}(A) \mapsto a \cdot \operatorname{det}(A)$.

Besonders nützlich ist auch die folgende Modifikationsregel.\\
Satz 6.17 Invarianz der Determinante unter einem GaUss-Schritt\\
Subtrahiert man von einer Zeile einer $n \times n$-Matrix ein beliebiges Vielfaches einer andern Zeile, dann ändert sich die Determinante der Matrix nicht.

Die Modifikationsregeln ermöglichen es, Determinanten mit Hilfe des Gauss-Verfahrens zu berechnen. Als Beispiel berechnen wir die Determinante der Matrix\\
$A=\left[\begin{array}{rrr}1 & 2 & -3 \\ 2 & -4 & 1 \\ 2 & 2 & 8\end{array}\right]$.\\
Wir erhalten

\[
\underline{\underline{\operatorname{det}(A)}}\left|\begin{array}{rrr}1 & 2 & -3  \tag{6.125}\\ 2 & -4 & 1 \\ 2 & 2 & 8\end{array}\right|=2\left|\begin{array}{rrr}{[1]} & 2 & -3 \\ 2 & -4 & 1 \\ 1 & 1 & 4\end{array}\right| \cdot 2=\left|\begin{array}{rrr}{[1]} & 2 & -3 \\ 0 & -8 & 7 \\ 0 & -1 & 7\end{array}\right| \cdot 2
\]

$$
=8\left|\begin{array}{rrr}
{[1]} & 2 & -3 \\
0 & {[1]} & -7 \\
0 & 8 & -7
\end{array}\right| \cdot 2 \cdot(-1)^{3}=\left|\begin{array}{rrr}
{[1]} & 2 & -3 \\
0 & {[1]} & -7 \\
0 & 0 & {[49]}
\end{array}\right| \cdot 2 \cdot(-1)=1 \cdot 1 \cdot 49 \cdot 2 \cdot(-1)=\underline{\underline{-98} .}
$$

Eine der Hauptanwendungen von Determinanten ist die Prüfung einer quadratischen Matrix auf Singularität bzw. Regularität. Es gilt nämlich der folgende, bemerkenswerte Satz.

Satz 6.18 Regularitätssatz\\
Eine quadratische Matrix $A$ ist genau dann regulär, wenn gilt $\operatorname{det}(A) \neq 0$.\\
Bemerkungen:\\
i) Gilt $\operatorname{det}(A) \neq 0$, dann ist die quadratische Matrix $A$ regulär, d.h. sie hat eine Inverse $A^{-1}$ und die zugehörige lineare Abbildung a ist bijektiv und hat eine Umkehrabbildung $a^{-1}$.


\end{document}