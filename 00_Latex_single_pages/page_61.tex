\documentclass[10pt]{article}
\usepackage[ngerman]{babel}
\usepackage[utf8]{inputenc}
\usepackage[T1]{fontenc}
\usepackage{amsmath}
\usepackage{amsfonts}
\usepackage{amssymb}
\usepackage[version=4]{mhchem}
\usepackage{stmaryrd}
\usepackage{bbold}

\begin{document}
\section*{Kapitel 3}
\section*{Integralrechnung}
\subsection*{3.1 Integrationsmethoden}
\subsection*{3.1.1 Substitution}
Wir betrachten den folgenden Satz.

\section*{Satz 3.1 Integration durch Substitution}
Seien $f: \mathbb{R} \rightarrow \mathbb{R}$ eine integrierbare Funktion mit Stammfunktion $F: \mathbb{R} \rightarrow \mathbb{R}, u: \mathbb{R} \rightarrow \mathbb{R}$ eine differentierbare Funktion und $x_{0}, x_{\mathrm{E}} \in \mathbb{R}$ mit $x_{0}<x_{\mathrm{E}}$, dann gilt folgendes.\\
(a) $\int f(u(x)) \cdot u^{\prime}(x) \mathrm{d} x=\int f(u) \mathrm{d} u=F(u(x))+c$\\
(b) $\int_{x_{0}}^{x_{\mathrm{E}}} f(u(x)) \cdot u^{\prime}(x) \mathrm{d} x=\int_{u\left(x_{0}\right)}^{u\left(x_{\mathrm{E}}\right)} f(u) \mathrm{d} u=F\left(u\left(x_{\mathrm{E}}\right)\right)-F\left(u\left(x_{0}\right)\right)$

Beweis: Durch Anwenden der Summen-Regel und Ketten-Regel und weil $F^{\prime}=f$ erhalten wir


\begin{equation*}
(F(u(x))+c)^{\prime}=F^{\prime}(u(x)) \cdot u^{\prime}(x)+0=f(u(x)) \cdot u^{\prime}(x) . \tag{3.1}
\end{equation*}


Aus der Newton-Leibniz-Formel folgt sofort (a) und durch Einsetzen der Integrationsgrenzen erhalten wir (b). Damit haben wir den Satz bewiesen.

Beispiele:

\begin{itemize}
  \item Wir betrachten das unbestimmte Integral
\end{itemize}


\begin{equation*}
F(x)=\int x \cdot \cos \left(x^{2}\right) \mathrm{d} x . \tag{3.2}
\end{equation*}


Als Substitution wählen wir


\begin{equation*}
u(x):=x^{2} \Rightarrow u^{\prime}(x)=2 x . \tag{3.3}
\end{equation*}


Wir zeigen mehrere Varianten, um das unbestimmte Integral zu berechnen.


\end{document}