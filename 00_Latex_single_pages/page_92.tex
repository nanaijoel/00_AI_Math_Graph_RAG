\documentclass[10pt]{article}
\usepackage[ngerman]{babel}
\usepackage[utf8]{inputenc}
\usepackage[T1]{fontenc}
\usepackage{amsmath}
\usepackage{amsfonts}
\usepackage{amssymb}
\usepackage[version=4]{mhchem}
\usepackage{stmaryrd}
\usepackage{bbold}

\begin{document}
v) Beispiel-Codes zur Multiplikation und Division von Matrizen mit Skalaren mit gängiger Software.

\begin{center}
\begin{tabular}{|l|l|}
\hline
MATLAB/Octave & $\mathrm{M}=2 * \mathrm{~A} ; \mathrm{M}=\mathrm{B} / 3$; \\
\hline
Mathematica/WolframAlpha & $\mathrm{M}=2 * \mathrm{~A}$; $\mathrm{M}=\mathrm{B} / 3$; \\
\hline
Python/Numpy & $\mathrm{M}=2 * \mathrm{~A} ; \mathrm{M}=\mathrm{B} / 3$; \\
\hline
Python/Sympy & $\mathrm{M}=2 * \mathrm{~A} ; \mathrm{M}=\mathrm{B} / 3$; \\
\hline
\end{tabular}
\end{center}

Beispiele:

\begin{itemize}
  \item $2 \cdot\left[\begin{array}{rr}2 & -1 \\ 7 & 5\end{array}\right]=\left[\begin{array}{rr}4 & -2 \\ 14 & 10\end{array}\right]$
  \item $\frac{1}{3} \cdot\left[\begin{array}{rr}9 & -12 \\ 3 & 0\end{array}\right]=\left[\begin{array}{rr}3 & -4 \\ 1 & 0\end{array}\right]$
  \item $(-1) \cdot\left[\begin{array}{ll}3 & -3\end{array}\right]=\left[\begin{array}{ll}-3 & 3\end{array}\right]$
\end{itemize}

\subsection*{6.1.2.3 Transposition}
Jede reelle Matrix lässt sich transponieren.\\
Definition 6.4 Transposition\\
Seien $m, n \in \mathbb{N}^{+}$und $A \in \mathbb{M}(m, n, \mathbb{R})$, dann ist

\[
A^{T}:=\left[\begin{array}{llll}
A^{1}{ }_{1} & A^{2}{ }_{1} & \ldots & A^{m}{ }_{1}  \tag{6.9}\\
A^{1}{ }_{2} & A^{2}{ }_{2} & \ldots & A^{m}{ }_{2} \\
\vdots & \vdots & \vdots & \vdots \\
A^{1}{ }_{n} & A^{2}{ }_{n} & \ldots & A^{m}{ }_{n}
\end{array}\right] .
\]

Bemerkungen:\\
i) Beim Transponieren einer reellen Matrix werden die Spalten mit den Zeilen vertauscht.\\
ii) Ist $A \in \mathbb{M}(m, n, \mathbb{R})$, dann gilt $A^{T} \in \mathbb{M}(n, m, \mathbb{R})$.\\
iii) Die Transposition ist eine Involution, d.h. für jede Matrix $A$ gilt


\begin{equation*}
\left(A^{T}\right)^{T}=A \tag{6.10}
\end{equation*}


iv) Beispiel-Codes zur Berechnung von Matrix-Transpositionen mit gängiger Software.

\begin{center}
\begin{tabular}{|l|l|}
\hline
MATLAB/Octave & $M=A^{\prime}$ \\
\hline
Mathematica/WolframAlpha & M=Transpose [A] \\
\hline
Python/Numpy & $M=A . T$ \\
\hline
Python/Sympy & M=A.T \\
\hline
\end{tabular}
\end{center}


\end{document}