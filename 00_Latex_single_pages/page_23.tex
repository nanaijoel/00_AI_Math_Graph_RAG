\documentclass[10pt]{article}
\usepackage[ngerman]{babel}
\usepackage[utf8]{inputenc}
\usepackage[T1]{fontenc}
\usepackage{amsmath}
\usepackage{amsfonts}
\usepackage{amssymb}
\usepackage[version=4]{mhchem}
\usepackage{stmaryrd}
\usepackage{bbold}

\begin{document}
iv) In der Literatur findet man auch die alternativen Schreibweisen


\begin{align*}
I & =\int_{G} f \mathrm{~d} A=\int_{G} f \mathrm{~d} G=\int_{G} f \mathrm{~d}^{2}=\int_{G} f \mathrm{~d} x^{2}=\int_{G} f \mathrm{~d}^{2} x=\int_{G} f \mathrm{~d} \mu \\
& =\iint_{G} f \mathrm{~d} A=\iint_{G} f \mathrm{~d} G=\iint_{G} f \mathrm{~d} \mathbb{R}^{2}=\iint_{G} f \mathrm{~d} x^{2}=\iint_{G} f \mathrm{~d}^{2} x=\iint_{G} f \mathrm{~d} \mu \\
& =\int_{G} f(x ; y) \mathrm{d} x \mathrm{~d} y=\iint_{G} f(x ; y) \mathrm{d} x \mathrm{~d} y . \tag{2.66}
\end{align*}


v) Integrale über Gebiete in 2D erhält man aus einem Archimedes-Cauchy-RiemannApproximationsprozess in 2D gemäss folgenden Schritten.

S1 Lokal: Der Beitrag zur Grösse $I$ eines kleinen Flächenstücks $\delta A$ auf der $x-y$-Ebene ist


\begin{equation*}
\underline{\delta I} \approx \ldots \approx \underline{f(x ; y) \cdot \delta A .} \tag{2.67}
\end{equation*}


S2 Global: Durch Integration über das Gebiet $G$ können wir die gesamte Grösse $I$ berechnen. Wir erhalten


\begin{equation*}
\underline{\underline{I}}=\int_{G} f \mathrm{~d} A=\underline{\underline{\ldots}} \tag{2.68}
\end{equation*}


\subsection*{2.3.1.2 Elementare Rechenregeln}
Wie in 1D ist die Linearität die wichtigste Eigenschaft des Integrals. Wir betrachten dazu den folgenden Satz.

Satz 2.2 Linearität des Integrals in 2D\\
Seien $G \subset \mathbb{R}^{2}$ ein Gebiet, $g, h: \mathbb{R}^{2} \rightarrow \mathbb{R}$ integrierbare Funktionen und $a, b \in \mathbb{R}$. Dann gelten die folgenden Rechenregeln.\\
(a) Faktor-Regel:


\begin{equation*}
\int_{G} a \cdot g \mathrm{~d} A=a \int_{G} g \mathrm{~d} A . \tag{2.69}
\end{equation*}


(b) Summen-Regel:


\begin{equation*}
\int_{G}(g+h) \mathrm{d} A=\int_{G} g \mathrm{~d} A+\int_{G} h \mathrm{~d} A . \tag{2.70}
\end{equation*}


(c) Linearität:


\begin{equation*}
\int_{G}(a \cdot g+b \cdot h) \mathrm{d} x=a \int_{G} g \mathrm{~d} A+b \int_{G} h \mathrm{~d} A . \tag{2.71}
\end{equation*}


Wie in 1D kann auch in 2D ein Integral in mehrere zerlegt werden. Dazu betrachten wir den folgenden Satz.


\end{document}