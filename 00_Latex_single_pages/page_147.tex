\documentclass[10pt]{article}
\usepackage[ngerman]{babel}
\usepackage[utf8]{inputenc}
\usepackage[T1]{fontenc}
\usepackage{amsmath}
\usepackage{amsfonts}
\usepackage{amssymb}
\usepackage[version=4]{mhchem}
\usepackage{stmaryrd}
\usepackage{bbold}

\begin{document}
Wir betrachten den folgenden Satz.\\
Satz 7.12 Metrische Skalar-Produkt-Formel\\
Seien $(V, \mathbb{R},+, \cdot)$ ein reeller Vektorraum mit endlicher Dimension $n \in \mathbb{N}^{+}$und Skalar-Produkt $\langle.,$.$\rangle und B=\left\{\mathbf{e}_{1}, \ldots, \mathbf{e}_{n}\right\} \subset V$ eine Basis von $V$ mit Metrik $g \in \mathbb{M}(n, n, \mathbb{R})$ und $\mathbf{v}, \mathbf{w} \in V$ mit Basis-Darstellungen


\begin{equation*}
\mathbf{v}=\sum_{r=1}^{n} v^{r} \cdot \mathbf{e}_{r} \quad \text { bzw. } \quad \mathbf{w}=\sum_{s=1}^{n} w^{s} \cdot \mathbf{e}_{s} \tag{7.96}
\end{equation*}


Dann gilt

\[
\langle\mathbf{v}, \mathbf{w}\rangle=\left[\begin{array}{lll}
v^{1} & \ldots & v^{n}
\end{array}\right] \cdot\left[\begin{array}{ccc}
g_{11} & \ldots & g_{1 n}  \tag{7.97}\\
\vdots & \vdots & \vdots \\
g_{n 1} & \cdots & g_{n n}
\end{array}\right] \cdot\left[\begin{array}{c}
w^{1} \\
\vdots \\
w^{n}
\end{array}\right]=\mathbf{v}^{T} \cdot g \cdot \mathbf{w} .
\]

Beweis: Durch Einsetzen der Basis-Darstellungen und mit Hilfe der Bilinearität des SkalarProdukts erhalten wir


\begin{align*}
\underline{\underline{\mathbf{v}, \mathbf{w}\rangle}} & =\left\langle\sum_{r=1}^{n} v^{r} \cdot \mathbf{e}_{r}, \sum_{s=1}^{n} w^{s} \cdot \mathbf{e}_{s}\right\rangle=\sum_{r=1}^{n} \sum_{s=1}^{n} v^{r} \cdot w^{s} \cdot\left\langle\mathbf{e}_{r}, \mathbf{e}_{s}\right\rangle=\sum_{r=1}^{n} \sum_{s=1}^{n} v^{r} \cdot w^{s} \cdot g_{r s} \\
& =\sum_{r=1}^{n} v^{r} \cdot \sum_{s=1}^{n} g_{r s} \cdot w^{s}=\left[\begin{array}{lll}
v^{1} & \ldots & v^{n}
\end{array}\right] \cdot\left[\begin{array}{c}
g_{11} \cdot w^{1}+g_{12} \cdot w^{2}+\ldots+g_{1 n} \cdot w^{n} \\
\vdots \\
g_{n 1} \cdot w^{1}+g_{n 2} \cdot w^{2}+\ldots+g_{n n} \cdot w^{n}
\end{array}\right] \\
& =\left[\begin{array}{lll}
v^{1} & \ldots & v^{n}
\end{array}\right] \cdot\left[\begin{array}{ccc}
g_{11} & \ldots & g_{1 n} \\
\vdots & \vdots & \vdots \\
g_{n 1} & \ldots & g_{n n}
\end{array}\right] \cdot\left[\begin{array}{c}
w^{1} \\
\vdots \\
w^{n}
\end{array}\right]=\mathbf{v}^{T} \cdot g \cdot \mathbf{w} . \tag{7.98}
\end{align*}


Damit haben wir den Satz bewiesen.\\
Bemerkungen:\\
i) Durch die metrische Skalar-Produkt-Formel kann das Skalar-Produkt in einem beliebigen Vektorraum mit fix gewählter Basis aus den Komponenten der Vektoren als MatrixProdukt mit drei Faktoren berechnet werden.\\
ii) In einem Vektorraum kann ein Skalar-Produkt definiert werden durch Angabe einer Basis und deren Metrik.\\
iii) In $\mathbb{R}^{n}$ mit Gram-Riemann-Skalar-Produkt folgt


\begin{equation*}
\underline{\underline{\langle\mathbf{v}, \mathbf{w}\rangle}}=\mathbf{v}^{T} \cdot g \cdot \mathbf{w}=\mathbf{v}^{T} \cdot \mathbb{1} \cdot \mathbf{w}=\underline{\underline{\mathbf{v}^{T}} \cdot \mathbf{w}} \tag{7.99}
\end{equation*}



\end{document}