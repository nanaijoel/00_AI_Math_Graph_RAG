\documentclass[10pt]{article}
\usepackage[ngerman]{babel}
\usepackage[utf8]{inputenc}
\usepackage[T1]{fontenc}
\usepackage{amsmath}
\usepackage{amsfonts}
\usepackage{amssymb}
\usepackage[version=4]{mhchem}
\usepackage{stmaryrd}
\usepackage{bbold}

\begin{document}
\subsection*{2.5 Mehrfach-Differentiale}
\subsection*{2.5.1 Partielle Ableitungen}
Wir betrachten die folgende Definition.

\section*{Definition 2.18 Partielle Ableitung}
Seien $n \in \mathbb{N}^{+}$und $f: \mathbb{R}^{n} \rightarrow \mathbb{R}$ eine reellwertige Funktion. Die partiellen Ableitungen von $f$ sind die Ableitungen von $f$ nach jeweils einer der $n$ Variablen, wobei die anderen als Konstanten betrachtet werden.

Bemerkungen:\\
i) Eine reellwertige Funktion heisst differntierbar, wenn alle partiellen Ableitungen existieren und stetig sind.\\
ii) Wie die Ableitung in 1D können auch die partiellen Ableitungen in nD mit Hilfe des Newton-Differenzenquotienten definiert werden gemäss


\begin{equation*}
f_{, \mu}\left(x_{1} ; x_{2} ; \ldots ; x_{n}\right):=\lim _{\delta s \rightarrow 0} \frac{f\left(x_{1} ; x_{2} ; \ldots ; x_{\mu}+\delta s ; \ldots ; x_{n}\right)-f\left(x_{1} ; x_{2} ; \ldots ; x_{n}\right)}{\delta s} . \tag{2.134}
\end{equation*}


iii) Für die Masseinheit erhalten wir


\begin{equation*}
\left[f_{, \mu}\right]=\frac{[f]}{\left[x_{\mu}\right]} . \tag{2.135}
\end{equation*}


iv) Die partiellen Ableitungen beschreiben an jedem Punkt die Steigungen des Funktionsgraphen in Richtung der Koordinatenachsen.\\
v) In der Literatur sind folgende Schreibweisen gebräuchlich


\begin{equation*}
f_{, \mu}=f_{, x_{\mu}}=f_{x_{\mu}}=\frac{\partial f}{\partial x_{\mu}}=\frac{\partial}{\partial x_{\mu}} f=\partial_{\mu} f . \tag{2.136}
\end{equation*}


\subsection*{2.5.2 Gradient}
Wir betrachten die folgende Definition.\\
Definition 2.19 Gradient\\
Seien $n \in \mathbb{N}^{+}$und $f: \mathbb{R}^{n} \rightarrow \mathbb{R}$ eine differentierbare reellwertige Funktion. Der Gradient von $f$ ist das Vektorfeld

\[
\boldsymbol{\nabla} f:=\left[\begin{array}{l}
f_{, 1}  \tag{2.137}\\
f_{, 2} \\
\vdots \\
f_{, n}
\end{array}\right]
\]

Bemerkungen:\\
i) Der Gradient ist eine allgemeine Konstruktion in nD.


\end{document}