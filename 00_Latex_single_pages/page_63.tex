\documentclass[10pt]{article}
\usepackage[ngerman]{babel}
\usepackage[utf8]{inputenc}
\usepackage[T1]{fontenc}
\usepackage{amsmath}
\usepackage{amsfonts}
\usepackage{amssymb}
\usepackage[version=4]{mhchem}
\usepackage{stmaryrd}
\usepackage{graphicx}
\usepackage[export]{adjustbox}
\graphicspath{ {./images/} }

\begin{document}
und somit


\begin{align*}
\underline{\underline{I}} & =\int_{0}^{\frac{\pi}{2}} \sqrt{1-\sin ^{2}(\varphi)} \cdot \cos (\varphi) \mathrm{d} \varphi=\int_{0}^{\frac{\pi}{2}} \sqrt{\cos ^{2}(\varphi)} \cdot \cos (\varphi) \mathrm{d} \varphi=\int_{0}^{\frac{\pi}{2}}|\cos (\varphi)| \cdot \cos (\varphi) \mathrm{d} \varphi \\
& =\int_{0}^{\frac{\pi}{2}} \cos (\varphi) \cdot \cos (\varphi) \mathrm{d} \varphi=\int_{0}^{\frac{\pi}{2}} \cos ^{2}(\varphi) \mathrm{d} \varphi=\ldots=\frac{\pi}{\underline{4}} \tag{3.14}
\end{align*}


Das verbleibende Integral ist ein Standard-Integral, dass durch partielle Integration berechnet werden kann.

Bemerkungen:\\
i) Die Idee hinter der Integration durch Substitution ist die Umkehrung der Ketten-Regel aus der Differentialrechnung.\\
ii) Durch Anwenden der Substitution kann eine schwierige Integration auf eine einfachere Integration zurückgeführt werden.\\
iii) Der Begriff Substitution bedeutet "Ersetzung".\\
iv) Bei der Substitution wird die unabhängige Variable gewechselt gemäss


\begin{equation*}
x \mapsto u \quad \text { oder } \quad u \mapsto x . \tag{3.15}
\end{equation*}


v) Das Kalkulieren mit den Differentialsymbolen ist analog zu den Berechnungsschritten in einem Archimedes-Cauchy-Riemann-Approximationsprozess gemäss\\
$\delta I \approx f(u(x)) \cdot u^{\prime}(x) \cdot \delta x \approx f(u(x)) \cdot \frac{\delta u}{\delta x} \cdot \delta x=f(u) \cdot \delta u$.\\
vi) Eine häufige Fehlerquelle bei der Berechnung von bestimmten Integralen durch Substitution ist das Anpassen der Integrationsgrenzen.\\
vii) In den meisten Fällen aus der Praxis (ca. 90\%) ist die Substitution $u$ eine lineare Funktion der Form\\
$u(x)=m \cdot x+q \Rightarrow \frac{\mathrm{~d} u}{\mathrm{~d} x}=u^{\prime}(x)=m \Leftrightarrow \mathrm{~d} x=\frac{1}{m} \mathrm{~d} u$.\\
Erkennt man dies vorweg, dann kann das einfachere Verfahren der linearen Modifikation durchgeführt werden.

Die Methode der Substitution hat nicht nur praktische Bedeutung bei der Berechnung von konkreten Integralen sondern auch bei der Herleitung von Integralformeln. Ein Standard-Beispiel ist die Formel für die kinetische Energie aus der Physik. Dazu betrachten wir einen Körper der Masse $m$, welcher von der Anfangsgeschwindigkeit $v_{0}$ auf die Endgeschwindigkeit $v_{\mathrm{E}}$ beschleunigt wird. Die Situation ist in der folgenden Skizze dargestellt.\\
\includegraphics[max width=\textwidth, center]{2025_05_07_38b1aac7586f7bdbab24g-1}

Wir betrachten dazu den folgenden Satz.


\end{document}