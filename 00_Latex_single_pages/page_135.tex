\documentclass[10pt]{article}
\usepackage[ngerman]{babel}
\usepackage[utf8]{inputenc}
\usepackage[T1]{fontenc}
\usepackage{amsmath}
\usepackage{amsfonts}
\usepackage{amssymb}
\usepackage[version=4]{mhchem}
\usepackage{stmaryrd}
\usepackage{bbold}

\begin{document}
Betrachten wir einen zweiten Satz Koeffizienten $\tilde{v}^{1}, \ldots, \tilde{v}^{n} \in \mathbb{K}$, so dass ebenfalls gilt


\begin{equation*}
\mathbf{v}=\sum_{k=1}^{n} \tilde{v}^{k} \cdot \mathbf{e}_{k} \tag{7.18}
\end{equation*}


dann erhalten wir\\
$0=\mathbf{v}-\mathbf{v}=\sum_{k=1}^{n} \tilde{v}^{k} \cdot \mathbf{e}_{k}-\sum_{k=1}^{n} v^{k} \cdot \mathbf{e}_{k}=\sum_{k=1}^{n}\left(\tilde{v}^{k} \cdot \mathbf{e}_{k}-v^{k} \cdot \mathbf{e}_{k}\right)=\sum_{k=1}^{n}\left(\tilde{v}^{k}-v^{k}\right) \cdot \mathbf{e}_{k}$.\\
Weil die Vektoren in $B$ gemäss Definition 7.4 auch linear unabhängig sind, folgt für jedes $k \in\{1, \ldots, n\}$, dass


\begin{align*}
& \tilde{v}^{k}-v^{k} & =0 & \mid+v^{k}  \tag{7.20}\\
\Leftrightarrow & \tilde{v}^{k} & =v^{k} . & \tag{7.21}
\end{align*}


Wir haben also sowohl die Existenz als auch die Eindeutigkeit der Koeffizienten $v^{1}, \ldots, v^{n} \in \mathbb{K}$ gezeigt und damit den Satz bewiesen.

Nicht jeder Vektorraum kann durch endlich viele Vektoren erzeugt werden. Betrachtet man hingegen auch Mengen aus unendlich vielen Vektoren, dann kann man in jedem nichttrivialen Vektorraum eine Basis finden.

Satz 7.3 Existenz einer Basis\\
In jedem nichttrivialen Vektorraum existiert eine Basis entweder aus endlich oder unendlich vielen Vektoren.

Von besonderem Interesse ist der folgende Satz über die Anzahl Vektoren in einer Basis.\\
Satz 7.4 Eindeutigkeit der Anzahl Basis-Vektoren\\
Jede Basis eines Vektorraums besteht aus der gleichen Anzahl Basis-Vektoren.\\
Die Tatsache, dass jede Basis eines Vektorraums aus gleich vielen Vektoren besteht, macht diese Anzahl zu einer wichtigen Kenngrösse, die einen eigenen Namen verdient.

Definition 7.5 Dimension\\
Die Anzahl Basis-Vektoren in jeder Basis eines Vektorraums heisst Dimension des Vektorraums.

Bemerkungen:\\
i) Insbesondere Funktionenräume können oft nicht endlich erzeugt werden und haben somit unendliche Dimension.\\
ii) Bemerkenswert ist die Tatsache, dass der Begriff Dimension hier rein algebraisch definiert wird. Es wird dafür keinerlei Geometrie oder gar geometrische Anschauung benötigt.\\
iii) Ist $n \in \mathbb{N}^{+}$die Dimension eines Vektorraums $(V, \mathbb{K},+, \cdot)$, dann schreibt man formell


\begin{equation*}
\operatorname{dim}(V)=n \tag{7.22}
\end{equation*}



\end{document}