\documentclass[10pt]{article}
\usepackage[ngerman]{babel}
\usepackage[utf8]{inputenc}
\usepackage[T1]{fontenc}
\usepackage{amsmath}
\usepackage{amsfonts}
\usepackage{amssymb}
\usepackage[version=4]{mhchem}
\usepackage{stmaryrd}
\usepackage{bbold}

\begin{document}
\section*{Kapitel 4}
\section*{Taylor-Entwicklungen}
\subsection*{4.1 Maclaurin-Entwicklungen}
\subsection*{4.1.1 Maclaurin-Formel}
Wir betrachten den folgenden Satz.\\
Satz 4.1 Maclaurin-Entwicklung\\
Seien $n \in \mathbb{N}, x \in \mathbb{R}$ und $f: \mathbb{R} \rightarrow \mathbb{R}$ unendlich oft differentierbar. Dann gilt


\begin{equation*}
f(x)=T_{n}(x)+R_{n}(x), \tag{4.1}
\end{equation*}


mit dem Maclaurin-Polynom $T_{n}(x)$ und Restglied $R_{n}(x)$ gemäss


\begin{align*}
& T_{n}(x)=\sum_{k=0}^{n} \frac{f^{(k)}(0)}{k!} \cdot x^{k}=f(0)+f^{\prime}(0) \cdot x+\frac{f^{\prime \prime}(0)}{2!} \cdot x^{2}+\ldots+\frac{f^{(n)}(0)}{n!} \cdot x^{n}  \tag{4.2}\\
& R_{n}(x)=\frac{(-1)^{n}}{n!} \int_{0}^{x} f^{(n+1)}(s) \cdot(s-x)^{n} \mathrm{~d} s .
\end{align*}


Beweis: Zunächst bemerken wir, dass gilt


\begin{equation*}
f(x)-f(0)=\left.[f(s)]\right|_{0} ^{x}=\int_{0}^{x} f^{\prime}(s) \mathrm{d} s \quad \mid+f(0) \tag{4.3}
\end{equation*}


Daraus und durch $n$-fache partielle Integration folgt

$$
\begin{aligned}
& \underline{\underline{f(x)}}=f(0)+\int_{0}^{x} f^{\prime}(s) \mathrm{d} s=f(0)+\int_{0}^{x} \begin{array}{c}
\downarrow \\
f^{\prime}(s) \cdot 1 \mathrm{~d} s
\end{array} \\
& \left.=f(0)+\left.\left[f^{\prime}(s) \cdot(s-x)\right]\right|_{0} ^{x}-\int_{0}^{x} \stackrel{\downarrow}{f^{\prime \prime}(s)} \cdot \stackrel{\uparrow}{\uparrow}(s) \mathrm{x}\right) \mathrm{d} s \\
& =f(0)+f^{\prime}(0) \cdot x-\left.\left[f^{\prime \prime}(s) \cdot \frac{(s-x)^{2}}{2}\right]\right|_{0} ^{x}+\int_{0}^{x} f^{\not \prime \prime \prime}(s) \cdot \frac{(s-x)^{2}}{2} \mathrm{~d} s \\
& =f(0)+f^{\prime}(0) \cdot x+\frac{f^{\prime \prime}(0)}{2} \cdot x^{2}+\left.\left[f^{\prime \prime \prime}(s) \cdot \frac{(s-x)^{3}}{2 \cdot 3}\right]\right|_{0} ^{x}-\int_{0}^{x} \begin{array}{r}
\downarrow \\
f^{(4)}(s) \\
\hline \uparrow \\
2 \cdot 3 \\
2 \cdot x)^{3} \\
\mathrm{~d} s
\end{array}
\end{aligned}
$$


\end{document}