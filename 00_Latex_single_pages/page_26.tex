\documentclass[10pt]{article}
\usepackage[ngerman]{babel}
\usepackage[utf8]{inputenc}
\usepackage[T1]{fontenc}
\usepackage{amsmath}
\usepackage{amsfonts}
\usepackage{amssymb}
\usepackage[version=4]{mhchem}
\usepackage{stmaryrd}
\usepackage{bbold}

\begin{document}
Je nach Struktur des Integranden kann die Integration über ein Rechteck weiter vereinfacht werden. Wir betrachten dazu den folgenden Satz.

Satz 2.6 Spezialfälle der Integration über ein Rechteck.\\
Seien $x_{0}, x_{\mathrm{E}}, y_{0}, y_{\mathrm{E}} \in \mathbb{R}$ mit $x_{0}<x_{\mathrm{E}}$ und $y_{0}<y_{\mathrm{E}}$ sowie $G$ das Rechteck


\begin{equation*}
G:=\left[x_{0}, x_{\mathrm{E}}\right] \times\left[y_{0}, y_{\mathrm{E}}\right] \tag{2.80}
\end{equation*}


Dann gilt folgendes.\\
(a) Konstanten-Regel: Für alle $C \in \mathbb{R}$ ist


\begin{equation*}
\int_{G} C \mathrm{~d} A=C \cdot\left(x_{\mathrm{E}}-x_{0}\right) \cdot\left(y_{\mathrm{E}}-y_{0}\right) \tag{2.81}
\end{equation*}


(b) Separation-Regel: Für zwei integrierbare Funktionen $g, h: \mathbb{R} \rightarrow \mathbb{R}$ ist


\begin{equation*}
\int_{G} g(x) \cdot h(y) \mathrm{d} A=\int_{x_{0}}^{x_{\mathrm{E}}} g(x) \mathrm{d} x \cdot \int_{y_{0}}^{y_{\mathrm{E}}} h(y) \mathrm{d} y \tag{2.82}
\end{equation*}


Beweis: Gemäss Fubini-Satz und Faktor-Regel gilt


\begin{align*}
\underline{\underline{\int_{G} g(x) \cdot h(y) \mathrm{d} A}} & =\int_{y_{0}}^{\int_{x_{0}}^{y_{\mathrm{E}}} \int_{x_{\mathrm{E}}}^{x_{\mathrm{E}}} g(x) \cdot h(y) \mathrm{d} x \mathrm{~d} y=\int_{y_{0}}^{y_{\mathrm{E}}} h(y) \cdot \int_{x_{0}}^{x_{\mathrm{E}}} g(x) \mathrm{d} x \mathrm{~d} y} \\
& =\underline{\underline{\int_{x_{0}}^{x_{\mathrm{E}}}} g(x) \mathrm{d} x \cdot \int_{y_{0}}^{y_{\mathrm{E}}} h(y) \mathrm{d} y .} \tag{2.83}
\end{align*}


Wir zeigen mehrere Varianten, um die Konstanten-Regel zu beweisen.\\
Variante 1: Gemäss Fubini-Satz und Faktor-Regel gilt


\begin{align*}
\underline{\underline{\int_{G} C \mathrm{~d} A}} & =\int_{y_{0}}^{y_{\mathrm{E}}} \int_{x_{0}}^{x_{\mathrm{E}}} C \mathrm{~d} x \mathrm{~d} y=C \int_{y_{0}}^{y_{\mathrm{E}}} \int_{x_{0}}^{x_{\mathrm{E}}} 1 \mathrm{~d} x \mathrm{~d} y=\left.C \int_{y_{0}}^{y_{\mathrm{E}}}[x]\right|_{x_{0}} ^{x_{\mathrm{E}}} \mathrm{~d} y=C \int_{y_{0}}^{y_{\mathrm{E}}}\left(x_{\mathrm{E}}-x_{0}\right) \mathrm{d} y \\
& =C \cdot\left(x_{\mathrm{E}}-x_{0}\right) \int_{y_{0}}^{y_{\mathrm{E}}} 1 \mathrm{~d} y=\left.C \cdot\left(x_{\mathrm{E}}-x_{0}\right) \cdot[y]\right|_{y_{0}} ^{y_{\mathrm{E}}} \\
& =\underline{\underline{C \cdot\left(x_{\mathrm{E}}-x_{0}\right) \cdot\left(y_{\mathrm{E}}-y_{0}\right) .}} \tag{2.84}
\end{align*}


Variante 2: Gemäss Faktor-Regel und Separation-Regel gilt


\begin{align*}
\underline{\underline{\int_{G} C \mathrm{~d} A}} & =C \int_{G} 1 \mathrm{~d} A=C \int_{G} 1 \cdot 1 \mathrm{~d} A=C \int_{x_{0}}^{x_{\mathrm{E}}} 1 \mathrm{~d} x \cdot \int_{y_{0}}^{y_{\mathrm{E}}} 1 \mathrm{~d} y=\left.\left.C \cdot[x]\right|_{x_{0}} ^{x_{\mathrm{E}}} \cdot[y]\right|_{y_{0}} ^{y_{\mathrm{E}}} \\
& =\underline{\underline{C \cdot\left(x_{\mathrm{E}}-x_{0}\right) \cdot\left(y_{\mathrm{E}}-y_{0}\right)}} \tag{2.85}
\end{align*}


Variante 3: Gemäss Faktor-Regel und Flächensatz gilt


\begin{equation*}
\underline{\underline{\int_{G} C \mathrm{~d} A}}=C \int_{G} 1 \mathrm{~d} A=C \cdot A=C \cdot \Delta x \cdot \Delta y=\underline{\underline{C \cdot\left(x_{\mathrm{E}}-x_{0}\right) \cdot\left(y_{\mathrm{E}}-y_{0}\right)}} \tag{2.86}
\end{equation*}


Damit haben wir den Satz bewiesen.


\end{document}