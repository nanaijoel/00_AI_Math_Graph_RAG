\documentclass[10pt]{article}
\usepackage[ngerman]{babel}
\usepackage[utf8]{inputenc}
\usepackage[T1]{fontenc}
\usepackage{amsmath}
\usepackage{amsfonts}
\usepackage{amssymb}
\usepackage[version=4]{mhchem}
\usepackage{stmaryrd}
\usepackage{bbold}
\usepackage{graphicx}
\usepackage[export]{adjustbox}
\graphicspath{ {./images/} }

\begin{document}
\subsection*{2.2.1.2 Standard-Parametrisierungen}
Die folgenden Parametrisierungen werden sehr häufig verwendet.

\begin{enumerate}
  \item Für Geraden und Strecken in $\mathbb{R}^{n}$.
\end{enumerate}


\begin{align*}
& \mathbf{s}(\tau)=\mathbf{s}_{0}+\mathbf{v} \cdot \tau  \tag{2.43}\\
& \mathbf{s}(\tau)=\mathbf{s}_{0}+\mathbf{v} \cdot\left(\tau-\tau_{0}\right)  \tag{2.44}\\
& \mathbf{s}(\tau)=\mathbf{s}_{0}+\left(\mathbf{s}_{\mathrm{E}}-\mathbf{s}_{0}\right) \cdot \tau \quad \text { für } \quad \tau \in[0,1] \tag{2.45}
\end{align*}


\begin{enumerate}
  \setcounter{enumi}{1}
  \item Für einen Kreis in $\mathbb{R}^{2}$ mit Mittelpunkt $M$ und Radius $r>0$.
\end{enumerate}


\begin{align*}
& \mathbf{s}(\tau)=\mathbf{M}+\left[\begin{array}{l}
r \cdot \cos (\tau) \\
r \cdot \sin (\tau)
\end{array}\right] \text { für } \tau \in[0,2 \pi]  \tag{2.46}\\
& \mathbf{s}(\tau)=\mathbf{M}+\left[\begin{array}{c}
r \cdot \cos (\omega \cdot \tau) \\
r \cdot \sin (\omega \cdot \tau)
\end{array}\right] \text { für } \tau \in[0, T] \quad \text { mit } \omega=\frac{2 \pi}{T} \tag{2.47}
\end{align*}


\begin{enumerate}
  \setcounter{enumi}{2}
  \item Für den Graph einer Funktion $f:\left[x_{0}, x_{\mathrm{E}}\right] \rightarrow \mathbb{R}$.
\end{enumerate}

\[
\mathbf{s}(\tau)=\left[\begin{array}{c}
\tau  \tag{2.48}\\
f(\tau)
\end{array}\right] \quad \text { für } \quad \tau \in\left[x_{0}, x_{\mathrm{E}}\right]
\]

Die Situation ist im folgenden $x$ - $y$-Diagramm dargestellt.\\
\includegraphics[max width=\textwidth, center]{2025_05_07_1282dab7562396bc85c1g-1}

\subsection*{2.2.1.3 Bogenlänge}
Wir betrachten die folgende Definition.

\section*{Definition 2.8 Bogenlänge}
Seien $n \in \mathbb{N}^{+}, \tau_{0}, \tau_{\mathrm{E}} \in \mathbb{R}$ mit $\tau_{0}<\tau_{\mathrm{E}}$ und $\mathbf{s}:\left[\tau_{0}, \tau_{\mathrm{E}}\right] \rightarrow \mathbb{R}^{n}$ eine parametrisierte Kurve mit Bahngeschwindigkeit $v(\tau)$. Die Bogenlänge der parametrisierten Kurve ist die reelle Zahl


\begin{equation*}
\Delta s:=\int_{\tau_{0}}^{\tau_{\mathrm{E}}} v(\tau) \mathrm{d} \tau . \tag{2.49}
\end{equation*}


Bemerkungen:\\
i) Für die Masseinheit erhalten wir


\begin{equation*}
\underline{\underline{[\Delta s]}}=[v] \cdot[\tau]=[\mathbf{v}] \cdot[\tau]=\frac{[\mathbf{s}]}{[\tau]} \cdot[\tau]=\underline{\underline{[\mathbf{s}]}} . \tag{2.50}
\end{equation*}



\end{document}