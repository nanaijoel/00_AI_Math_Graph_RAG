\documentclass[10pt]{article}
\usepackage[ngerman]{babel}
\usepackage[utf8]{inputenc}
\usepackage[T1]{fontenc}
\usepackage{graphicx}
\usepackage[export]{adjustbox}
\graphicspath{ {./images/} }
\usepackage{amsmath}
\usepackage{amsfonts}
\usepackage{amssymb}
\usepackage[version=4]{mhchem}
\usepackage{stmaryrd}
\usepackage{bbold}

\begin{document}
\begin{itemize}
  \item Die Level-Mengen der Temperatur werden in der Thermographie durch eine Farbskala dargestellt.\\
\includegraphics[max width=\textwidth, center]{2025_05_07_a0cf7425ae259516461fg-1(1)}
\end{itemize}

\subsection*{2.1.1.3 Visualisierung}
Es gibt mehrere Möglichkeiten, um eine reellwertige Funktion in mehreren reellen Variablen zu visualisieren.

\begin{enumerate}
  \item Funktionsgraph: Die Funktionswerte werden in eine weitere Dimension aufgetragen. Wir zeigen eine Beispielskizze für eine Funktion $f: \mathbb{R}^{2} \rightarrow \mathbb{R}$.\\
\includegraphics[max width=\textwidth, center]{2025_05_07_a0cf7425ae259516461fg-1}
\end{enumerate}

Anwendungen: Flächen-Plot.\\
2. Farb-Plot: Die Funktionswerte werden durch unterschiedliche Farben gekennzeichnet.

Anwendungen: Thematische Landkarten, Thermographie.


\end{document}