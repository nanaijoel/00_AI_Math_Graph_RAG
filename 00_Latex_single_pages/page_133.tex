\documentclass[10pt]{article}
\usepackage[ngerman]{babel}
\usepackage[utf8]{inputenc}
\usepackage[T1]{fontenc}
\usepackage{amsmath}
\usepackage{amsfonts}
\usepackage{amssymb}
\usepackage[version=4]{mhchem}
\usepackage{stmaryrd}
\usepackage{bbold}

\begin{document}
\begin{itemize}
  \item Für $n \in \mathbb{N}^{+}$die rationalen und komplexen Euklid-Räume ( $\left.\mathbb{Q}^{n}, \mathbb{Q},+, \cdot\right)$ bzw. ( $\left.\mathbb{C}^{n}, \mathbb{C},+, \cdot\right)$.
  \item Für $n, m \in \mathbb{N}^{+}$die Matrix-Räume $(\mathbb{M}(m, n, \mathbb{K}), \mathbb{K},+, \cdot)$ aus Matrizen mit Komponenten aus dem Zahlenkörper $\mathbb{K}$.
  \item In der Geometrie lässt sich jede Gerade, Ebene, Hyperebene, etc.. als Vektorraum beschreiben.
  \item Für $n \in \mathbb{N}^{+}$der Funktionenraum der Polynome mit reellen Koefizienten vom Grad kleiner oder gleich $n$, d.h. $\left(\mathcal{P}_{n}(\mathbb{R}), \mathbb{R},+, \cdot\right)$ mit\\
$\mathcal{P}_{n}(\mathbb{R}):=\{f: \mathbb{R} \rightarrow \mathbb{R} \mid f$ ist ein Polynom vom Grad $p \leq n\}$.
  \item Der Funktionenraum der stetigen, reellwertigen Funktionen, d.h. $(C(\mathbb{R}), \mathbb{R},+, \cdot)$ mit\\
$C(\mathbb{R}):=\{f: \mathbb{R} \rightarrow \mathbb{R} \mid f$ ist stetig $\}$.
  \item Für $p \in \mathbb{N}^{+}$die Lebesgue-Funktionenräume, d.h. $\left(\mathcal{L}^{p}(\mathbb{R}), \mathbb{R},+, \cdot\right)$ mit\\
$\mathcal{L}^{p}(\mathbb{R}):=\left\{f: \mathbb{R} \rightarrow \mathbb{R} \mid f\right.$ ist integrierbar $\left.\wedge \int_{-\infty}^{\infty}|f(x)|^{p} \mathrm{~d} x<\infty\right\}$.\\
In der Physik spielt der Lebesgue-Funktionenraum $\mathcal{L}^{2}(\mathbb{R})$ eine sehr wichtige Rolle.
\end{itemize}

\subsection*{7.1.2 Linearkombinationen, Basis \& Dimension}
Die Operationen + und • und die gemäss den Vektorraum-Axiomen dafür geforderten Rechenregeln garantieren, das sich in jedem Vektorraum auf sinnvolle Weise Linearkombinationen bilden lassen.

Definition 7.2 Linearkombination\\
Seien $(V, \mathbb{K},+, \cdot)$ ein Vektorraum, $m \in \mathbb{N}^{+},\left\{\mathbf{v}_{1}, \ldots, \mathbf{v}_{m}\right\} \subseteq V$ und $\left\{x_{1}, \ldots, x_{m}\right\} \subseteq \mathbb{K}$. Eine Linearkombination der Vektoren $\mathbf{v}_{1}, \ldots, \mathbf{v}_{m}$ ist eine Formel der Form


\begin{equation*}
\mathbf{w}=\sum_{k=1}^{m} x_{k} \cdot \mathbf{v}_{k} . \tag{7.11}
\end{equation*}


Sind Linearkombinationen erst einmal definiert, dann stellt sich die Frage, welche Teilmenge des Vektorraums durch Linearkombinieren von einigen Vektoren mit beliebigen Koeffizienten aus dem Zahlenkörper erzeugt wird.

Definition 7.3 Lineare Hülle\\
Seien $(V, \mathbb{K},+, \cdot)$ ein Vektorraum und $m \in \mathbb{N}^{+}$. Die lineare Hülle von $\left\{\mathbf{v}_{1}, \ldots, \mathbf{v}_{m}\right\} \subseteq V$ ist


\begin{equation*}
\operatorname{span}\left(\left\{\mathbf{v}_{1}, \ldots, \mathbf{v}_{m}\right\}\right):=\left\{\sum_{k=1}^{m} x_{k} \cdot \mathbf{v}_{k} \mid x_{1}, \ldots, x_{m} \in \mathbb{K}\right\} . \tag{7.12}
\end{equation*}


Bemerkungen:\\
i) In der Literatur wird die lineare Hülle auch Spann oder Spannweite genannt.


\end{document}