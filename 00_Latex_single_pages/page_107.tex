\documentclass[10pt]{article}
\usepackage[ngerman]{babel}
\usepackage[utf8]{inputenc}
\usepackage[T1]{fontenc}
\usepackage{amsmath}
\usepackage{amsfonts}
\usepackage{amssymb}
\usepackage[version=4]{mhchem}
\usepackage{stmaryrd}
\usepackage{bbold}

\begin{document}
\subsection*{6.3 Orthogonale Matrizen}
\subsection*{6.3.1 Definition}
Es gibt reguläre Matrizen, deren Inverse gerade auch ihre Transponierte ist. Dies ist eine sehr interessante Eigenschaft, die sowohl algebraische als auch geometrische Konsequenzen hat.

\section*{Definition 6.15 Orthogonale Matrix}
Sei $n \in \mathbb{N}^{+}$. Eine reguläre Matrix $A \in \mathbb{M}(n, n, \mathbb{R})$ heisst orthogonal, falls


\begin{equation*}
A^{-1}=A^{T} \text {. } \tag{6.52}
\end{equation*}


Ferner definieren wir die Menge aller orthogonalen $n \times n$-Matrizen.

\section*{Definition 6.16 Orthogonale Gruppe}
Sei $n \in \mathbb{N}^{+}$. Die orthogonale Gruppe in nD ist die Menge


\begin{equation*}
\mathrm{O}(n):=\left\{A \in \mathbb{M}(n, n, \mathbb{R}) \mid A^{-1}=A^{T}\right\} . \tag{6.53}
\end{equation*}


Wir betrachten den folgenden Satz.

\section*{Satz 6.6 Orthogonale Gruppe}
Sei $n \in \mathbb{N}^{+}$, dann bildet $\mathrm{O}(n)$ eine algebraische Gruppe, d.h. für alle $A, B, C \in \mathrm{O}(n)$ gilt folgendes.\\
(a) Endogenität:

$$
A \cdot B \in \mathrm{O}(n)
$$

(c) Neutrales Element:

$$
\mathbb{1} \in \mathrm{O}(n)
$$

(b) Assoziativität:

$$
(A \cdot B) \cdot C=A \cdot(B \cdot C)
$$

(d) Inverse Elemente:

$$
A^{-1} \in \mathrm{O}(n)
$$

Bemerkungen:\\
i) Offensichtlich gilt $\mathbb{1} \in \mathrm{O}(n)$ für alle $n \in \mathbb{N}^{+}$, denn es gilt


\begin{equation*}
\mathbb{1}^{-1}=\mathbb{1}=\mathbb{1}^{T} . \tag{6.54}
\end{equation*}


ii) Ist eine Matrix orthogonal und symmetrisch, dann folgt


\begin{equation*}
\underline{\underline{A^{-1}}}=A^{T}=\underline{\underline{A}} \Rightarrow \underline{\underline{A^{2}}}=A \cdot A=A^{-1} \cdot A=\underline{\underline{\mathbb{1}}} \tag{6.55}
\end{equation*}


Die orthogonalen symmetrischen Matrizen verhalten sich wie Wurzeln der Einheitsmatrix.\\
iii) Ist eine Matrix orthogonal und schiefsymmetrisch, dann folgt


\begin{equation*}
\underline{\underline{A^{-1}}}=A^{T}=\underline{\underline{-A}} \Rightarrow \underline{\underline{A^{2}}}=-A^{-1} \cdot A=\underline{\underline{-\mathbb{1}}} . \tag{6.56}
\end{equation*}


Die orthogonalen schiefsymmetrischen Matrizen verhalten sich ähnlich wie die imaginäre Einheit $\mathrm{i} \in \mathbb{C}$.


\end{document}