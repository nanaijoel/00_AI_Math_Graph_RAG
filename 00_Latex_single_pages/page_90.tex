\documentclass[10pt]{article}
\usepackage[ngerman]{babel}
\usepackage[utf8]{inputenc}
\usepackage[T1]{fontenc}
\usepackage{amsmath}
\usepackage{amsfonts}
\usepackage{amssymb}
\usepackage[version=4]{mhchem}
\usepackage{stmaryrd}
\usepackage{bbold}

\begin{document}
Beispiel-Codes zur Definition dieser Matrix mit gängiger Software.

\begin{center}
\begin{tabular}{|l|l|}
\hline
MATLAB/Octave & $\mathrm{M}=[1,2,3 ; 4,5,6]$ \\
\hline
Mathematica/WolframAlpha & $\mathrm{M}=\{\{1,2,3\},\{4,5,6\}\}$ \\
\hline
Python/Numpy & \begin{tabular}{l}
import numpy as np; \\
M=np.array([ $[1,2,3],[4,5,6]])$ \\
\end{tabular} \\
\hline
Python/Sympy & \begin{tabular}{l}
import sympy as sp; \\
M=sp.Matrix ([[1, 2, 3], [4, 5, 6] ]) \\
\end{tabular} \\
\hline
\end{tabular}
\end{center}

Beispiele:

\begin{itemize}
  \item Eine $2 \times 3$-Matrix: $A=\left[\begin{array}{rrr}2 & -1 & 3 \\ 7 & 5 & -4\end{array}\right]$
  \item Eine $2 \times 2$-Matrix: $B=\left[\begin{array}{rr}2 & -1 \\ 7 & 5\end{array}\right]$
  \item Eine $1 \times 3$-Matrix: $C=\left[\begin{array}{lll}2 & -1 & 3\end{array}\right]$
  \item Eine $2 \times 1$-Matrix: $D=\left[\begin{array}{l}2 \\ 7\end{array}\right]$
\end{itemize}

\subsection*{6.1.2 Operationen}
\subsection*{6.1.2.1 Addition \& Subtraktion}
Zwei reelle Matrizen mit gleichen Dimensionen lassen sich addieren und subtrahieren.\\
Definition 6.2 Addition \& Subtraktion\\
Seien $m, n \in \mathbb{N}^{+}$und $A, B \in \mathbb{M}(m, n, \mathbb{R})$, dann ist

\[
A+B:=\left[\begin{array}{llll}
A^{1}{ }_{1}+B^{1}{ }_{1} & A^{1}{ }_{2}+B^{1}{ }_{2} & \ldots & A^{1}{ }_{n}+B^{1}{ }_{n}  \tag{6.4}\\
A^{2}{ }_{1}+B^{2}{ }_{1} & A^{2}{ }_{2}+B^{2}{ }_{2} & \ldots & A^{2}{ }_{n}+B^{2}{ }_{n} \\
\vdots & \vdots & \vdots & \vdots \\
A^{m}{ }_{1}+B^{m}{ }_{1} & A^{m}{ }_{2}+B^{m}{ }_{2} & \ldots & A^{m}{ }_{n}+B^{m}{ }_{n}
\end{array}\right] .
\]

und

\[
A-B:=\left[\begin{array}{llll}
A^{1}{ }_{1}-B^{1}{ }_{1} & A^{1}{ }_{2}-B^{1}{ }_{2} & \ldots & A^{1}{ }_{n}-B^{1}{ }_{n}  \tag{6.5}\\
A^{2}{ }_{1}-B^{2}{ }_{1} & A^{2}{ }_{2}-B^{2}{ }_{2} & \ldots & A^{2}{ }_{n}-B^{2}{ }_{n} \\
\vdots & \vdots & \vdots & \vdots \\
A^{m}{ }_{1}-B^{m}{ }_{1} & A^{m}{ }_{2}-B^{m}{ }_{2} & \ldots & A^{m}{ }_{n}-B^{m}{ }_{n}
\end{array}\right] .
\]

Bemerkungen:\\
i) Für alle $A, B \in \mathbb{M}(m, n, \mathbb{R})$ gilt $A \pm B \in \mathbb{M}(m, n, \mathbb{R})$.


\end{document}