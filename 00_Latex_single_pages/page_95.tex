\documentclass[10pt]{article}
\usepackage[ngerman]{babel}
\usepackage[utf8]{inputenc}
\usepackage[T1]{fontenc}
\usepackage{amsmath}
\usepackage{amsfonts}
\usepackage{amssymb}
\usepackage[version=4]{mhchem}
\usepackage{stmaryrd}
\usepackage{bbold}

\begin{document}
\subsection*{6.1.3 Spezielle Matrizen}
\subsection*{6.1.3.1 Quadratische Matrix}
Eine ganz spezielle Rolle spielen reelle Matrizen, die gleich viele Zeilen wie Spalten haben.\\
Definition 6.6 Quadratische Matrix\\
Sei $n \in \mathbb{N}^{+}$. Eine reelle Matrix $A \in \mathbb{M}(n, n, \mathbb{R})$ heisst quadratische Matrix.

Bemerkungen:\\
i) Quadratische Matrizen haben genau $n^{2}$ Komponenten.\\
ii) Das Produkt von zwei quadratischen Matrizen ist wieder eine quadratische Matrix, d.h. $A, B \in \mathbb{M}(n, n, \mathbb{R}) \Rightarrow A \cdot B \in \mathbb{M}(n, n, \mathbb{R})$.\\
iii) Quadratische Matrizen können mit sich selbst multipliziert werden. So lassen sich Potenzen bilden. Für $n, p \in \mathbb{N}^{+}$und $A \in \mathbb{M}(n, n, \mathbb{R})$ ist


\begin{equation*}
A^{p}:=\underbrace{A \cdot \ldots \cdot A}_{p \text { Faktoren }} . \tag{6.12}
\end{equation*}


Die Potenz ist dann wieder eine quadratische Matrix, d.h. $A^{p} \in \mathbb{M}(n, n, \mathbb{R})$.\\
Beispiele:

\begin{itemize}
  \item \hspace{0pt} [2]
  \item $\left[\begin{array}{rr}2 & -1 \\ 7 & 5\end{array}\right]$
  \item $\left[\begin{array}{rrr}3 & -6 & 7 \\ 1 & 0 & -2 \\ 1 & 8 & 9\end{array}\right]$
\end{itemize}

Von zwei quadratischen Matrizen gleicher Dimension können beide Produkt, d.h. sowohl $A \cdot B$ als auch $B \cdot A$ gebildet werden. Die Differenz ist in vielen Anwendungen von Interesse.

Definition 6.7 Kommutator\\
Seien $n \in \mathbb{N}^{+}$und $A, B \in \mathbb{M}(n, n, \mathbb{R})$. Der Kommutator von $A$ und $B$ ist die Matrix


\begin{equation*}
[A, B]:=A \cdot B-B \cdot A \tag{6.13}
\end{equation*}


Bemerkungen:\\
i) Der Kommutator von zwei quadratischen Matrizen ist wieder eine quadratische Matrix, d.h. $A, B \in \mathbb{M}(n, n, \mathbb{R}) \Rightarrow[A, B] \in \mathbb{M}(n, n, \mathbb{R})$.\\
ii) Der Kommutator verschwindet genau dann, wenn die Matrizen kommutieren, d.h. wenn gilt $A \cdot B=B \cdot A$.\\
iii) Der Kommutator ist als Operation schiefsymmetrisch, d.h. es gilt


\begin{equation*}
[A, B]=-[B, A] . \tag{6.14}
\end{equation*}



\end{document}