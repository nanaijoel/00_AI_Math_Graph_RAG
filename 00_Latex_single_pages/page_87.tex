\documentclass[10pt]{article}
\usepackage[ngerman]{babel}
\usepackage[utf8]{inputenc}
\usepackage[T1]{fontenc}
\usepackage{amsmath}
\usepackage{amsfonts}
\usepackage{amssymb}
\usepackage[version=4]{mhchem}
\usepackage{stmaryrd}
\usepackage{graphicx}
\usepackage[export]{adjustbox}
\graphicspath{ {./images/} }
\usepackage{bbold}

\begin{document}
Die Lösungsmenge ist daher


\begin{equation*}
\underline{\underline{L}=\left\{3 \cdot \mathrm{e}^{\mathrm{i} \cdot \frac{\pi}{3}},-3,3 \cdot \mathrm{e}^{-\mathrm{i} \cdot \frac{\pi}{3}}\right\} .} \tag{5.41}
\end{equation*}


Die Lösungen bilden ein gleichseitiges Dreieck in der Gauss-Ebene.\\
\includegraphics[max width=\textwidth, center]{2025_05_07_68c58d275f0b63bcefaeg-1}

Wir betrachten den folgenden Satz.\\
Satz 5.7 Lösungen von Potenz-Gleichungen\\
Seien $w, z \in \mathbb{C}, n \in \mathbb{N}^{+}, r, \varphi \in \mathbb{R}$ mit $r>0$ und $z$ erfülle die Potenz-Gleichung


\begin{equation*}
z^{n}=w=r \cdot \mathrm{e}^{\mathrm{i} \cdot \varphi} . \tag{5.42}
\end{equation*}


Dann gilt folgendes.\\
(a) Für $w=0$ hat (5.42) genau eine Lösung für $z$, nämlich


\begin{equation*}
z=0 \text {. } \tag{5.43}
\end{equation*}


(b) Für $w \neq 0$ hat (5.42) genau $n$ Lösungen für $z$, nämlich


\begin{equation*}
z_{k}=\sqrt[n]{r} \cdot \mathrm{e}^{\mathrm{i} \cdot \frac{\varphi+(k-1) \cdot 2 \pi}{n}} \quad \text { für } \quad k \in\{1, \ldots, n\} . \tag{5.44}
\end{equation*}


Bemerkungen:\\
i) Die Tatsache, dass (5.42) ausser für $w=0$ genau $n$ Lösungen hat, wird als algebraische Vollständigkeit oder algebraische Abgeschlossenheit von $\mathbb{C}$ bezeichnet.\\
ii) Für alle Lösungen von (5.42) gilt


\begin{equation*}
\left|z_{k}\right|=\sqrt[n]{r}=\sqrt[n]{|w|} . \tag{5.45}
\end{equation*}


Die Lösungen liegen in der Gauss-Ebene daher auf dem Kreis um den Ursprung mit Radius $\sqrt[n]{|w|}$.\\
iii) Zwischen zwei benachbarten Lösungen von (5.42) liegt jeweils ein Winkel von


\begin{equation*}
\Delta \varphi=\frac{2 \pi}{n} . \tag{5.46}
\end{equation*}


iv) Die Lösungen von (5.42) bilden ein reguläres $n$-Eck in der Gauss-Ebene.


\end{document}