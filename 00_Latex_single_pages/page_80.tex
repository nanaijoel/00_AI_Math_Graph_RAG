\documentclass[10pt]{article}
\usepackage[ngerman]{babel}
\usepackage[utf8]{inputenc}
\usepackage[T1]{fontenc}
\usepackage{amsmath}
\usepackage{amsfonts}
\usepackage{amssymb}
\usepackage[version=4]{mhchem}
\usepackage{stmaryrd}
\usepackage{bbold}

\begin{document}
\subsection*{5.1.3 Zahlenumfang}
Nach Konstruktion enthält $\mathbb{C}$ offensichtlich die folgenden Zahlen.

\begin{enumerate}
  \item alle $x \in \mathbb{R}$
  \item $\mathrm{i} \in \mathbb{C} \backslash \mathbb{R}$
  \item alle reellen Vielfachen von i, d.h. $y \cdot \mathrm{i}$ mit $y \in \mathbb{R}$
  \item alle Linearkombinationen der Form $z=x+y \cdot \mathrm{i}$ mit $x, y \in \mathbb{R}$
\end{enumerate}

Tatsache ist, dass diese Aufzählung bereits alle komplexen Zahlen umfasst. Wir betrachten dazu den folgenden Satz.

Satz 5.1 Arithmetische Form\\
Für jedes $z \in \mathbb{C}$ gibt es eindeutige $x, y \in \mathbb{R}$, so dass gilt


\begin{equation*}
z=x+y \cdot \mathrm{i} . \tag{5.2}
\end{equation*}


Bemerkungen:\\
i) Jede komplexe Zahl lässt sich darstellen durch 2 reelle Zahlen. Somit muss $\mathbb{C}$ eine ähnliche Struktur haben wie $\mathbb{R}^{2}$.\\
ii) Für $y=0$ erhält man gerade alle reellen Zahlen.\\
iii) Die reellen Vielfachen von i sind die gesuchten neuen Zahlen, deren Quadrate einen negativen reellen Wert haben. Beispiele:


\begin{align*}
& z=2 \cdot \mathrm{i} \Rightarrow z^{2}=(2 \cdot \mathrm{i})^{2}=2^{2} \cdot \mathrm{i}^{2}=4 \cdot(-1)=-4  \tag{5.3}\\
& z=\sqrt{3} \cdot \mathrm{i} \Rightarrow z^{2}=(\sqrt{3} \cdot \mathrm{i})^{2}=3 \cdot(-1)=-3  \tag{5.4}\\
& z=y \cdot \mathrm{i} \Rightarrow z^{2}=(y \cdot \mathrm{i})^{2}=y^{2} \cdot(-1)=-y^{2} . \tag{5.5}
\end{align*}


Wir betrachten folgende Definition.

\section*{Definition 5.2 Weitere Bezeichnungen}
Seien $z \in \mathbb{C}$ und $x, y \in \mathbb{R}$ mit


\begin{equation*}
z=x+y \cdot \mathrm{i} . \tag{5.6}
\end{equation*}


Es werden die folgenden Bezeichnungen verwendet.\\
(a) Realteil von $z$\\
(c) Betrag von $z$

$$
\operatorname{Re}(z):=x .
$$

$$
|z|:=\sqrt{x^{2}+y^{2}} .
$$

(b) Imaginärteil von $z$\\
(d) Komplex-Konjugierte von $z$

$$
\operatorname{Im}(z):=y .
$$

$$
z^{*}:=x-y \cdot \mathrm{i} .
$$


\end{document}