\documentclass[10pt]{article}
\usepackage[ngerman]{babel}
\usepackage[utf8]{inputenc}
\usepackage[T1]{fontenc}
\usepackage{amsmath}
\usepackage{amsfonts}
\usepackage{amssymb}
\usepackage[version=4]{mhchem}
\usepackage{stmaryrd}
\usepackage{bbold}
\usepackage{graphicx}
\usepackage[export]{adjustbox}
\graphicspath{ {./images/} }

\begin{document}
\subsection*{2.4 Parametrisierte Flächen \& Flussintegrale}
\subsection*{2.4.1 Parametrisierte Flächen}
\subsection*{2.4.1.1 Definition}
Wir betrachten eine allgemeine Fläche $M \subset \mathbb{R}^{3}$, welche durch zwei Parameter bzw. Koordinaten $u$ und $v$ aus einem Gebiet $U \subseteq \mathbb{R}^{2}$ parametrisiert wird. Die Situation ist in der folgenden Skizze dargestellt.\\
\includegraphics[max width=\textwidth, center]{2025_05_07_20b6346c67068df5eb72g-1}

Definition 2.11 Parametrisierte Fläche\\
Sei $U \subseteq \mathbb{R}^{2}$ ein Gebiet und $M \subset \mathbb{R}^{3}$ ein Fläche. Eine Parametrisierung von $M$ ist eine Funktion der Form


\begin{align*}
& \mathbf{P}: U \rightarrow \mathbb{R}^{3} \\
& (u ; v) \mapsto \mathbf{P}(u ; v)=\left[\begin{array}{c}
x(u ; v) \\
y(u ; v) \\
z(u ; v)
\end{array}\right] \tag{2.98}
\end{align*}


so dass $M=\mathbf{P}(U)$.

Bemerkungen:\\
i) In Anlehnung an die Geographie werden $U$ Karten-Gebiet oder Karte und die Parameter $u$ und $v$ auch Koordinaten genannt.\\
ii) Erfahrungsgemäss können viele Rechnungen im Umgang mit der Parametrisierung einfach gehalten werden, wenn man, sofern möglich, für das Karten-Gebiet $U$ ein Rechteck wählt.\\
iii) Weil die Parametrisierung $\mathbf{P}$ von einem Gebiet in 2D in den Raum in 3D führt, kann sie als Funktion niemals surjektiv sein.\\
iv) Die Parametrisierung $\mathbf{P}$ soll als Funktion so "injektiv wie möglich" gewählt werden. Solange nur mehrere einzelne Punkte oder Kurvenstücke aus dem Karten-Gebiet auf den gleichen Punkt in $M$ abgebildet werden, ergeben sich in der Regel keine Probleme.\\
v) Die Parameter bzw. Koordinaten $u$ und $v$ können je nach Fläche ganz unterschiedliche Bedeutungen und Masseinheiten haben, z.B. Längen, Winkel, etc..


\end{document}