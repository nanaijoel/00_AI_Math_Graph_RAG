\documentclass[10pt]{article}
\usepackage[ngerman]{babel}
\usepackage[utf8]{inputenc}
\usepackage[T1]{fontenc}
\usepackage{amsmath}
\usepackage{amsfonts}
\usepackage{amssymb}
\usepackage[version=4]{mhchem}
\usepackage{stmaryrd}
\usepackage{graphicx}
\usepackage[export]{adjustbox}
\graphicspath{ {./images/} }

\begin{document}
\begin{enumerate}
  \setcounter{enumi}{2}
  \item Wir betrachten ein dreieckartiges Gebiet, wie im folgendem $x$ - $y$-Diagramm dargestellt.\\
\includegraphics[max width=\textwidth, center]{2025_05_07_280b6f4360f4bba83187g-1(1)}
\end{enumerate}

Gemäss Fubini-Satz erhalten wir in diesem Fall\\
$\int_{G} f \mathrm{~d} A=\int_{x_{0}}^{x_{\mathrm{E}}} \int_{y_{0}}^{g(x)} f(x ; y) \mathrm{d} y \mathrm{~d} x=\int_{y_{0}}^{y_{\mathrm{E}}} \int_{x_{0}}^{g^{-1}(y)} f(x ; y) \mathrm{d} x \mathrm{~d} y$.\\
4. Ein komplizierteres Gebiet kann meistens in einfachere Gebiete zerlegt und das Integral mit Hilfe des Zerlegungssatzes berechnet werden.\\
Bemerkungen:\\
i) Bei allen Anwendungen des Fubini-Satzes werden die Zweifach-Integrale in der Reihenfolge von innen nach aussen berechnet. Dabei dürfen die Grenzen des inneren Integrals von der Integrationsvariablen des äusseren Integrals abhängen.\\
ii) Bei der Integration über ein dreieckartiges Gebiet kann die Integrationsreihenfolge auf einfache Weise vertauscht werden.\\
iii) Es gibt eine Analogie zwischen den Grenzen eines Zweifach-Integrals und den Bewegungen des Stifts in einem Plotter.\\
\includegraphics[max width=\textwidth, center]{2025_05_07_280b6f4360f4bba83187g-1}

Dabei kann man sich vorstellen, dass der Stift des Plotters das Gebiet auf dem Papier komplett "überstreichen" muss.

\begin{center}
\begin{tabular}{|l|l|}
\hline
Inneres Integral & schnelle hin und her Bewegungen des Stifts entlang des Balkens \\
\hline
Äusseres Integral & langsame Bewegung des Balkens entlang des Papiers \\
\hline
\end{tabular}
\end{center}

Beim Vertauschen der Integrationsreihenfolge wird das Papier um einen rechten Winkel gedreht.


\end{document}