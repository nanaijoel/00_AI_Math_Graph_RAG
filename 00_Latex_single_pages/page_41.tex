\documentclass[10pt]{article}
\usepackage[ngerman]{babel}
\usepackage[utf8]{inputenc}
\usepackage[T1]{fontenc}
\usepackage{amsmath}
\usepackage{amsfonts}
\usepackage{amssymb}
\usepackage[version=4]{mhchem}
\usepackage{stmaryrd}
\usepackage{bbold}

\begin{document}
ii) Ist $f$ ein Skalarfeld, dann ist $\boldsymbol{\nabla} f$ ein Vektorfeld.

Beispiele:

\begin{itemize}
  \item Wir betrachten
\end{itemize}


\begin{equation*}
f(x ; y):=x^{2} \cdot y^{2} . \tag{2.138}
\end{equation*}


Der Gradient von $f$ ist

\[
\underline{\underline{\nabla f f}}=\left[\begin{array}{l}
f_{, 1}  \tag{2.139}\\
f_{, 2}
\end{array}\right]=\left[\begin{array}{l}
2 x \cdot y^{2} \\
x^{2} \cdot 2 y
\end{array}\right]=\left[\begin{array}{l}
2 x y^{2} \\
2 x^{2} y
\end{array}\right] .
\]

\begin{itemize}
  \item Wir betrachten
\end{itemize}


\begin{equation*}
f(x ; y ; z):=x^{2} \cdot y+z . \tag{2.140}
\end{equation*}


Der Gradient von $f$ ist

\[
\underline{\underline{\boldsymbol{\nabla} f}}=\left[\begin{array}{l}
f_{, 1}  \tag{2.141}\\
f_{, 2} \\
f_{3,3}
\end{array}\right]=\left[\begin{array}{c}
2 x \cdot y+0 \\
x^{2} \cdot 1+0 \\
0+1
\end{array}\right]=\underline{\underline{\left[\begin{array}{c}
2 x y \\
x^{2} \\
1
\end{array}\right]} . . . . ~}
\]

Wir betrachten den folgenden Satz.\\
Satz 2.12 Elementare Rechenregeln für Gradienten\\
Seien $n \in \mathbb{N}^{+}, g, h: \mathbb{R}^{n} \rightarrow \mathbb{R}$ differentierbare Funktionen und $a, b \in \mathbb{R}$, dann gelten die folgenden Rechenregeln.\\
(a) Faktor-Regel:\\
(c) Linearität:

$$
\boldsymbol{\nabla}(a \cdot g)=a \cdot \nabla g
$$

$$
\boldsymbol{\nabla}(a \cdot g+b \cdot h)=a \cdot \boldsymbol{\nabla} g+b \cdot \boldsymbol{\nabla} h
$$

(b) Summen-Regel:\\
(d) Produkt-Regel:

$$
\boldsymbol{\nabla}(g+h)=\boldsymbol{\nabla} g+\boldsymbol{\nabla} h
$$

$$
\boldsymbol{\nabla}(g \cdot h)=(\boldsymbol{\nabla} g) \cdot h+g \cdot \boldsymbol{\nabla} h
$$

Wir betrachten den folgenden Satz.

\section*{Satz 2.13 Ketten-Regeln für Gradienten}
Seien $n \in \mathbb{N}^{+}$, dann gelten folgende Ketten-Regeln.\\
(a) Ketten-Regel A: Für differentierbare $g: \mathbb{R} \rightarrow \mathbb{R}$ und $h: \mathbb{R}^{n} \rightarrow \mathbb{R}$ gilt

$$
f\left(x^{1} ; \ldots ; x^{n}\right):=g\left(h\left(x^{1} ; \ldots ; x^{n}\right)\right) \Rightarrow \boldsymbol{\nabla} f=g^{\prime}\left(h\left(x^{1} ; \ldots ; x^{n}\right)\right) \cdot \boldsymbol{\nabla} h .
$$

(b) Ketten-Regel B: Für differentierbare $g: \mathbb{R}^{n} \rightarrow \mathbb{R}$ und $\mathbf{h}: \mathbb{R} \rightarrow \mathbb{R}^{n}$ gilt

$$
f(x):=g(\mathbf{h}(x)) \Rightarrow f^{\prime}(x)=\left\langle\boldsymbol{\nabla} g(\mathbf{h}(x)), \mathbf{h}^{\prime}(x)\right\rangle .
$$


\end{document}