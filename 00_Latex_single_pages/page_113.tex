\documentclass[10pt]{article}
\usepackage[ngerman]{babel}
\usepackage[utf8]{inputenc}
\usepackage[T1]{fontenc}
\usepackage{amsmath}
\usepackage{amsfonts}
\usepackage{amssymb}
\usepackage[version=4]{mhchem}
\usepackage{stmaryrd}
\usepackage{bbold}

\begin{document}
Definition 6.17 Rotationsgenerator\\
Für jedes $\mathbf{w} \in \mathbb{R}^{3}$ ist der Rotationsgenerator die Matrix

\[
J(\mathbf{w}):=\left[\begin{array}{ccc}
0 & -w_{3} & w_{2}  \tag{6.82}\\
w_{3} & 0 & -w_{1} \\
-w_{2} & w_{1} & 0
\end{array}\right] .
\]

Im folgenden Satz stellen wir die wichtigsten algebraischen Eigenschaften des Rotationsgenerators zusammen.

Satz 6.11 Eigenschaften des Rotationsgenerators\\
Seien $\mathbf{v}, \mathbf{w} \in \mathbb{R}^{3}$ mit $w=|\mathbf{w}|$, dann gilt folgendes.\\
(a) $J(\mathbf{w})$ ist schiefsymmetrisch\\
(d) $J^{T}(\mathbf{w})=-J(\mathbf{w})=J(-\mathbf{w})$\\
(b) $J^{2}(\mathbf{w})$ ist symmetrisch\\
(e) $J^{3}(\mathbf{w})=-w^{2} \cdot J(\mathbf{w})$\\
(c) $J(\mathbf{w}) \cdot \mathbf{v}=\mathbf{w} \times \mathbf{v}$\\
(f) $J^{4}(\mathbf{w})=-w^{2} \cdot J^{2}(\mathbf{w})$

Bemerkungen:\\
i) Gemäss Eigenschaft (c) beschreibt $J(\mathbf{w})$ gerade das Grassmann-Vektor-Produkt von links mit dem Vektor w.\\
ii) Für alle $\mathbf{v}, \mathbf{w} \in \mathbb{R}^{3}$ gilt die berühmte Drehimpuls-Kommutatonsrelation


\begin{equation*}
[J(\mathbf{v}), J(\mathbf{w})]=J(\mathbf{v} \times \mathbf{w}) \tag{6.83}
\end{equation*}


Mit Hilfe des Rotationsgenerators lässt sich die Abbildungsmatrix einer Rotation in 3D einfach ausdrücken.

\section*{Satz 6.12 Rodrigues-Formel}
Die Rotation in $\mathbb{R}^{3}$ um den Winkel $\varphi \in \mathbb{R}$ rechtshändig um die Drehachse in Richtung $\hat{\boldsymbol{\varphi}} \in \mathbb{R}^{3}$ wird beschrieben durch die Abbildungsmatrix


\begin{equation*}
R(\boldsymbol{\varphi})=\mathbb{1}+(1-\cos (\varphi)) \cdot J^{2}(\hat{\boldsymbol{\varphi}})+\sin (\varphi) \cdot J(\hat{\boldsymbol{\varphi}}) . \tag{6.84}
\end{equation*}


Beweis: Um das Bild eines Vektors $\mathbf{v} \in \mathbb{R}^{3}$ unter der Rotation zu berechnen, zerlegen wir $\mathbf{v}$ in seine Anteile parallel und senkrecht zu $\hat{\varphi}$. Gemäss Skizze und den Eigenschaften des Grassmann- Vektor-Produkts gilt


\begin{align*}
\hat{\boldsymbol{\varphi}} \times \mathbf{v} & =\hat{\boldsymbol{\varphi}} \times \mathbf{v}_{\perp}=|\hat{\boldsymbol{\varphi}}| \cdot\left|\mathbf{v}_{\perp}\right| \cdot \hat{\mathbf{a}}=1 \cdot v_{\perp} \cdot \hat{\mathbf{a}}=v_{\perp} \cdot \hat{\mathbf{a}}  \tag{6.85}\\
\hat{\boldsymbol{\varphi}} \times(\hat{\boldsymbol{\varphi}} \times \mathbf{v}) & =\hat{\boldsymbol{\varphi}} \times\left(v_{\perp} \cdot \hat{\mathbf{a}}\right)=v_{\perp} \cdot \hat{\boldsymbol{\varphi}} \times \hat{\mathbf{a}}=v_{\perp} \cdot \hat{\mathbf{b}} . \tag{6.86}
\end{align*}


Mit Hilfe der Skizze und durch Einsetzen des Rotationsgenerators erhalten wir daraus

$$
\begin{aligned}
\underline{R(\boldsymbol{\varphi}) \cdot \mathbf{v}} & =\mathbf{v}+v_{\perp} \cdot \hat{\mathbf{b}}-v_{\perp} \cdot \cos (\varphi) \cdot \hat{\mathbf{b}}+v_{\perp} \cdot \sin (\varphi) \cdot \hat{\mathbf{a}} \\
& =\mathbf{v}+(1-\cos (\varphi)) \cdot v_{\perp} \cdot \hat{\mathbf{b}}+\sin (\varphi) \cdot v_{\perp} \cdot \hat{\mathbf{a}} \\
& =\mathbf{v}+(1-\cos (\varphi)) \cdot \hat{\boldsymbol{\varphi}} \times(\hat{\boldsymbol{\varphi}} \times \mathbf{v})+\sin (\varphi) \cdot \hat{\boldsymbol{\varphi}} \times \mathbf{v}
\end{aligned}
$$


\end{document}