\documentclass[10pt]{article}
\usepackage[ngerman]{babel}
\usepackage[utf8]{inputenc}
\usepackage[T1]{fontenc}
\usepackage{amsmath}
\usepackage{amsfonts}
\usepackage{amssymb}
\usepackage[version=4]{mhchem}
\usepackage{stmaryrd}
\usepackage{bbold}
\usepackage{graphicx}
\usepackage[export]{adjustbox}
\graphicspath{ {./images/} }

\begin{document}
\subsection*{2.4.1.2 Koordinatenbasis-Vektorfelder}
Wir betrachten folgende Definition.

\section*{Definition 2.12 Koordinatenbasis-Vektorfelder}
Seien $U \subseteq \mathbb{R}^{2}$ und $\mathbf{P}: U \rightarrow \mathbb{R}^{3}$ eine parametrisierte Fläche mit Koordinaten $u$ und $v$. Die Koordinatenbasis-Vektorfelder sind die Vektorfelder

\[
\begin{array}{|lrl|}
\mathbf{e}_{u}: U & \rightarrow \mathbb{R}^{3}  \tag{2.101}\\
(u ; v) & \mapsto \mathbf{e}_{u}(u ; v):=\mathbf{P}_{, u}(u ; v)
\end{array} \quad \text { und } \quad \begin{aligned}
& \mathbf{e}_{v}: U \rightarrow \mathbb{R}^{3} \\
&(u ; v) \mapsto \mathbf{e}_{v}(u ; v):=\mathbf{P}_{, v}(u ; v), \\
& \hline
\end{aligned}
\]

wobei mit $\mathbf{P}_{, u}$ und $\mathbf{P}_{, v}$ die Ableitungen von $\mathbf{P}$ nach den Koordinaten $u$ bzw. $v$ gemeint sind.

\section*{Bemerkungen:}
i) Die Koordinatenbasis-Vektorfelder sind Funktionen der Koordinaten $u$ und $v$ und in diesem Sinne wirklich Vektorfelder.\\
ii) Die Koordinatenbasis-Vektorfelder werden bezeichnet durch $\left\{\mathbf{e}_{u}, \mathbf{e}_{v}\right\}$ oder $\left\{\mathbf{e}_{1}, \mathbf{e}_{2}\right\}$.\\
iii) An jedem Punkt von $M$ zeigen die Koordinatenbasis-Vektorfelder je in eine tangentiale Richtung. Sie spannen daher die Tangentialebene an $M$ auf.\\
\includegraphics[max width=\textwidth, center]{2025_05_07_e77e3b9ac183d430c62ag-1}

\subsection*{2.4.1.3 Normalen-Vektor}
Wir betrachten folgende Definition.\\
Definition 2.13 Normalen-Vektor\\
Der Normalen-Vektor einer parametrisierten Fläche ist das Grassmann-Vektor-Produkt der Koordinatenbasis-Vektorfelder, d.h.


\begin{equation*}
\mathbf{n}:=\mathbf{e}_{u} \times \mathbf{e}_{v} . \tag{2.102}
\end{equation*}


Bemerkungen:\\
i) Der Normalen-Vektor steht senkrecht auf der Fläche M.\\
ii) Die Länge des Normalen-Vektors hängt ab von der Parametrisierung.


\end{document}