\documentclass[10pt]{article}
\usepackage[ngerman]{babel}
\usepackage[utf8]{inputenc}
\usepackage[T1]{fontenc}
\usepackage{amsmath}
\usepackage{amsfonts}
\usepackage{amssymb}
\usepackage[version=4]{mhchem}
\usepackage{stmaryrd}
\usepackage{bbold}

\begin{document}
\begin{enumerate}
  \setcounter{enumi}{2}
  \item Level-Mengen-Plot: Die Level-Mengen zu verschiedenen Levels werden gezeichnet.
\end{enumerate}

Anwendungen: Höhenlinien, Äquipotentiallinien, Äquipotentialfl̈̈chen, Atomorbitale.\\
Bemerkungen:\\
i) Welche Visualisierung am besten geeignet ist, hängt sowohl von der Dimension als auch vom Verlauf der Funktion ab.\\
ii) Nicht jede Fläche in 3D ist der Graph einer Funktion $f: \mathbb{R}^{2} \rightarrow \mathbb{R}$. Gegenbeispiel: Alle geschlossenen Flächen, wie z.B. eine Sphäre.\\
iii) Jede Ebene in 3D kann als Graph einer Funktion $f: \mathbb{R}^{2} \rightarrow \mathbb{R}$ beschrieben werden. Dies geschieht mit Hilfe einer Normalen-Gleichung.

\subsection*{2.1.2 Vektorfelder}
\subsection*{2.1.2.1 Definition}
Wir betrachten die folgende Definition.\\
Definition 2.3 Vektorwertige Funktion in mehreren reellen Variablen\\
Seien $n \in \mathbb{N}^{+} \backslash\{1\}, A, B \subseteq \mathbb{R}^{n}$. Eine Funktion auf $A$ der Form $\mathbf{v}: A \rightarrow B$ heisst vektorwertige Funktion in $n$ reellen Variablen.

Bemerkungen:\\
i) Die reellen Variablen werden nach dem Funktionsnamen v in runden Klammern aufgezählt, jeweils durch ein Semicolon getrennt.

\[
\left.\begin{array}{rlrl}
n=2: & \mathbf{v}(x ; y) & =\left[\begin{array}{c}
v_{x}(x ; y) \\
v_{y}(x ; y)
\end{array}\right] \\
n=3: & \mathbf{v}(x ; y ; z) & =\left[\begin{array}{c}
v_{x}(x ; y ; z) \\
v_{y}(x ; y ; z) \\
v_{z}(x ; y ; z)
\end{array}\right] \\
& & \text { allg: } & \mathbf{v}\left(x_{1} ; x_{2} ; \ldots ; x_{n}\right)
\end{array}\right)=\left[\begin{array}{c}
v_{1}\left(x_{1} ; x_{2} ; \ldots ; x_{n}\right)  \tag{2.20}\\
v_{2}\left(x_{1} ; x_{2} ; \ldots ; x_{n}\right) \\
\vdots \\
v_{n}\left(x_{1} ; x_{2} ; \ldots ; x_{n}\right)
\end{array}\right], ~ ل
\]

ii) Jede Komponente eines Vektorfeldes ist eine reellwertige Funktion in $n$ reellen Variablen.


\end{document}