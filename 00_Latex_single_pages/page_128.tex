\documentclass[10pt]{article}
\usepackage[ngerman]{babel}
\usepackage[utf8]{inputenc}
\usepackage[T1]{fontenc}
\usepackage{amsmath}
\usepackage{amsfonts}
\usepackage{amssymb}
\usepackage[version=4]{mhchem}
\usepackage{stmaryrd}
\usepackage{bbold}

\begin{document}
iii) Beispiel-Codes zur Berechnung des charakteristischen Polynoms mit gängiger Software.

\begin{center}
\begin{tabular}{|l|l|}
\hline
MATLAB/Octave & $\mathrm{p}=$ charpoly (A) \\
\hline
Mathematica/WolframAlpha & $\mathrm{p}=$ CharacteristicPolynomial $[\mathrm{A}, \mathrm{x}]$ \\
\hline
Python/Numpy & import numpy as np; p=np.poly(A) \\
\hline
Python/Sympy & import sympy as sp; p=A.charpoly() \\
\hline
\end{tabular}
\end{center}

Beispiele:

\begin{itemize}
  \item Wir betrachten die Matrix\\
$A=\left[\begin{array}{ll}7 & 2 \\ 3 & 8\end{array}\right]$.\\
Es folgt
\end{itemize}


\begin{align*}
\operatorname{tr}(A) & =7+8=15  \tag{6.164}\\
\operatorname{det}(A) & =7 \cdot 8-3 \cdot 2=56-6=50  \tag{6.165}\\
\underline{\underline{p_{A}(\lambda)}} & =\lambda^{2}-\operatorname{tr}(A) \cdot \lambda+\operatorname{det}(A)=\underline{\underline{\lambda^{2}-15 \lambda+50 .}} \tag{6.166}
\end{align*}


\begin{itemize}
  \item $A=\left[\begin{array}{ll}5 & 1 \\ 1 & 5\end{array}\right] \Rightarrow \ldots \Rightarrow p_{A}(\lambda)=\lambda^{2}-10 \lambda+24$
  \item $A=\left[\begin{array}{ll}0 & 1 \\ 1 & 0\end{array}\right] \Rightarrow \ldots \Rightarrow p_{A}(\lambda)=\lambda^{2}-1$
\end{itemize}

\subsection*{6.5.4 Eigenwerte \& Eigenvektoren berechnen}
Um die Eigenwerte und Eigenvektoren einer Matrix zu berechnen, hilft uns der folgende Satz weiter.

Satz 6.22 Eigenwerte sind Nullstellen des charakteristischen Polynoms\\
Seien $n \in \mathbb{N}^{+}, A \in \mathbb{M}(n, n, \mathbb{R})$ und $\lambda \in \mathbb{R}$. Dann gilt


\begin{equation*}
\lambda \in \operatorname{Spec}(A) \Leftrightarrow p_{A}(\lambda)=0 . \tag{6.167}
\end{equation*}


Beweis: Für einen Eigenwert $\lambda \in \mathbb{R}$ und einen zugehörigen Eigenvektor $\mathbf{E} \in \mathbb{R}^{n} \backslash\{0\}$ gilt

$$
\begin{array}{lll} 
& A \cdot \mathbf{E}=\lambda \cdot \mathbf{E} & \mid-A \cdot \mathbf{E} \\
\Leftrightarrow & \lambda \cdot \mathbf{E}-A \cdot \mathbf{E}=0 & \\
\Leftrightarrow & (\lambda \cdot \mathbb{1}-A) \cdot \mathbf{E}=0 . &
\end{array}
$$

Dies ist ein homogenes, lineares Gleichungssystem für $\mathbf{E}$, das genau dann von 0 verschiedene Lösungen hat, wenn die Matrix $\lambda \cdot \mathbb{1}-A$ singulär ist. Demnach muss gelten


\begin{equation*}
\underline{\underline{0}}=\operatorname{det}(\lambda \cdot \mathbb{1}-A)=\underline{\underline{p_{A}(\lambda)}} . \tag{6.171}
\end{equation*}


Damit haben wir den Satz bewiesen.


\end{document}