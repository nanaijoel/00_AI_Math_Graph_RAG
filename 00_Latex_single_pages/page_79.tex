\documentclass[10pt]{article}
\usepackage[ngerman]{babel}
\usepackage[utf8]{inputenc}
\usepackage[T1]{fontenc}
\usepackage{amsmath}
\usepackage{amsfonts}
\usepackage{amssymb}
\usepackage[version=4]{mhchem}
\usepackage{stmaryrd}
\usepackage{bbold}

\begin{document}
\section*{Kapitel 5}
\section*{Komplexe Zahlen}
\subsection*{5.1 Grundlagen}
\subsection*{5.1.1 Einleitung}
Bekanntlich gibt es kein reelles $x \in \mathbb{R}$ mit $x^{2}<0$. Das bedeutet, für $a<0$ hat die Gleichung


\begin{equation*}
x^{2}=a \tag{5.1}
\end{equation*}


keine Lösungen. Die Idee ist nun, eine noch grössere Zahlenmenge zu konstruieren, so dass auch alle negative reellen Zahlen als Quadrate geschrieben werden können.

\subsection*{5.1.2 Konstruktion}
Wir betrachten folgende Definition.

\section*{Definition 5.1 Komplexe Zahlen}
Die komplexen Zahlen $\mathbb{C}$ sind die kleinste Menge, welche die folgenden Eigenschaften erfült.\\
A1 $\mathbb{R} \subseteq \mathbb{C}$.\\
A2 $(\mathbb{C} ;+; \cdot)$ bildet einen Zahlenkörper.\\
A3 Es gibt ein $\mathrm{i} \in \mathbb{C}$ mit $\mathrm{i}^{2}=-1$.

Bemerkungen:\\
i) Die Zahl i heisst imaginäre Einheit.\\
ii) Weil in der Elektrotechnik $i$ schon die elektrische Stromstärke bezeichnet, wird in der Literatur oft auch ein $j$ verwendet.\\
iii) Alle Elemente von $\mathbb{C}$ lassen sich durch reelle Zahlen und i beschreiben.\\
iv) Man kann zeigen, dass $\mathbb{C}$ der grösstmögliche Zahlenkörper ist.\\
v) Man kann zeigen, dass $\mathbb{C}$ algebraisch abgeschlossen ist und somit kein Bedarf für eine noch grössere Zahlenmenge besteht.


\end{document}