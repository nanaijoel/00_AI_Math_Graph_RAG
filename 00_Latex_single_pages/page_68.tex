\documentclass[10pt]{article}
\usepackage[ngerman]{babel}
\usepackage[utf8]{inputenc}
\usepackage[T1]{fontenc}
\usepackage{amsmath}
\usepackage{amsfonts}
\usepackage{amssymb}
\usepackage[version=4]{mhchem}
\usepackage{stmaryrd}
\usepackage{bbold}

\begin{document}
Bemerkungen:\\
i) Uneigentliche Integrale sind als Grenzwerte definiert und existieren nur, wenn die betreffenden Grenzwerte konvergieren.\\
ii) Bei der Berechnung eines uneigentlichen Integrals ist die Reihenfolge der Rechenschritte sehr wichtig. Zuerst muss eine Stammfunktion $F$ von $f$ gefunden und an den Grenzen $x_{0}$ und $s$ bzw. $-s$ und $x_{\mathrm{E}}$ ausgewertet werden. Erst dann wird der Grenzwert $s \rightarrow \infty$ betrachtet. Dieses Vorgehen führt zu Rechnungen der Form


\begin{equation*}
I=\int_{x_{0}}^{\infty} f(x) \mathrm{d} x=\lim _{s \rightarrow \infty} \int_{x_{0}}^{s} f(x) \mathrm{d} x=\left.\lim _{s \rightarrow \infty}[F(x)]\right|_{x_{0}} ^{s}=\lim _{s \rightarrow \infty}\left(F(s)-F\left(x_{0}\right)\right)=\ldots \tag{3.33}
\end{equation*}


iii) Das uneigentliche Integral kann nur existieren, wenn sich der Integrand der $x$-Achse asymptotisch annähert, d.h. es muss gelten


\begin{equation*}
\lim _{x \rightarrow \infty} f(x)=0 \quad \text { bzw. } \quad \lim _{x \rightarrow-\infty} f(x)=0 . \tag{3.34}
\end{equation*}


iv) Allein die Tatsache, dass sich der Integrand der $x$-Achse asymptotisch annähert reicht für die Existenz des uneigentlichen Integrals jedoch nicht aus! Es kommt vielmehr darauf an, wie "schnell" diese asymptotische Annäherung stattfindet.\\
v) Beispiel-Codes zur Berechnung von uneigentlichen Integralen mit gängiger Software.

\begin{center}
\begin{tabular}{|l|l|}
\hline
Mathematica/WolframAlpha & Integrate[1/x\^{}2,\{x,2, Infinity $\}]$ \\
\hline
Python/Sympy & \begin{tabular}{l}
import sympy as $\mathrm{sp} ;$ \\
sp.integrate $(1 / \mathrm{x} * * 2,(\mathrm{x}, 2, \mathrm{sp} .00)) ;$ \\
\end{tabular} \\
\hline
\end{tabular}
\end{center}

Wir betrachten den folgenden Satz.\\
Satz 3.4 Uneigentliches Integral von reziproken Potenzen\\
Seien $p, x_{0} \in \mathbb{R}$ mit $x_{0}>0$, dann gilt

\[
\int_{x_{0}}^{\infty} \frac{1}{x^{p}} \mathrm{~d} x=\left\{\begin{array}{r|l|}
\frac{1}{p-1} \cdot \frac{1}{x_{0}^{p-1}} & p>1  \tag{3.35}\\
\text { divergent } & p \leq 1 .
\end{array}\right.
\]

Beweis: Übung.


\end{document}