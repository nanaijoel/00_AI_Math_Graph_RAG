\documentclass[10pt]{article}
\usepackage[ngerman]{babel}
\usepackage[utf8]{inputenc}
\usepackage[T1]{fontenc}
\usepackage{amsmath}
\usepackage{amsfonts}
\usepackage{amssymb}
\usepackage[version=4]{mhchem}
\usepackage{stmaryrd}
\usepackage{bbold}

\begin{document}
\subsection*{5.1.4 Grundoperationen}
Wir betrachten $x_{1}, y_{1}, x_{2}, y_{2} \in \mathbb{R}$ und die beiden komplexen Zahlen


\begin{equation*}
z_{1}=x_{1}+y_{1} \cdot \mathrm{i} \quad \text { und } \quad z_{2}=x_{2}+y_{2} \cdot \mathrm{i} . \tag{5.7}
\end{equation*}


Beim Durchführen der algebraischen Grundoperationen erhalten wir die folgenden Ergebnisse:\\
Addition: Es gilt


\begin{equation*}
\underline{\underline{z_{1}+z_{2}}}=x_{1}+y_{1} \cdot \mathrm{i}+x_{2}+y_{2} \cdot \mathrm{i}=x_{1}+x_{2}+y_{1} \cdot \mathrm{i}+y_{2} \cdot \mathrm{i}=\underline{\underline{x_{1}+x_{2}+\left(y_{1}+y_{2}\right) \cdot \mathrm{i}}} \tag{5.8}
\end{equation*}


Subtraktion: Es gilt


\begin{equation*}
\underline{\underline{z_{1}-z_{2}}}=x_{1}+y_{1} \cdot \mathrm{i}-x_{2}-y_{2} \cdot \mathrm{i}=x_{1}-x_{2}+y_{1} \cdot \mathrm{i}-y_{2} \cdot \mathrm{i}=\underline{\left.\underline{x_{1}-x_{2}+\left(y_{1}-y_{2}\right)}\right) \cdot \mathrm{i} .} \tag{5.9}
\end{equation*}


Multiplikation: Es gilt


\begin{align*}
\underline{z_{1} \cdot z_{2}} & =\left(x_{1}+y_{1} \cdot \mathrm{i}\right) \cdot\left(x_{2}+y_{2} \cdot \mathrm{i}\right)=x_{1} \cdot x_{2}+y_{1} \cdot \mathrm{i} \cdot x_{2}+x_{1} \cdot y_{2} \cdot \mathrm{i}+y_{1} \cdot \mathrm{i} \cdot y_{2} \cdot \mathrm{i} \\
& =x_{1} \cdot x_{2}+x_{2} \cdot y_{1} \cdot \mathrm{i}+x_{1} \cdot y_{2} \cdot \mathrm{i}+y_{1} \cdot y_{2} \cdot \mathrm{i}^{2} \\
& =x_{1} \cdot x_{2}+x_{1} \cdot y_{2} \cdot \mathrm{i}+x_{2} \cdot y_{1} \cdot \mathrm{i}+y_{1} \cdot y_{2} \cdot(-1) \\
& =\underline{x_{1} \cdot x_{2}-y_{1} \cdot y_{2}+\left(x_{1} \cdot y_{2}+x_{2} \cdot y_{1}\right) \cdot \mathrm{i} .} \tag{5.10}
\end{align*}


Division: Für $z_{2} \neq 0$ gilt


\begin{align*}
\frac{z_{1}}{\underline{z_{2}}} & =\frac{x_{1}+y_{1} \cdot \mathrm{i}}{x_{2}+y_{2} \cdot \mathrm{i}}=\frac{\left(x_{1}+y_{1} \cdot \mathrm{i}\right) \cdot\left(x_{2}-y_{2} \cdot \mathrm{i}\right)}{\left(x_{2}+y_{2} \cdot \mathrm{i}\right) \cdot\left(x_{2}-y_{2} \cdot \mathrm{i}\right)}=\frac{x_{1} \cdot x_{2}+y_{1} \cdot y_{2}+\left(-x_{1} \cdot y_{2}+x_{2} \cdot y_{1}\right) \cdot \mathrm{i}}{x_{2} \cdot x_{2}+y_{2} \cdot y_{2}+\left(x_{2} \cdot y_{2}-x_{2} \cdot y_{2}\right) \cdot \mathrm{i}} \\
& =\frac{x_{1} \cdot x_{2}+y_{1} \cdot y_{2}+\left(x_{2} \cdot y_{1}-x_{1} \cdot y_{2}\right) \cdot \mathrm{i}}{x_{2}^{2}+y_{2}^{2}+0}=\frac{x_{1} \cdot x_{2}+y_{1} \cdot y_{2}}{x_{2}^{2}+y_{2}^{2}}+\frac{x_{2} \cdot y_{1}-x_{1} \cdot y_{2}}{x_{2}^{2}+y_{2}^{2}} \cdot \mathrm{i} . \tag{5.11}
\end{align*}


Die Ergebnisse lassen sich also auf einfache Weise wieder in arithmetischer Form schreiben. Wir betrachten den folgenden Satz.

Satz 5.2 Betrag \& Konjugation\\
Sei $z \in \mathbb{C}$, dann gilt


\begin{equation*}
z \cdot z^{*}=|z|^{2} . \tag{5.12}
\end{equation*}


Beweis: Es gibt $x, y \in \mathbb{R}$, so dass gilt


\begin{equation*}
z=x+y \cdot \mathrm{i} \quad \text { und } \quad z^{*}=x-y \cdot \mathrm{i} . \tag{5.13}
\end{equation*}


Daraus erhalten wir


\begin{equation*}
\underline{\underline{z \cdot z^{*}}}=(x+y \cdot \mathrm{i}) \cdot(x-y \cdot \mathrm{i})=x^{2}+y \cdot \mathrm{i} \cdot x-x \cdot y \cdot \mathrm{i}-y^{2} \cdot \mathrm{i}^{2}=x^{2}+y^{2}=\underline{\underline{|z|^{2}} .} \tag{5.14}
\end{equation*}


Damit haben wir den Satz bewiesen.


\end{document}