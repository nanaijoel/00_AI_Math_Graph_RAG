\documentclass[10pt]{article}
\usepackage[ngerman]{babel}
\usepackage[utf8]{inputenc}
\usepackage[T1]{fontenc}
\usepackage{amsmath}
\usepackage{amsfonts}
\usepackage{amssymb}
\usepackage[version=4]{mhchem}
\usepackage{stmaryrd}

\begin{document}
Beispiele:

\begin{itemize}
  \item Durch partielle Integration erhalten wir
\end{itemize}

\[
\underline{\underline{F(x)}}=\int \begin{align*}
& \downarrow  \tag{3.22}\\
& x
\end{align*} \cdot \mathrm{e}^{x} \mathrm{~d} x=x \mathrm{e}^{x}-\int 1 \cdot \mathrm{e}^{x} \mathrm{~d} x=\underline{\underline{\mathrm{e}^{x}}-\mathrm{e}^{x}+c=(x-1) \mathrm{e}^{x}+c .}
\]

\begin{itemize}
  \item Durch partielle Integration und mit Hilfe des Pythagoras-Satzes für trigonometrische Funktionen finden wir die Gleichung
\end{itemize}


\begin{align*}
F(x) & =\int \sin ^{2}(x) \mathrm{d} x=\int \stackrel{\downarrow}{\sin (x) \cdot \sin (x) \mathrm{d} x} \\
& =\sin (x) \cdot(-\cos (x))-\int \cos (x) \cdot(-\cos (x)) \mathrm{d} x=-\sin (x) \cos (x)+\int \cos ^{2}(x) \mathrm{d} x \\
& =-\sin (x) \cos (x)+\int\left(1-\sin ^{2}(x)\right) \mathrm{d} x=-\sin (x) \cos (x)+\int 1 \mathrm{~d} x-\int \sin ^{2}(x) \mathrm{d} x \\
& =-\sin (x) \cos (x)+x+b-F(x) . \tag{3.23}
\end{align*}


Es gilt also


\begin{align*}
F(x) & =-\sin (x) \cos (x)+x+b-F(x) & & \mid+F(x)  \tag{3.24}\\
2 \cdot F(x) & =-\sin (x) \cos (x)+x+b & & \mid: 2 . \tag{3.25}
\end{align*}


Daraus erhalten wir


\begin{equation*}
\underline{\underline{F(x)}}=\frac{-\sin (x) \cos (x)+x+b}{2}=\underline{=\frac{x-\sin (x) \cos (x)}{2}+c .} \tag{3.26}
\end{equation*}


\begin{itemize}
  \item Durch partielle Integration erhalten wir
\end{itemize}


\begin{align*}
\underline{\underline{F(x)}} & =\int \ln (x) \mathrm{d} x=\int \ln (x) \cdot 1 \mathrm{~d} x=\ln (x) \cdot x-\int \ln ^{\prime}(x) \cdot x \mathrm{~d} x=x \cdot \ln (x)-\int \frac{1}{x} \cdot x \mathrm{~d} x \\
& =x \cdot \ln (x)-\int 1 \mathrm{~d} x=x \cdot \ln (x)-x+c=\underline{\underline{x \cdot(\ln (x)-1)+c .}} \tag{3.27}
\end{align*}


Bemerkungen:\\
i) Die Idee hinter der partiellen Integration ist die Umkehrung der Produkt-Regel aus der Differentialrechnung.\\
ii) Durch Anwenden der partiellen Integration kann eine schwierige Integration auf eine einfachere Integration zurückgeführt werden.\\
iii) Der Begriff partielle Integration bedeutet "teilweise Integration".\\
iv) Eine häufige Fehlerquelle bei der Anwendung der partiellen Integration ist das negative Vorzeichen vor dem Integral auf der rechten Seite.


\end{document}