\documentclass[10pt]{article}
\usepackage[ngerman]{babel}
\usepackage[utf8]{inputenc}
\usepackage[T1]{fontenc}
\usepackage{amsmath}
\usepackage{amsfonts}
\usepackage{amssymb}
\usepackage[version=4]{mhchem}
\usepackage{stmaryrd}
\usepackage{bbold}

\begin{document}
iv) Ganz besonders bequem lässt sich ein LGLS lösen, wenn die linke Seite eine orthogonale Matrix ist. Gilt $A \in \mathrm{O}(n)$, dann folgt

\[
\begin{array}{rlrl} 
& A \cdot \mathbf{u} & =\mathbf{b} & \mid A^{T} . \\
\Leftrightarrow & A^{T} \cdot A \cdot \mathbf{u}=A^{-1} \cdot A \cdot \mathbf{u}=A^{T} \cdot \mathbf{b} & \\
\Leftrightarrow & \mathbb{1} \cdot \mathbf{u}=A^{T} \cdot \mathbf{b} . & \tag{6.59}
\end{array}
\]

Daraus erhalten wir die Lösung


\begin{equation*}
\underline{\underline{\mathbf{u}}=A^{T} \cdot \mathbf{b}} \tag{6.60}
\end{equation*}


Das LGLS ist demnach eindeutig lösbar und die Lösung lässt sich ganz einfach durch Multiplikation der rechten Seite von links mit der Transponierten von $A$ berechnen.

Beispiele:

\begin{itemize}
  \item $\left[\begin{array}{ll}0 & 1 \\ 1 & 0\end{array}\right]$
  \item $\frac{1}{\sqrt{13}} \cdot\left[\begin{array}{rr}-2 & 3 \\ 3 & 2\end{array}\right]$
  \item $\frac{1}{5} \cdot\left[\begin{array}{rrr}3 & -4 & 0 \\ 4 & 3 & 0 \\ 0 & 0 & 5\end{array}\right]$
\end{itemize}

\subsection*{6.3.2 Geometrische Eigenschaften}
Die Spalten-Vektoren einer orthogonalen Matrix haben bemerkenswerte geometrische Eigenschaften.

Satz 6.7 Orthonormalität der Spalten-Vektoren\\
Seien $n \in \mathbb{N}^{+}$und $A \in \mathrm{O}(n)$ mit Spaltenvektoren $\mathbf{a}_{1}, \ldots, \mathbf{a}_{n} \in \mathbb{R}^{n}$, d.h.

\[
A=\left[\begin{array}{llll}
\mathbf{a}_{1} & \mathbf{a}_{2} & \ldots & \mathbf{a}_{n} \tag{6.61}
\end{array}\right] .
\]

Dann gilt

\[
\left\langle\mathbf{a}_{i}, \mathbf{a}_{j}\right\rangle=\delta_{i j}=\left\{\begin{array}{l|l}
1 & i=j  \tag{6.62}\\
0 & i \neq j
\end{array}\right.
\]

Beweis: Weil $A$ orthogonal ist, gilt\\
$\mathbb{\#}=A^{-1} \cdot A=A^{T} \cdot A=\left[\begin{array}{c}\mathbf{a}_{1}^{T} \\ \mathbf{a}_{2}^{T} \\ \vdots \\ \mathbf{a}_{n}^{T}\end{array}\right] \cdot\left[\begin{array}{llll}\mathbf{a}_{1} & \mathbf{a}_{2} & \ldots & \mathbf{a}_{n}\end{array}\right]=\left[\begin{array}{cccc}\mathbf{a}_{1}^{T} \cdot \mathbf{a}_{1} & \mathbf{a}_{1}^{T} \cdot \mathbf{a}_{2} & \ldots & \mathbf{a}_{1}^{T} \cdot \mathbf{a}_{n} \\ \mathbf{a}_{2}^{T} \cdot \mathbf{a}_{1} & \mathbf{a}_{2}^{T} \cdot \mathbf{a}_{2} & \ldots & \mathbf{a}_{2}^{T} \cdot \mathbf{a}_{n} \\ \vdots & \vdots & \vdots & \vdots \\ \mathbf{a}_{n}^{T} \cdot \mathbf{a}_{1} & \mathbf{a}_{n}^{T} \cdot \mathbf{a}_{2} & \ldots & \mathbf{a}_{n}^{T} \cdot \mathbf{a}_{n}\end{array}\right]$


\end{document}