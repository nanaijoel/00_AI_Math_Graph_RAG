\documentclass[10pt]{article}
\usepackage[ngerman]{babel}
\usepackage[utf8]{inputenc}
\usepackage[T1]{fontenc}
\usepackage{amsmath}
\usepackage{amsfonts}
\usepackage{amssymb}
\usepackage[version=4]{mhchem}
\usepackage{stmaryrd}
\usepackage{bbold}

\begin{document}
Anwendungen:

\begin{itemize}
  \item Mechanik: Arbeitsintegral gemäss
\end{itemize}


\begin{equation*}
\Delta W=\int_{s_{0}}^{s_{\mathrm{E}}}\langle\mathbf{F}(s), \hat{\mathbf{e}}\rangle \mathrm{d} s . \tag{2.60}
\end{equation*}


\begin{itemize}
  \item Elektrodynamik: Elektrische Spannung gemäss
\end{itemize}


\begin{equation*}
U=\int_{s_{0}}^{s_{\mathrm{E}}}\langle\mathbf{E}, \hat{\mathbf{e}}\rangle \mathrm{d} s . \tag{2.61}
\end{equation*}


Wir betrachten den folgenden Satz.\\
Satz 2.1 Linienintegral bei konstantem Anteil entlang der Bahn\\
Seien $n \in \mathbb{N}^{+}, \tau_{0}, \tau_{\mathrm{E}} \in \mathbb{R}$ mit $\tau_{0}<\tau_{\mathrm{E}}, \mathbf{s}:\left[\tau_{0}, \tau_{\mathrm{E}}\right] \rightarrow \mathbb{R}^{n}$ eine parametrisierte Kurve mit Bahnvektor $\hat{\mathbf{e}}(\tau)$ und Bogenlänge $\Delta s$ sowie $\mathbf{w}: \mathbb{R}^{n} \rightarrow \mathbb{R}^{n}$ ein Vektorfeld. Gilt entlang der Bahn der Kurve $\langle\mathbf{w}, \widehat{\mathbf{e}}\rangle=: C \equiv$ konst., dann beträgt das Linienintegral von $\mathbf{w}$ entlang der Kurve


\begin{equation*}
I=C \cdot \Delta s . \tag{2.62}
\end{equation*}


Beweis: Für das Linienintegral von w entlang der Kurve erhalten wir


\begin{align*}
\underline{\underline{I}} & =\int_{\tau_{0}}^{\tau_{\mathrm{E}}}\langle\mathbf{w}, \mathbf{v}\rangle \mathrm{d} \tau=\int_{s_{0}}^{s_{\mathrm{E}}}\langle\mathbf{w}, \hat{\mathbf{e}}\rangle \mathrm{d} s=\int_{s_{0}}^{s_{\mathrm{E}}} C \mathrm{~d} s=C \int_{s_{0}}^{s_{\mathrm{E}}} 1 \mathrm{~d} s=\left.C \cdot[s]\right|_{s_{0}} ^{s_{\mathrm{E}}}=C \cdot\left(s_{\mathrm{E}}-s_{0}\right) \\
& =\underline{\underline{C \cdot \Delta s} .} \tag{2.63}
\end{align*}


Damit haben wir den Satz bewiesen.


\end{document}