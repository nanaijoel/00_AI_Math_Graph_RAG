\documentclass[10pt]{article}
\usepackage[ngerman]{babel}
\usepackage[utf8]{inputenc}
\usepackage[T1]{fontenc}
\usepackage{amsmath}
\usepackage{amsfonts}
\usepackage{amssymb}
\usepackage[version=4]{mhchem}
\usepackage{stmaryrd}
\usepackage{bbold}

\begin{document}
\subsection*{6.2 Lineare Abbildungen}
\subsection*{6.2.1 Definition}
Mit Hilfe von Matrizen können geometrische Abbildungen beschrieben werden. Wir machen dazu zwei Definitionen.

Definition 6.13 Lineare Abbildung - Version 1\\
Seien $m, n \in \mathbb{N}^{+}$und $A \in \mathbb{M}(n, m, \mathbb{R})$. Eine Abbildung der Form


\begin{align*}
a: \mathbb{R}^{m} & \rightarrow \mathbb{R}^{n} \\
\mathbf{x} & \mapsto a(\mathbf{x}):=A \cdot \mathbf{x} \tag{6.32}
\end{align*}


heisst lineare Abbildung.

Definition 6.14 Lineare Abbildung - Version 2\\
Seien $m, n \in \mathbb{N}^{+}$. Eine lineare Abbildung ist eine Abbildung des Typs $a: \mathbb{R}^{m} \rightarrow \mathbb{R}^{n}$ mit der Eigenschaft, dass für alle $\mathbf{v}, \mathbf{w} \in \mathbb{R}^{m}$ und $x, y \in \mathbb{R}$ gilt


\begin{equation*}
a(x \cdot \mathbf{v}+y \cdot \mathbf{w})=x \cdot a(\mathbf{v})+y \cdot a(\mathbf{w}) \tag{6.33}
\end{equation*}


Bemerkungen:\\
i) Die beiden Definitionen sind äquivalent. Es ist leicht einzusehen, dass die linearen Abbildungen gemäss Definition 6.13 die Haupteigenschaft (6.33) aus Definition 6.14 erfüllen. Es gilt


\begin{align*}
\underline{\underline{a(x \cdot \mathbf{v}+y \cdot \mathbf{w})}} & =A \cdot(x \cdot \mathbf{v}+y \cdot \mathbf{w})=A \cdot x \cdot \mathbf{v}+A \cdot y \cdot \mathbf{w}=x \cdot A \cdot \mathbf{v}+y \cdot A \cdot \mathbf{w} \\
& =\underline{\underline{x \cdot a(\mathbf{v})+y \cdot a(\mathbf{w})}} \tag{6.34}
\end{align*}


ii) Die Matrix A, welche gemäss Definition 6.13 eine lineare Abbildung a beschreibt, wird Abbildungsmatrix genannt.\\
iii) Ist die Abbildungsmatrix A quadratisch, d.h. $m=n$, dann ist $a$ eine Selbstabbildung des Typs $a: \mathbb{R}^{n} \rightarrow \mathbb{R}^{n}$.\\
iv) Bekannte geometrische Abbildungen wie Streckungen, Projektionen, Spiegelungen und Rotationen sind lineare Abbildungen, welche jeweils durch eine Abbildungsmatrix ausgedrückt werden können.\\
v) Für alle linearen Abbildungen gilt offensichtlich


\begin{equation*}
\underline{\underline{a(0)}}=A \cdot 0=\underline{\underline{0 .}} \tag{6.35}
\end{equation*}


vi) Für $n=m=1$ gibt es eine historisch bedingte Begriffskollision zwischen einer linearen Funktion in der Analysis und einer linearen Abbildung in der linearen Algebra.


\begin{align*}
\text { Analysis: } & f(x)=m \cdot x+q \\
\text { Lineare Algebra: } & a(x)=A \cdot x \tag{6.36}
\end{align*}



\end{document}