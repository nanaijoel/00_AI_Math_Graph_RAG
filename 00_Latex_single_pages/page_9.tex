\documentclass[10pt]{article}
\usepackage[ngerman]{babel}
\usepackage[utf8]{inputenc}
\usepackage[T1]{fontenc}
\usepackage{amsmath}
\usepackage{amsfonts}
\usepackage{amssymb}
\usepackage[version=4]{mhchem}
\usepackage{stmaryrd}
\usepackage{bbold}

\begin{document}
\section*{Kapitel 2}
\section*{Vektoranalysis}
\subsection*{2.1 Grundlagen}
\subsection*{2.1.1 Skalarfelder}
\subsection*{2.1.1.1 Definition}
Wir betrachten die folgende Definition.\\
Definition 2.1 Reellwertige Funktion in mehreren reellen Variablen\\
Seien $n \in \mathbb{N}^{+}, A \subseteq \mathbb{R}^{n}$ und $B \subseteq \mathbb{R}$. Eine Funktion auf $A$ der Form $f: A \rightarrow B$ heisst reellwertige Funktion in $n$ reellen Variablen.

Bemerkungen:\\
i) Die reellen Variablen werden nach dem Funktionsnamen $f$ in runden Klammern aufgezählt, jeweils durch ein Semicolon getrennt.


\begin{align*}
n=1: & f(x) & =\ldots  \tag{2.1}\\
n=2: & f(x ; y) & =\ldots  \tag{2.2}\\
n=3: & f(x ; y ; z) & =\ldots  \tag{2.3}\\
\text { allg: } & f\left(x_{1} ; x_{2} ; \ldots ; x_{n}\right) & =\ldots \tag{2.4}
\end{align*}


ii) Wird die Definitionsmenge $A$ als Teilmenge eines geometrischen Raumes interpretiert, dann wird $f$ als Skalarfeld auf $A$ bezeichnet.\\
iii) Im allgemeinen können die reellen Variablen $x_{1}, x_{2}, \ldots, x_{n} \in \mathbb{R}$ und der Funktionswert $f\left(x_{1} ; x_{2} ; \ldots ; x_{n}\right)$ bliebige und auch unterschiedliche Masseinheiten tragen.\\
iv) Gilt $n \geq 2$ und enthält die Definitionsmenge $A$ einen auch noch so kleinen Würfel in nD, dann kann $f$ nicht injektiv sein.\\
v) Aus dem Wert von $n$ kann apriori keine Aussage über die Surjektivität der Funktion gemacht werden.\\
vi) Der Begriff Beschränktheit lässt sich direkt von 1D auf $n D$ übertragen.\\
vii) Der Begriff Monotonie lässt sich nicht von 1D auf nD übertragen.


\end{document}