\documentclass[10pt]{article}
\usepackage[ngerman]{babel}
\usepackage[utf8]{inputenc}
\usepackage[T1]{fontenc}
\usepackage{amsmath}
\usepackage{amsfonts}
\usepackage{amssymb}
\usepackage[version=4]{mhchem}
\usepackage{stmaryrd}
\usepackage{bbold}
\usepackage{graphicx}
\usepackage[export]{adjustbox}
\graphicspath{ {./images/} }

\begin{document}
\subsection*{5.4.4 Wurzeln}
Es gibt verschiedene Konventionen, was eine Wurzel mit komplexem Radikanden bedeuten soll.

\begin{enumerate}
  \item Alte Konvention: Die Wurzel ist als Menge aller Lösungen der entsprechenden PotenzGleichung aufzufassen. Das Element mit dem kleinsten Argument gemäss Basler-Variante wird dabei als Hauptwert bezeichnet und die anderen Elemente als Nebenwerte. Beispiele:
\end{enumerate}

\begin{itemize}
  \item $\sqrt{4}=\{\underbrace{4}_{\text {Hauptwert }}, \underbrace{-4}_{\text {Nebenwert }}\}$
  \item $\sqrt{-4}=\{\underbrace{2 \cdot \mathrm{i}}_{\text {Hauptwert }}, \underbrace{-2 \cdot \mathrm{i}}_{\text {Nebenwert }}\}$
  \item $\sqrt{4 \cdot \mathrm{i}}=\{\underbrace{2 \cdot \mathrm{e}_{\text {Nebenwert }}^{\mathrm{i} \cdot \frac{\pi}{4}}}_{\text {Hauptwert }} \underbrace{2 \cdot \mathrm{e}^{-\mathrm{i} \cdot \frac{\pi}{4}}}\}$
  \item $\sqrt[3]{-27}=\{\underbrace{3 \cdot \cdot^{\mathrm{i} \cdot \frac{\pi}{3}}}_{\text {Hauptwert }}, \underbrace{-3,3 \cdot \mathrm{e}^{-\mathrm{i} \cdot \frac{\pi}{3}}}_{\text {Nebenwerte }}\}$
\end{itemize}

\begin{enumerate}
  \setcounter{enumi}{1}
  \item Neue Konvention: Es wird zwischen reellen und komplexen Wurzeln unterschieden. Beispiele:
\end{enumerate}

\begin{itemize}
  \item $\sqrt{4}=2 \quad$ (reelle Lösung)
  \item $\sqrt{-4}$ existiert nicht
  \item $\sqrt{4 \cdot \mathrm{i}}$ existiert nicht
  \item $\sqrt[3]{-27}=-3 \quad$ (reelle Lösung)
  \item $\sqrt{4}^{\mathbb{C}}=2$ (reelle Lösung, Hauptwert)
  \item $\sqrt{-4}^{\mathbb{C}}=2 \cdot \mathrm{i}$ (Hauptwert)\\
\includegraphics[max width=\textwidth, center]{2025_05_07_a91aa67dcaef93a320c0g-1}
  \item $\sqrt[3]{-27}^{\mathbb{C}}=3 \cdot \mathrm{e}^{\mathrm{i} \cdot \frac{\pi}{3}} \quad$ (Hauptwert)
\end{itemize}

\begin{enumerate}
  \setcounter{enumi}{2}
  \item Allgemeine Praxis: Es werden nur reelle Radikanden zugelassen und reelle Lösungen vor Hauptwerten bevorzugt. Ansonsten werden keine Wurzeln verwendet. Beispiele:
\end{enumerate}

\begin{itemize}
  \item $\sqrt{4}=2 \quad$ (reelle Lösung, Hauptwert)
  \item $\sqrt{-4}=2 \cdot \mathrm{i} \quad$ (Hauptwert)
  \item $\sqrt[3]{-27}=-3 \quad$ (reelle Lösung)
\end{itemize}

Generell ist im Umgang mit Wurzeln Vorsicht geboten, wie die folgende Rechnung zeigt. Man könnte vermeintlich zu Schluss kommen


\begin{equation*}
1=\sqrt{1}=\sqrt{(-1) \cdot(-1)}=\sqrt{-1} \cdot \sqrt{-1}=\mathrm{i} \cdot \mathrm{i}=\mathrm{i}^{2}=-1 . \tag{5.47}
\end{equation*}


Das dritte Gleichheitszeichen in der obigen Rechnung ist falsch. Allgmein gilt für alle WurzelKonventionen\\
$\sqrt{x \cdot y}=\sqrt{x} \cdot \sqrt{y}$ für $\quad x, y \in \mathbb{R}^{+}$\\
$\sqrt{w \cdot z}= \pm \sqrt{w} \cdot \sqrt{z}$ für $w, z \in \mathbb{C}$,\\
wobei das Vorzeichen von Fall zu Fall entschieden werden muss.


\end{document}