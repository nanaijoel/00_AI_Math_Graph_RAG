\documentclass[10pt]{article}
\usepackage[ngerman]{babel}
\usepackage[utf8]{inputenc}
\usepackage[T1]{fontenc}
\usepackage{amsmath}
\usepackage{amsfonts}
\usepackage{amssymb}
\usepackage[version=4]{mhchem}
\usepackage{stmaryrd}
\usepackage{bbold}
\usepackage{graphicx}
\usepackage[export]{adjustbox}
\graphicspath{ {./images/} }

\begin{document}
\subsection*{2.3 Mehrfach-Integrale}
\subsection*{2.3.1 Zweifach-Integrale}
\subsection*{2.3.1.1 Definition}
Wir betrachten ein Gebiet $G \subset \mathbb{R}^{2}$ und eine Funktion $f: \mathbb{R}^{2} \rightarrow \mathbb{R}$. Der Graph von $f$ ist im folgenden $x-y$ - $z$-Diagramm dargestellt.\\
\includegraphics[max width=\textwidth, center]{2025_05_07_4f86020a2e29be38a40fg-1}

Wir betrachten die folgende Definition.\\
Definition 2.10 Integration über ein Gebiet\\
Seien $G \subset \mathbb{R}^{2}$ ein Gebiet und $f: \mathbb{R}^{2} \rightarrow \mathbb{R}$ eine integrierbare Funktion. Das Integral von $f$ über das Gebiet $G$ ist


\begin{equation*}
\int_{G} f \mathrm{~d} A: \equiv \text { Volumen zwischen } G \text { und dem Graphen von } f \text {. } \tag{2.64}
\end{equation*}


Bemerkungen:\\
i) Wie in 1D lassen sich präzise Bedingungen definieren, die an eine Teilmenge $G \subset \mathbb{R}^{2}$ und an eine Funktion $f$ gestellt werden müssen, damit das Integral existiert. Ohne im Detail auf diese Voraussetzungen einzugehen, bezeichnen wir eine Teilmenge $G \subset \mathbb{R}^{2}$ als Gebiet und eine Funktion $f: \mathbb{R}^{2} \rightarrow \mathbb{R}$ als integrierbar, falls alle notwendigen Kriterien erfüllt sind.\\
ii) Für die Masseinheit erhalten wir


\begin{equation*}
\underline{\underline{\left[\int_{G} f \mathrm{~d} A\right]}}=[V]=[A] \cdot[f]=\underline{\underline{[x] \cdot[y] \cdot[f]=[x] \cdot[y] \cdot[z]} .} \tag{2.65}
\end{equation*}


iii) Ein Integral über ein Gebiet in 2D wird synonym auch als Zweifach-Integral oder DoppelIntegral bezeichnet.


\end{document}