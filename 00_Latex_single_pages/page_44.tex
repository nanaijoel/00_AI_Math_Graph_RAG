\documentclass[10pt]{article}
\usepackage[ngerman]{babel}
\usepackage[utf8]{inputenc}
\usepackage[T1]{fontenc}
\usepackage{amsmath}
\usepackage{amsfonts}
\usepackage{amssymb}
\usepackage[version=4]{mhchem}
\usepackage{stmaryrd}
\usepackage{bbold}

\begin{document}
\subsection*{2.5.5 Rotation}
Wir betrachten die folgende Definition.\\
Definition 2.23 Rotation in 2D\\
Sei $\mathbf{v}: \mathbb{R}^{2} \rightarrow \mathbb{R}^{2}$ ein differentierbares Vektorfeld mit Komponenten

\[
\mathbf{v}\left(x^{1} ; x^{2}\right)=\left[\begin{array}{c}
v^{1}\left(x^{1} ; x^{2}\right)  \tag{2.154}\\
v^{2}\left(x^{1} ; x^{2}\right)
\end{array}\right] .
\]

Die Rotation von $\mathbf{v}$ ist


\begin{equation*}
\operatorname{rot}(\mathbf{v}):=v^{2}{ }_{, 1}-v^{1}{ }_{, 2} . \tag{2.155}
\end{equation*}


Wir betrachten die folgende Definition.\\
Definition 2.24 Rotation in 3D\\
Sei $\mathbf{v}: \mathbb{R}^{3} \rightarrow \mathbb{R}^{3}$ ein differentierbares Vektorfeld mit Komponenten

\[
\mathbf{v}\left(x^{1} ; x^{2} ; x^{3}\right)=\left[\begin{array}{c}
v^{1}\left(x^{1} ; x^{2} ; x^{3}\right)  \tag{2.156}\\
v^{2}\left(x^{1} ; x^{2} ; x^{3}\right) \\
v^{3}\left(x^{1} ; x^{2} ; x^{3}\right)
\end{array}\right] .
\]

Die Rotation von $\mathbf{v}$ ist

\[
\operatorname{rot}(\mathbf{v}):=\left[\begin{array}{c}
v^{3}, 2-v^{2},{ }_{, 3}  \tag{2.157}\\
v^{1}, 3-v^{3}, 1 \\
v^{2}, v_{1}-v_{, 2}^{1}
\end{array}\right] .
\]

Bemerkungen:\\
i) Die Rotation eines Vektorfeldes ist eine spezielle Konstruktion in 2D und 3D.\\
ii) Die Rotation eines Vektorfeldes in 2D ist ein Skalarfeld.\\
iii) Die Rotation eines Vektorfeldes in 3D ist ein Vektorfeld in 3D.\\
iv) Die Rotation eines Vektorfeldes ist ein Mass für dessen Wirbeldichte.\\
v) Ein Vektorfeld $\mathbf{v}$ heisst wirbelfrei, falls gilt


\begin{equation*}
\operatorname{rot}(\mathbf{v})=0 . \tag{2.158}
\end{equation*}


vi) In 3D steht rot(v) senkrecht auf der "Wirbel-Ebene" von v, analog zum Drehimpuls einer rotierenden Scheibe.\\
vii) Im Englischen wird die Rotation als curl bezeichnet und entsprechend auch so in Formeln notiert, d.h. durch $\operatorname{curl}(\mathbf{v})$.


\end{document}