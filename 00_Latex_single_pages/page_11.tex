\documentclass[10pt]{article}
\usepackage[ngerman]{babel}
\usepackage[utf8]{inputenc}
\usepackage[T1]{fontenc}
\usepackage{amsmath}
\usepackage{amsfonts}
\usepackage{amssymb}
\usepackage[version=4]{mhchem}
\usepackage{stmaryrd}
\usepackage{bbold}
\usepackage{graphicx}
\usepackage[export]{adjustbox}
\graphicspath{ {./images/} }

\begin{document}
\subsection*{2.1.1.2 Level-Mengen}
Wir betrachten die folgende Definition.

\section*{Definition 2.2 Level-Menge}
Seien $n \in \mathbb{N}^{+}, A \subseteq \mathbb{R}^{n}, B \subseteq \mathbb{R}$ und $f$ eine reellwertige Funktion der Form $f: A \rightarrow B$. Die Level-Menge zu einem Level $L \in \mathbb{R}$ ist das Urbild


\begin{equation*}
f^{-1}(\{L\})=\{p \in A \mid f(p)=L\} . \tag{2.17}
\end{equation*}


Bemerkungen:\\
i) Die Level-Menge zum Level $L$ besteht aus allen Elementen der Definitionsmenge, an welchen die Funktion den Wert des Levels $L$ hat.\\
ii) Die Begriffe Level-Menge und Niveau-Menge sind synonym.\\
iii) Für $n=2$ sind die Level-Mengen meistens Kurven in der $x$ - $y$-Ebene, welche als LevelLinien bzw. Niveau-Linien bezeichnet werden.\\
iv) Für $n=3$ sind die Level-Mengen meistens Flächen in 3D, welche als Level-Flächen bzw. Niveau-Flächen bezeichnet werden.\\
v) Insbesondere für $n \in\{2,3\}$ ist die Struktur der Level-Mengen sehr nützlich für die Visualisierung (Plot) von $f$.

Beispiele:

\begin{itemize}
  \item Die Level-Mengen des Abstandes zum Ursprung sind Sphären mit Mittelpunkt am Ursprung.\\
\includegraphics[max width=\textwidth, center]{2025_05_07_42d0546c429a162d1d00g-1}
  \item Die Level-Linien der Höhe über Meer sind die Höhenlinien auf einer topographischen Karte.\\
\includegraphics[max width=\textwidth, center]{2025_05_07_42d0546c429a162d1d00g-1(1)}
\end{itemize}

\end{document}