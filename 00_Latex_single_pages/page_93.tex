\documentclass[10pt]{article}
\usepackage[ngerman]{babel}
\usepackage[utf8]{inputenc}
\usepackage[T1]{fontenc}
\usepackage{amsmath}
\usepackage{amsfonts}
\usepackage{amssymb}
\usepackage[version=4]{mhchem}
\usepackage{stmaryrd}
\usepackage{bbold}

\begin{document}
Beispiele:

\begin{itemize}
  \item $\left[\begin{array}{rrr}2 & -1 & 3 \\ 7 & 5 & -4\end{array}\right]^{T}=\left[\begin{array}{rr}2 & 7 \\ -1 & 5 \\ 3 & -4\end{array}\right]$
  \item $\left[\begin{array}{rr}2 & -1 \\ 7 & 5\end{array}\right]^{T}=\left[\begin{array}{rr}2 & 7 \\ -1 & 5\end{array}\right]$
  \item $\left[\begin{array}{ll}2 & -1\end{array}\right]^{T}=\left[\begin{array}{r}2 \\ -1\end{array}\right]$
\end{itemize}

\subsection*{6.1.2.4 Matrix-Multiplikation}
Zwei reelle Matrizen mit passenden Dimensionen lassen sich multiplizieren.

\section*{Definition 6.5 Matrix-Produkt}
Seien $l, m, n \in \mathbb{N}^{+}, A \in \mathbb{M}(l, m, \mathbb{R})$ und $B \in \mathbb{M}(m, n, \mathbb{R})$, dann ist das Matrix-Produkt


\begin{equation*}
C=A \cdot B \quad \text { definiert durch } \quad C_{j}^{i}:=\sum_{s=1}^{m} A_{s}^{i} \cdot B_{j}^{s} . \tag{6.11}
\end{equation*}


Bemerkungen:\\
i) Das Matrix-Produkt $A \cdot B$ ist also genau dann definiert, wenn $A$ so viele Spalten wie $B$ Zeilen hat.\\
ii) Für $A \in \mathbb{M}(l, m, \mathbb{R})$ und $B \in \mathbb{M}(m, n, \mathbb{R})$ gilt $A \cdot B \in \mathbb{M}(l, n, \mathbb{R})$.\\
iii) Die Komponenten des Matrix-Produkts $A \cdot B$ sind gerade die Kontraktionen ("SkalarProdukte") der Zeilen von $A$ mit den Spalten von $B$.\\
iv) Die Berechnung eines Matrix-Produkts ist im allgemeinen recht aufwändig.\\
v) Beispiel-Codes zur Berechnung von Matrix-Produkten mit gängiger Software.

\begin{center}
\begin{tabular}{|l|l|}
\hline
MATLAB/Octave & $M=A * B$ \\
\hline
Mathematica/WolframAlpha & $\mathrm{M}=\mathrm{A} . \mathrm{B}$ \\
\hline
Python/Numpy & $\mathrm{M}=\mathrm{A} @ \mathrm{~B}$ \\
\hline
Python/Sympy & $\mathrm{M}=\mathrm{A} * \mathrm{~B}$ \\
\hline
\end{tabular}
\end{center}

Beispiele:

\begin{itemize}
  \item $\left[\begin{array}{ll}1 & 2 \\ 3 & 4\end{array}\right] \cdot\left[\begin{array}{ll}5 & 6 \\ 7 & 8\end{array}\right]=\left[\begin{array}{ll}1 \cdot 5+2 \cdot 7 & 1 \cdot 6+2 \cdot 8 \\ 3 \cdot 5+4 \cdot 7 & 3 \cdot 6+4 \cdot 8\end{array}\right]=\left[\begin{array}{ll}19 & 22 \\ 43 & 50\end{array}\right]$
  \item $\left[\begin{array}{ll}5 & 6 \\ 7 & 8\end{array}\right] \cdot\left[\begin{array}{ll}1 & 2 \\ 3 & 4\end{array}\right]=\left[\begin{array}{ll}5 \cdot 1+6 \cdot 3 & 5 \cdot 2+6 \cdot 4 \\ 7 \cdot 1+8 \cdot 3 & 7 \cdot 2+8 \cdot 4\end{array}\right]=\left[\begin{array}{ll}23 & 34 \\ 31 & 46\end{array}\right]$
\end{itemize}

\end{document}