\documentclass[10pt]{article}
\usepackage[ngerman]{babel}
\usepackage[utf8]{inputenc}
\usepackage[T1]{fontenc}
\usepackage{amsmath}
\usepackage{amsfonts}
\usepackage{amssymb}
\usepackage[version=4]{mhchem}
\usepackage{stmaryrd}

\begin{document}
Bemerkungen:\\
i) Linear modifizierte Elementar-Integrale machen ca. 90\% aller Integrale in Alltag, Naturwissenschaft, Technik und Wirtschaft aus.\\
ii) Die Methode der linearen Modifikation ist ein einfacher Spezialfall der allgemeineren Methode der Substitution.

Liste mit den wichtigsten linear modifizierten Standard-Integralen:

\begin{itemize}
  \item $\int(m \cdot x+q)^{p} \mathrm{~d} x=\frac{1}{m \cdot(p+1)} \cdot(m \cdot x+q)^{p+1}+c$
  \item $\int a^{m \cdot x+q} \mathrm{~d} x=\frac{1}{m \cdot \ln (a)} \cdot a^{m \cdot x+q}+c$
  \item $\int y_{0} \cdot a^{\frac{x-x_{0}}{\Sigma}} \mathrm{~d} x=\frac{\Sigma}{\ln (a)} \cdot y_{0} \cdot a^{\frac{x-x_{0}}{\Sigma}}+c$
  \item $\int A \cdot \sin (\omega \cdot t+\varphi) \mathrm{d} t=-\frac{A}{\omega} \cdot \cos (\omega \cdot t+\varphi)+c$
\end{itemize}

\subsection*{1.1.2 Integration mit Software}
Siehe Demonstrationen mit Python/Numpy \& Python/Sympy.


\end{document}