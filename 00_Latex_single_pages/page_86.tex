\documentclass[10pt]{article}
\usepackage[ngerman]{babel}
\usepackage[utf8]{inputenc}
\usepackage[T1]{fontenc}
\usepackage{amsmath}
\usepackage{amsfonts}
\usepackage{amssymb}
\usepackage[version=4]{mhchem}
\usepackage{stmaryrd}
\usepackage{bbold}

\begin{document}
Beweis: Die Aussage folgt sofort aus der Euler-Formel.\\
Bemerkungen:\\
i) Jede komplexe Zahl $z \in \mathbb{C}$ lässt sich auf drei Arten darstellen, nämlich\\
$z=\underbrace{x+y \cdot \mathrm{i}}_{\text {arithmetische Form }}=\underbrace{r \cdot \operatorname{cis}(\varphi)}_{\text {trigonometrische Form }}=\underbrace{r \cdot \mathrm{e}^{\mathrm{i} \cdot \varphi}}_{\text {exponentielle Form }}$.\\
ii) Man bezeichnet die Schreibweise $z=r \cdot \mathrm{e}^{\mathrm{i} \cdot \varphi}$ nur dann als exponentielle Form, wenn gilt $r=|z| \geq 0$. Der Winkel $\varphi \in \mathbb{R}$ darf jedoch beliebig gewählt werden, d.h. $\varphi$ muss nicht unbedingt das Argument gemäss einer der beiden Varianten sein.\\
iii) In exponentieller Form lässt sich die Multiplikation von zwei komplexen Zahlen auf einfache Weise in der Gauss-Ebene geometrisch interpretieren. Es gilt\\
$\underline{\underline{z_{1}} \cdot z_{2}}=r_{1} \cdot \mathrm{e}^{\mathrm{i} \cdot \varphi_{1}} \cdot r_{2} \cdot \mathrm{e}^{\mathrm{i} \cdot \varphi_{2}}=r_{1} \cdot r_{2} \cdot \mathrm{e}^{\mathrm{i} \cdot \varphi_{1}+\mathrm{i} \cdot \varphi_{2}}=\underline{\underline{r_{1} \cdot r_{2}} \cdot \mathrm{e}^{\mathrm{i} \cdot\left(\varphi_{1}+\varphi_{2}\right)}}$.\\
iv) In exponentieller Form lässt sich das Potenzieren einer komplexen Zahl mit einem strikt positiven Exponenten $p>0$ auf einfache Weise in der Gauss-Ebene geometrisch interpretieren. Es gilt\\
$\underline{\underline{z^{p}}}=\left(r \cdot \mathrm{e}^{\mathrm{i} \cdot \varphi}\right)^{p}=r^{p} \cdot\left(\mathrm{e}^{\mathrm{i} \cdot \varphi}\right)^{p}=\underline{\underline{r^{p}} \cdot \mathrm{e}^{\mathrm{i} \cdot p \cdot \varphi}}$.

\subsection*{5.4.3 Potenz-Gleichungen}
Seien $w \in \mathbb{C}$ und $n \in \mathbb{N}^{+}$. Wir betrachten eine Potenz-Gleichung der Form


\begin{equation*}
z^{n}=w . \tag{5.34}
\end{equation*}


Dabei suchen wir alle komplexen Lösungen $z \in \mathbb{C}$. Beispiele:

\begin{itemize}
  \item Wir betrachten die Potenz-Gleichung
\end{itemize}


\begin{equation*}
z^{2}=4 \Rightarrow z_{1,2}= \pm \sqrt{4}= \pm 2 \Rightarrow \underline{\mathbb{L}=\{-2,2\}} . \tag{5.35}
\end{equation*}


\begin{itemize}
  \item Wir betrachten die Potenz-Gleichung
\end{itemize}


\begin{equation*}
z^{2}=-4 \Rightarrow z_{1,2}= \pm \mathrm{i} \cdot \sqrt{4}= \pm 2 \cdot \mathrm{i} . \Rightarrow \underline{\underline{\mathbb{Q}}=\{-2 \cdot \mathrm{i}, 2 \cdot \mathrm{i}\}} . \tag{5.36}
\end{equation*}


\begin{itemize}
  \item Wir betrachten die Potenz-Gleichung
\end{itemize}


\begin{equation*}
z^{3}=-27=|-27| \cdot \mathrm{e}^{\mathrm{i} \cdot \arg (-27)}=27 \cdot \mathrm{e}^{\mathrm{i} \cdot \pi}=27 \cdot \mathrm{e}^{\mathrm{i} \cdot(\pi+(k-1) \cdot 2 \pi)} \quad \text { für } \quad k \in \mathbb{Z} . \tag{5.37}
\end{equation*}


Wir erhalten die folgenden drei Lösungen


\begin{equation*}
k=1: \quad z_{1}=\sqrt[3]{27} \cdot \mathrm{e}^{\mathrm{i} \cdot \frac{\pi+0 \cdot 2 \pi}{3}}=3 \cdot \mathrm{e}^{\mathrm{i} \cdot \frac{\pi}{3}} \tag{5.38}
\end{equation*}



\begin{equation*}
k=2: \quad z_{2}=\sqrt[3]{27} \cdot \mathrm{e}^{\mathrm{i} \cdot \frac{\pi+1 \cdot 2 \pi}{3}}=3 \cdot \mathrm{e}^{\mathrm{i} \cdot \pi}=-3 \tag{5.39}
\end{equation*}



\begin{equation*}
k=3: \quad z_{3}=\sqrt[3]{27} \cdot \mathrm{e}^{\mathrm{i} \cdot \frac{\pi+2 \cdot 2 \pi}{3}}=3 \cdot \mathrm{e}^{\mathrm{i} \cdot \frac{5 \pi}{3}}=3 \cdot \mathrm{e}^{-\mathrm{i} \cdot \frac{\pi}{3}} \tag{5.40}
\end{equation*}



\end{document}