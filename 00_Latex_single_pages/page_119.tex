\documentclass[10pt]{article}
\usepackage[ngerman]{babel}
\usepackage[utf8]{inputenc}
\usepackage[T1]{fontenc}
\usepackage{amsmath}
\usepackage{amsfonts}
\usepackage{amssymb}
\usepackage[version=4]{mhchem}
\usepackage{stmaryrd}
\usepackage{bbold}

\begin{document}
iv) In jedem Fall gilt $\sigma(p) \in\{-1,+1\}$. Es werden folgende Bezeichnungen verwendet:


\begin{align*}
& \sigma(p)=-1 \Leftrightarrow p \text { ist eine ungerade Permutation }  \tag{6.107}\\
& \sigma(p)=+1 \Leftrightarrow p \text { ist eine gerade Permutation. }
\end{align*}


Mit diesen Vorbreitungen sind wir nun bereit, die Determinante einer beliebigen quadratischen Matrix mit Hilfe der Leibniz-Formel zu definieren.

\section*{Definition 6.21 Determinante}
Seien $n \in \mathbb{N}^{+}$und $A \in \mathbb{M}(n, n, \mathbb{R})$. Die Determinante der Matrix $A$ ist die reelle Zahl


\begin{equation*}
\operatorname{det}(A):=\sum_{p \in S_{n}} \sigma(p) \cdot A^{p(1)}{ }_{1} \cdot \ldots \cdot A^{p(n)}{ }_{n} . \tag{6.108}
\end{equation*}


Bemerkungen:\\
i) Gemäss Leibniz-Formel (6.108) wird die Determinante einer quadratischen Matrix $A \in$ $M(n, n, \mathbb{R})$ durch folgende Schritte gebildet.

S1 Man sucht sich alle Möglichkeiten, um $n$ Komponenten aus $A$ auszuwählen, so dass aus jeder Zeile und jeder Spalte genau eine Komponente vertreten ist.\\
S2 Man berechnet für jede Möglichkeit aus Schritt S1 das Produkt der gefundenen Komponenten.\\
S3 Man multipliziert jedes dieser Produkte aus Schritt S2 mit dem Vorzeichen der Permutation zwischen Zeilen- und Spaltenindizes.\\
S4 Man addiert alle Produkte mit den entsprechenden Vorzeichen.\\
ii) Für $n=2$ gibt es nur die folgenden $n!=2$ ! $=1 \cdot 2=2$ Permutationen.

\begin{center}
\begin{tabular}{|c|cc|cc|}
\hline
$k$ & $p_{k}(1)$ & $p_{k}(2)$ & $\rho\left(p_{k}\right)$ & $\sigma\left(p_{k}\right)$ \\
\hline\hline
1 & 1 & 2 & 0 & +1 \\
2 & 2 & 1 & 1 & -1 \\
\hline
\end{tabular}
\end{center}

Die Determinante einer $2 \times 2$-Matrix ist folglich


\begin{equation*}
\operatorname{det}(A)=A_{1}^{1} \cdot A^{2}{ }_{2}-A_{1}^{2} \cdot A^{1}{ }_{2} . \tag{6.110}
\end{equation*}


iii) Für $n=3$ gibt es die folgenden $n!=3!=1 \cdot 2 \cdot 3=6$ Permutationen.

\begin{center}
\begin{tabular}{|c|ccc|cc|}
\hline
$k$ & $p_{k}(1)$ & $p_{k}(2)$ & $p_{k}(3)$ & $\rho\left(p_{k}\right)$ & $\sigma\left(p_{k}\right)$ \\
\hline\hline
1 & 1 & 2 & 3 & 0 & +1 \\
2 & 2 & 3 & 1 & 2 & +1 \\
3 & 3 & 1 & 2 & 2 & +1 \\
4 & 3 & 2 & 1 & 3 & -1 \\
5 & 1 & 3 & 2 & 1 & -1 \\
6 & 2 & 1 & 3 & 1 & -1 \\
\hline
\end{tabular}
\end{center}


\end{document}