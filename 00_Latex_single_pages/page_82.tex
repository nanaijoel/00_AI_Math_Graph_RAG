\documentclass[10pt]{article}
\usepackage[ngerman]{babel}
\usepackage[utf8]{inputenc}
\usepackage[T1]{fontenc}
\usepackage{graphicx}
\usepackage[export]{adjustbox}
\graphicspath{ {./images/} }
\usepackage{amsmath}
\usepackage{amsfonts}
\usepackage{amssymb}
\usepackage[version=4]{mhchem}
\usepackage{stmaryrd}
\usepackage{bbold}

\begin{document}
\subsection*{5.2 Gauss-Ebene}
\subsection*{5.2.1 Elementare Entsprechungen}
Wir betrachten das folgende Diagramm.\\
\includegraphics[max width=\textwidth, center]{2025_05_07_4606323b5212d7e51e00g-1(2)}

Bemerkungen:\\
i) Die Re-Achse entspricht den reellen Zahlen.\\
ii) Die Gauss-Ebene entspricht den komplexen Zahlen gemäss

\[
x+y \cdot \mathrm{i} \longleftrightarrow\left[\begin{array}{l}
x  \tag{5.15}\\
y
\end{array}\right] .
\]

iii) Es gibt die folgenden weiteren Entsprechungen zwischen $\mathbb{C}$ und $\mathbb{R}^{2}$ :

\begin{center}
\begin{tabular}{|l|l|}
\hline
Addition in $\mathbb{C}$ & Addition in $\mathbb{R}^{2}$ \\
\hline
Betrag in $\mathbb{C}$ & Betrag in $\mathbb{R}^{2}$ \\
\hline
Konjugation in $\mathbb{C}$ & Spiegelung an Re-Achse in $\mathbb{R}^{2}$ \\
\hline
\end{tabular}
\end{center}

\subsection*{5.2.2 Trigonometrische Form}
Die Punkte in der Gauss-Ebene lassen sich sowohl durch kartesische Koordinaten als auch durch Polarkoordinaten beschreiben. Wir betrachten dazu folgende Definition.

\section*{Definition 5.3 Argument}
Sei $z \in \mathbb{C} \backslash\{0\}$. Das Argument von $\arg (z)$ ist der Winkel in der Gauss-Ebene zwischen der $x$-Achse und der Verbindungslinie vom Ursprung zu $z$ nach folgenden Varianten.\\
(a) Basler-Variante: $\varphi \in[0,2 \pi[$.\\
(b) Zürcher-Variante: $\varphi \in]-\pi, \pi]$.\\
\includegraphics[max width=\textwidth, center]{2025_05_07_4606323b5212d7e51e00g-1}\\
\includegraphics[max width=\textwidth, center]{2025_05_07_4606323b5212d7e51e00g-1(1)}


\end{document}