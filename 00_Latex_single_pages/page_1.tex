\documentclass[10pt]{article}
\usepackage[ngerman]{babel}
\usepackage[utf8]{inputenc}
\usepackage[T1]{fontenc}
\usepackage{amsmath}
\usepackage{amsfonts}
\usepackage{amssymb}
\usepackage[version=4]{mhchem}
\usepackage{stmaryrd}
\usepackage{bbold}

\begin{document}
\section*{Kapitel 1}
\section*{Integrale in Anwendungen}
\subsection*{1.1 Einfache Berechnung von Integralen}
\subsection*{1.1.1 Lineare Modifikation}
Wir betrachten den folgenden Satz.

\section*{Satz 1.1 Integration durch lineare Modifikation}
Seien $f: \mathbb{R} \rightarrow \mathbb{R}$ eine integrierbare Funktion mit Stammfunktion $F: \mathbb{R} \rightarrow \mathbb{R}$ und $m, q, x_{0}, x_{\mathrm{E}} \in \mathbb{R}$ mit $m \neq 0$ und $x_{0}<x_{\mathrm{E}}$, dann gilt folgendes.\\
(a) $\int f(m \cdot x+q) \mathrm{d} x=\frac{1}{m} \cdot F(m \cdot x+q)+c$\\
(b) $\int_{x_{0}}^{x_{\mathrm{E}}} f(m \cdot x+q) \mathrm{d} x=\frac{1}{m} \cdot\left(F\left(m \cdot x_{\mathrm{E}}+q\right)-F\left(m \cdot x_{0}+q\right)\right)$

Beweis: Übung.\\
Beispiele:

\begin{itemize}
  \item Wir betrachten ein unbestimmtes Elementarintegral und eine lineare Modifikation.
\end{itemize}


\begin{align*}
\text { elementar: } & F(x)=\int \cos (x) \mathrm{d} x=\sin (x)+c \\
\text { linear modifiziert: } & F(x)=\int \cos (2 x+3) \mathrm{d} x=\frac{1}{2} \cdot \sin (2 x+3)+c \tag{1.1}
\end{align*}


\begin{itemize}
  \item Wir betrachten das unbestimmte Integral
\end{itemize}


\begin{equation*}
\underline{\underline{F(x)}}=\int(7 x-2)^{3} \mathrm{~d} x=\frac{1}{7} \cdot \frac{1}{4} \cdot(7 x-2)^{4}+c=\underline{\underline{\frac{1}{28}} \cdot(7 x-2)^{4}+c .} \tag{1.2}
\end{equation*}


\begin{itemize}
  \item Wir betrachten das unbestimmte Integral
\end{itemize}


\begin{equation*}
\underline{\underline{F(x)}}=\int 3^{2 x+9} \mathrm{~d} x=\frac{1}{2} \cdot \frac{1}{\ln (3)} \cdot 3^{2 x+9}+c=\underline{\underline{\frac{1}{\ln (3)}} \cdot 3^{2 x+9}+c .} \tag{1.3}
\end{equation*}



\end{document}