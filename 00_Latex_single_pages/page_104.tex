\documentclass[10pt]{article}
\usepackage[ngerman]{babel}
\usepackage[utf8]{inputenc}
\usepackage[T1]{fontenc}
\usepackage{amsmath}
\usepackage{amsfonts}
\usepackage{amssymb}
\usepackage[version=4]{mhchem}
\usepackage{stmaryrd}
\usepackage{bbold}

\begin{document}
\subsection*{6.2.3 Spalten-Vektor-Konstruktionsverfahren}
Wir betrachten eine lineare Abbildung $a: \mathbb{R}^{m} \rightarrow \mathbb{R}^{n}$ deren geometrische Wirkung auf Vektoren in $\mathbb{R}^{m}$ wir kennen, z.B. eine Streckung, Projektion, Spiegelung oder Drehung. Wie lässt sich nun die zugehörige Abbildungsmatrix $A$ berechnen? Um die überraschend einfache Antwort auf diese Frage zu finden, betrachten wir zunächst eine allgemeine $2 \times 2$-Matrix sowie die StandardEinheitsvektoren in $\mathbb{R}^{2}$, d.h.\\
$A=\left[\begin{array}{ll}a & b \\ c & d\end{array}\right] \quad$ bzw. $\quad \hat{\mathbf{e}}_{1}=\left[\begin{array}{l}1 \\ 0\end{array}\right], \quad \hat{\mathbf{e}}_{2}=\left[\begin{array}{l}0 \\ 1\end{array}\right]$.\\
Es gilt

\[
a\left(\hat{\mathbf{e}}_{1}\right)=A \cdot \hat{\mathbf{e}}_{1}=\left[\begin{array}{ll}a & b  \tag{6.43}\\ c & d\end{array}\right] \cdot\left[\begin{array}{l}1 \\ 0\end{array}\right]=\left[\begin{array}{l}a \cdot 1+b \cdot 0 \\ c \cdot 1+d \cdot 0\end{array}\right]=\left[\begin{array}{l}a \\ c\end{array}\right]
\]

$a\left(\hat{\mathbf{e}}_{2}\right)=A \cdot \hat{\mathbf{e}}_{2}=\left[\begin{array}{ll}a & b \\ c & d\end{array}\right] \cdot\left[\begin{array}{l}0 \\ 1\end{array}\right]=\left[\begin{array}{l}a \cdot 0+b \cdot 1 \\ c \cdot 0+d \cdot 1\end{array}\right]=\left[\begin{array}{l}b \\ d\end{array}\right]$.\\
Die Bilder der Standard-Einheitsvektoren unter der linearen Abbildung a sind demnach gerade die Spalten der Abbildungsmatrix A. Dies ist eine allgemeingültige Tatsache, die für beliebige lineare Abbildungen gilt.

Satz 6.5 Spalten-Vektor-Satz\\
Seien $m, n \in \mathbb{N}^{+}, a: \mathbb{R}^{m} \rightarrow \mathbb{R}^{n}$ eine lineare Abbildung mit Abbildungsmatrix $A \in \mathbb{M}(n, m, \mathbb{R})$ und

\[
\hat{\mathbf{e}}_{1}=\left[\begin{array}{c}
1  \tag{6.45}\\
0 \\
0 \\
\vdots \\
0
\end{array}\right], \hat{\mathbf{e}}_{2}=\left[\begin{array}{c}
0 \\
1 \\
0 \\
\vdots \\
0
\end{array}\right], \ldots, \hat{\mathbf{e}}_{m}=\left[\begin{array}{c}
0 \\
0 \\
\vdots \\
0 \\
1
\end{array}\right] .
\]

Dann gilt

\[
A=\left[\begin{array}{llll}
a\left(\hat{\mathbf{e}}_{1}\right) & a\left(\hat{\mathbf{e}}_{2}\right) & \ldots & a\left(\hat{\mathbf{e}}_{m}\right) \tag{6.46}
\end{array}\right] .
\]

Beweis: Die Bilder der Einheitsvektoren aus (6.45) unter der linearen Abbildung a sind

\[
a\left(\hat{\mathbf{e}}_{1}\right)=A \cdot \hat{\mathbf{e}}_{1}=\left[\begin{array}{cccc}A^{1}{ }_{1} & A^{1}{ }_{2} & \ldots & A^{1}{ }_{m}  \tag{6.47}\\ A^{2}{ }_{1} & A^{2}{ }_{2} & \ldots & A^{2}{ }_{m} \\ A^{3}{ }_{1} & A^{3}{ }_{2} & \ldots & A^{3}{ }_{m} \\ \vdots & \vdots & \vdots & \vdots \\ A^{n}{ }_{1} & A^{n}{ }_{2} & \ldots & A^{n}{ }_{m}\end{array}\right] \cdot\left[\begin{array}{c}1 \\ 0 \\ 0 \\ \vdots \\ 0\end{array}\right]=\left[\begin{array}{c}A^{1}{ }_{1} \\ A^{2}{ }_{1} \\ A^{3}{ }_{1} \\ \vdots \\ A^{n}{ }_{1}\end{array}\right]
\]

\[
a\left(\hat{\mathbf{e}}_{2}\right)=A \cdot \hat{\mathbf{e}}_{2}=\left[\begin{array}{cccc}A^{1}{ }_{1} & A^{1}{ }_{2} & \ldots & A^{1}{ }_{m}  \tag{6.48}\\ A^{2}{ }_{1} & A^{2}{ }_{2} & \ldots & A^{2}{ }_{m} \\ A^{3}{ }_{1} & A^{3}{ }_{2} & \ldots & A^{3}{ }_{m} \\ \vdots & \vdots & \vdots & \vdots \\ A^{n}{ }_{1} & A^{n}{ }_{2} & \ldots & A^{n}{ }_{m}\end{array}\right] \cdot\left[\begin{array}{c}0 \\ 1 \\ 0 \\ \vdots \\ 0\end{array}\right]=\left[\begin{array}{c}A^{1}{ }_{2} \\ A^{2}{ }_{2} \\ A^{3}{ }_{2} \\ \vdots \\ A^{n}{ }_{2}\end{array}\right]
\]


\end{document}