\documentclass[10pt]{article}
\usepackage[ngerman]{babel}
\usepackage[utf8]{inputenc}
\usepackage[T1]{fontenc}
\usepackage{amsmath}
\usepackage{amsfonts}
\usepackage{amssymb}
\usepackage[version=4]{mhchem}
\usepackage{stmaryrd}
\usepackage{bbold}

\begin{document}
\subsection*{2.5.4 Divergenz}
Wir betrachten die folgende Definition.

\section*{Definition 2.22 Divergenz}
Seien $n \in \mathbb{N}^{+}$und $\mathbf{v}: \mathbb{R}^{n} \rightarrow \mathbb{R}^{n}$ ein differentierbares Vektorfeld mit Komponenten

\[
\mathbf{v}\left(x^{1} ; \ldots ; x^{n}\right)=\left[\begin{array}{c}
v^{1}\left(x^{1} ; \ldots ; x^{n}\right)  \tag{2.149}\\
\vdots \\
v^{n}\left(x^{1} ; \ldots ; x^{n}\right)
\end{array}\right] .
\]

Die Divergenz von $\mathbf{v}$ ist


\begin{equation*}
\operatorname{div}(\mathbf{v}):=v^{1}{ }_{, 1}+v^{2}{ }_{, 2}+\ldots+v^{n}{ }_{, n} . \tag{2.150}
\end{equation*}


Beispiele:

\begin{itemize}
  \item Wir betrachten
\end{itemize}

\[
\mathbf{v}(x ; y):=\left[\begin{array}{l}
x \cdot y^{2}  \tag{2.151}\\
x^{3} \cdot y^{3}
\end{array}\right] .
\]

Die Divergenz von v ist


\begin{equation*}
\underline{\underline{\operatorname{div}(\mathbf{v})}}=v^{1}{ }_{, 1}+v_{, 2}^{2}=\left(x \cdot y^{2}\right)_{, x}+\left(x^{3} \cdot y^{3}\right)_{, y}=1 \cdot y^{2}+x^{3} \cdot 3 \cdot y^{2}=\underline{\underline{y^{2}} \cdot\left(1+3 x^{3}\right) .} \tag{2.152}
\end{equation*}


Bemerkungen:\\
i) Die Divergenz eines Vektorfeldes ist eine allgemeine Konstruktion in nD.\\
ii) Die Divergenz eines Vektorfeldes ist ein Skalarfeld.\\
iii) Die Divergenz eines Vektorfeldes ist ein Mass für dessen Quellendichte.\\
iv) Ein Vektorfeld $\mathbf{v}$ heisst quellenfrei, falls gilt


\begin{equation*}
\operatorname{div}(\mathbf{v})=0 . \tag{2.153}
\end{equation*}


Wir betrachten den folgenden Satz.\\
Satz 2.15 Elementare Rechenregeln für Divergenzen\\
Seien $n \in \mathbb{N}^{+}, \mathbf{v}, \mathbf{w}: \mathbb{R}^{n} \rightarrow \mathbb{R}^{n}$ differentierbare Vektorfelder, $f: \mathbb{R}^{n} \rightarrow \mathbb{R}$ eine differentierbare Funktion und $a, b \in \mathbb{R}$, dann gelten die folgenden Rechenregeln.\\
(a) Faktor-Regel:\\
(c) Linearität:

$$
\operatorname{div}(a \cdot \mathbf{v})=a \cdot \operatorname{div}(\mathbf{v})
$$

$$
\operatorname{div}(a \cdot \mathbf{v}+b \cdot \mathbf{w})=a \cdot \operatorname{div}(\mathbf{v})+b \cdot \operatorname{div}(\mathbf{w})
$$

(b) Summen-Regel:\\
(d) Produkt-Regel:

$$
\operatorname{div}(\mathbf{v}+\mathbf{w})=\operatorname{div}(\mathbf{v})+\operatorname{div}(\mathbf{w})
$$


\end{document}