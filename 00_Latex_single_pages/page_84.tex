\documentclass[10pt]{article}
\usepackage[ngerman]{babel}
\usepackage[utf8]{inputenc}
\usepackage[T1]{fontenc}
\usepackage{amsmath}
\usepackage{amsfonts}
\usepackage{amssymb}
\usepackage[version=4]{mhchem}
\usepackage{stmaryrd}
\usepackage{bbold}

\begin{document}
\subsection*{5.3 Quadratische Gleichungen}
Wir betrachten eine quadratische Gleichung mit reellen Koeffizienten $a, b, c \in \mathbb{R}$ mit $a \neq 0$ der Form


\begin{equation*}
a z^{2}+b z+c=0 \tag{5.20}
\end{equation*}


Dabei suchen wir alle komplexen Lösungen $z \in \mathbb{C}$ dieser Gleichung. Zunächst betrachten die Diskriminante


\begin{equation*}
D:=b^{2}-4 \cdot a \cdot c \tag{5.21}
\end{equation*}


Je nach Vorzeichen der Diskriminante ergeben sich unterschiedliche Lösungsmengen. Wir betrachten diese Fälle getrennt.

Fall 1: Fall $D>0$. In diesem Fall erhalten wir zwei reelle Lösungen


\begin{equation*}
z_{1,2}=\frac{-b \pm \sqrt{D}}{2 \cdot a} \tag{5.22}
\end{equation*}


Fall 2: Fall $D=0$. In diesem Fall erhalten wir eine reelle Lösung


\begin{equation*}
z=x_{\mathrm{s}}=-\frac{b}{2 \cdot a} \tag{5.23}
\end{equation*}


Fall 3: Fall $D<0$. In diesem Fall erhalten wir zwei komplexe Lösungen


\begin{equation*}
z_{1,2}=\frac{-b \pm \sqrt{D}}{2 \cdot a}=\frac{-b \pm \mathrm{i} \cdot \sqrt{|D|}}{2 \cdot a}=-\frac{b}{2 \cdot a} \pm \mathrm{i} \cdot \frac{\sqrt{|D|}}{2 \cdot a} \tag{5.24}
\end{equation*}


Bemerkungen:\\
i) Jede quadratische Gleichung mit reellen Koeffizienten hat demnach in $\mathbb{C}$ mindestens 1 Lösung und höchstens 2 Lösungen.\\
ii) $\operatorname{Im}$ Fall $D<0$ gilt


\begin{align*}
& \operatorname{Re}\left(z_{1}\right)=\operatorname{Re}\left(z_{2}\right)=x_{\mathrm{s}}=-\frac{b}{2 \cdot a}  \tag{5.25}\\
& \operatorname{Im}\left(z_{1}\right)=-\operatorname{Im}\left(z_{2}\right) \tag{5.26}
\end{align*}


Daraus folgt, dass die beiden Lösungen zueinander komplex konjugiert sind, d.h.


\begin{equation*}
z_{2}=z_{1}^{*} \tag{5.27}
\end{equation*}


iii) Quadratische Gleichungen mit nichtreellen Koeffizienten haben so gut wie keine Anwendungen in der Praxis.


\end{document}