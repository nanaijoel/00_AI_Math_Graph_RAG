\documentclass[10pt]{article}
\usepackage[ngerman]{babel}
\usepackage[utf8]{inputenc}
\usepackage[T1]{fontenc}
\usepackage{graphicx}
\usepackage[export]{adjustbox}
\graphicspath{ {./images/} }
\usepackage{amsmath}
\usepackage{amsfonts}
\usepackage{amssymb}
\usepackage[version=4]{mhchem}
\usepackage{stmaryrd}
\usepackage{bbold}

\begin{document}
vi) Ein Flussintegral über eine geschlossene Fläche heisst Perforation des Vektorfeldes durch die betreffende Fläche.\\
\includegraphics[max width=\textwidth, center]{2025_05_07_0e92860443a2101407cdg-1}

In diesem Fall verwendet man das Ring-Integralzeichen, d.h.


\begin{equation*}
\Phi=\oint_{M}\langle\mathbf{v}, \hat{\mathbf{n}}\rangle \mathrm{d} A . \tag{2.125}
\end{equation*}


vii) Der Fluss eines Vektorfeldes $\mathbf{v}$ durch eine Fläche ist das Integral des Anteils von v, welcher entlang ̂̂ und somit senkrecht zur Fläche zeigt. Der zu ̂̂ senkrechte bzw. zur Fläche parallele Anteil von v ist für den Fluss irrelevant.

Wir betrachten den folgenden Satz.\\
Satz 2.10 Fluss bei konstantem Anteil senkrecht zur Fläche\\
Seien $M$ eine parametrisierte Fläche mit Einheitsnormalen-Vektor n̂ und Flächeninhalt A sowie $\mathbf{v}: \mathbb{R}^{3} \rightarrow \mathbb{R}^{3}$ ein Vektorfeld. Gilt entlang der Fläche $\langle\mathbf{v}, \hat{\mathbf{n}}\rangle=: C \equiv$ konst., dann beträgt der Fluss von v durch die Fläche


\begin{equation*}
\Phi=C \cdot A . \tag{2.126}
\end{equation*}


Beweis: Für den Fluss von v durch die Fläche erhalten wir


\begin{equation*}
\underline{\underline{\Phi}}=\int_{M}\langle\mathbf{v}, \hat{\mathbf{n}}\rangle \mathrm{d} A=\int_{M} C \mathrm{~d} A=C \int_{M} 1 \mathrm{~d} A=\underline{\underline{C \cdot A}} \tag{2.127}
\end{equation*}


Damit haben wir den Satz bewiesen.\\
Wir betrachten den folgenden Satz.\\
Satz 2.11 Perforation eines homogenen Vektorfeldes\\
Die Perforation eines homogenen Vektorfeldes verschwindet.\\
Anwendungen:

\begin{itemize}
  \item Strömungsdynamik:
\end{itemize}


\begin{equation*}
\Phi_{\mathbf{v}}=\int_{M}\langle\mathbf{v}, \hat{\mathbf{n}}\rangle \mathrm{d} A \equiv \text { Volumen-Fluss des Mediums durch die Fläche }\left[\frac{\mathrm{m}^{3}}{\mathrm{~s}}\right] . \tag{2.128}
\end{equation*}


\begin{itemize}
  \item Elektrodynamik: Wir betrachten die folgenden Fälle getrennt.
\end{itemize}

Fall 1: $M$ ist eine geschlossene Fläche. Dann gilt


\begin{align*}
& \Phi_{\mathbf{E}}=\oint_{M}\langle\mathbf{E}, \hat{\mathbf{n}}\rangle \mathrm{d} A=\frac{1}{\varepsilon_{0}} \cdot Q_{\mathrm{eg}}  \tag{2.129}\\
& \Phi_{\mathbf{B}}=\oint_{M}\langle\mathbf{B}, \hat{\mathbf{n}}\rangle \mathrm{d} A=0 . \tag{2.130}
\end{align*}



\end{document}