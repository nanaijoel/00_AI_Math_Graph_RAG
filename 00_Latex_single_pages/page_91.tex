\documentclass[10pt]{article}
\usepackage[ngerman]{babel}
\usepackage[utf8]{inputenc}
\usepackage[T1]{fontenc}
\usepackage{amsmath}
\usepackage{amsfonts}
\usepackage{amssymb}
\usepackage[version=4]{mhchem}
\usepackage{stmaryrd}
\usepackage{bbold}

\begin{document}
ii) Addiert oder subtrahiert man zwei reelle Matrizen, dann addiert bzw. subtrahiert man ihre Komponenten.\\
iii) Die Addition bzw. Subtraktion von zwei reellen Matrizen unterschiedlicher Dimensionen ist nicht definiert.\\
iv) Beispiel-Codes zur Addition und Subtraktion von Matrizen mit gängiger Software.

\begin{center}
\begin{tabular}{|l|l|}
\hline
MATLAB/Octave & $\mathrm{M}=\mathrm{A}+\mathrm{B} ; \mathrm{M}=\mathrm{A}-\mathrm{B}$; \\
\hline
Mathematica/WolframAlpha & $\mathrm{M}=\mathrm{A}+\mathrm{B} ; \mathrm{M}=\mathrm{A}-\mathrm{B}$; \\
\hline
Python/Numpy & $\mathrm{M}=\mathrm{A}+\mathrm{B} ; \mathrm{M}=\mathrm{A}-\mathrm{B}$; \\
\hline
Python/Sympy & $\mathrm{M}=\mathrm{A}+\mathrm{B} ; \mathrm{M}=\mathrm{A}-\mathrm{B}$; \\
\hline
\end{tabular}
\end{center}

Beispiele:

\begin{itemize}
  \item $\left[\begin{array}{rr}2 & -1 \\ 7 & 5\end{array}\right]+\left[\begin{array}{ll}1 & -2 \\ 6 & -5\end{array}\right]=\left[\begin{array}{rr}3 & -3 \\ 13 & 0\end{array}\right]$
  \item $\left[\begin{array}{rr}2 & -1 \\ 7 & 5\end{array}\right]-\left[\begin{array}{ll}1 & -2 \\ 6 & -5\end{array}\right]=\left[\begin{array}{rr}1 & 1 \\ 1 & 10\end{array}\right]$
  \item $\left[\begin{array}{ll}0 & 1\end{array}\right]+\left[\begin{array}{ll}-1 & -1\end{array}\right]=\left[\begin{array}{ll}-1 & 0\end{array}\right]$
\end{itemize}

\subsection*{6.1.2.2 Multiplikation mit einem Skalar}
Jede reelle Matrix lässt sich mit einer reellen Zahl multiplizieren.\\
Definition 6.3 Multiplikation mit einem Skalar\\
Seien $m, n \in \mathbb{N}^{+}, a \in \mathbb{R}$ und $A \in \mathbb{M}(m, n, \mathbb{R})$, dann ist

\[
a \cdot A:=\left[\begin{array}{llll}
a \cdot A^{1}{ }_{1} & a \cdot A^{1}{ }_{2} & \ldots & a \cdot A^{1}{ }_{n}  \tag{6.6}\\
a \cdot A^{2}{ }_{1} & a \cdot A^{2}{ }_{2} & \ldots & a \cdot A^{2}{ }_{n} \\
\vdots & \vdots & \vdots & \vdots \\
a \cdot A^{m}{ }_{1} & a \cdot A^{m}{ }_{2} & \ldots & a \cdot A^{m}{ }_{n}
\end{array}\right] .
\]

Bemerkungen:\\
i) Für alle $a \in \mathbb{R}$ und $A \in \mathbb{M}(m, n, \mathbb{R})$ gilt $a \cdot A \in \mathbb{M}(m, n, \mathbb{R})$.\\
ii) Multipliziert man eine reelle Matrix mit einem Skalar, dann multipliziert man ihre Komponenten mit dem Skalar.\\
iii) Es soll keine Rolle spielen, ob der Skalar links oder rechts der reellen Matrix geschrieben wird. Man definiert


\begin{equation*}
A \cdot a:=a \cdot A . \tag{6.7}
\end{equation*}


iv) Aus der Multiplikation mit einem Skalar ergibt sich auf natürliche Weise eine Division. Es soll gelten


\begin{equation*}
\frac{A}{a}:=\frac{1}{a} \cdot A . \tag{6.8}
\end{equation*}



\end{document}