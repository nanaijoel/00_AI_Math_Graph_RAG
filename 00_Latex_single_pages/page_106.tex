\documentclass[10pt]{article}
\usepackage[ngerman]{babel}
\usepackage[utf8]{inputenc}
\usepackage[T1]{fontenc}
\usepackage{amsmath}
\usepackage{amsfonts}
\usepackage{amssymb}
\usepackage[version=4]{mhchem}
\usepackage{stmaryrd}
\usepackage{bbold}

\def\AA{\mathring{\mathrm{A}}}

\begin{document}
\begin{itemize}
  \item Spiegelung an der $y$-Achse: $S_{y}:=\left[\begin{array}{rr}-1 & 0 \\ 0 & 1\end{array}\right]$
  \item Spiegelung an der Geraden $y=x: S_{x y}:=\left[\begin{array}{ll}0 & 1 \\ 1 & 0\end{array}\right]$
  \item Drehung um den Ursprung um $\pi / 2: R(\pi / 2)=\circ:=\left[\begin{array}{rr}0 & -1 \\ 1 & 0\end{array}\right]$
  \item Drehung um den Ursprung um $-\pi / 2: R(-\pi / 2)=-\AA:=\left[\begin{array}{rr}0 & 1 \\ -1 & 0\end{array}\right]$
  \item Drehung um den Ursprung um $\varphi: R(\varphi)=\left[\begin{array}{rr}\cos (\varphi) & -\sin (\varphi) \\ \sin (\varphi) & \cos (\varphi)\end{array}\right]$
\end{itemize}

Die Matrix i̊, welche die Drehung um den Ursprung um $\pi / 2$ beschreibt, heisst symplektische Matrix. Sie hat analoge algebraische Eigenschaften wie die imaginäre Einheit in den komplexen Zahlen. Insbesondere gilt


\begin{equation*}
\AA^{2}=R^{2}(\pi / 2)=R(\pi / 2) \cdot R(\pi / 2)=R(\pi)=P=-\mathbb{1} . \tag{6.51}
\end{equation*}



\end{document}