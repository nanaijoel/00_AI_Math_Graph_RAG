\documentclass[10pt]{article}
\usepackage[ngerman]{babel}
\usepackage[utf8]{inputenc}
\usepackage[T1]{fontenc}
\usepackage{amsmath}
\usepackage{amsfonts}
\usepackage{amssymb}
\usepackage[version=4]{mhchem}
\usepackage{stmaryrd}
\usepackage{bbold}
\usepackage{graphicx}
\usepackage[export]{adjustbox}
\graphicspath{ {./images/} }

\begin{document}
\subsection*{6.3.3 Spezielle orthogonale Matrizen}
\subsection*{6.3.3.1 Spiegelungen in nD}
Wir betrachten $n \in \mathbb{N}^{+}$und die Spiegelung in $\mathbb{R}^{n}$ an einem ( $n-1$ )-dimensionalen Unterraum, welcher senkrecht steht zur Richtung $\hat{\mathbf{n}} \in \mathbb{R}^{n}$. Die Situation ist in der folgenden Skizze dargestellt.\\
\includegraphics[max width=\textwidth, center]{2025_05_07_381794b021f120bc7337g-1}

Die Spiegelungsmatrix lässt sich durch eine einfache Formel aus $\hat{\mathbf{n}}$ berechnen.

\section*{Satz 6.10 Householder-Formel}
Die Spiegelung wird beschrieben durch die Matrix


\begin{equation*}
S(\hat{\mathbf{n}})=\mathbb{1}-2 \cdot \hat{\mathbf{n}} \cdot \hat{\mathbf{n}}^{T} . \tag{6.74}
\end{equation*}


Beweis: Um das Bild eines Vektors $\mathbf{v} \in \mathbb{R}^{n}$ unter der Spiegelung zu berechnen, zerlegen wir $\mathbf{v}$ in seine Anteile parallel und senkrecht zu $\hat{\mathbf{n}}$. Es sei also


\begin{equation*}
\mathbf{v}=\mathbf{v}_{\|}+\mathbf{v}_{\perp} \tag{6.75}
\end{equation*}


mit


\begin{equation*}
\mathbf{v}_{\|}=\langle\mathbf{v}, \hat{\mathbf{n}}\rangle \cdot \hat{\mathbf{n}} \quad \text { und } \quad \mathbf{v}_{\perp}=\mathbf{v}-\mathbf{v}_{\|} . \tag{6.76}
\end{equation*}


Für die Spiegelungsmatrix $S(\hat{\mathbf{n}})$ gilt gemäss Skizze


\begin{align*}
\underline{\underline{S(\hat{\mathbf{n}}) \cdot \mathbf{v}}} & =\mathbf{v}-2 \cdot \mathbf{v}_{\|}=\mathbf{v}-2 \cdot\langle\mathbf{v}, \hat{\mathbf{n}}\rangle \cdot \hat{\mathbf{n}}=\mathbf{v}-2 \cdot \hat{\mathbf{n}} \cdot\langle\mathbf{v}, \hat{\mathbf{n}}\rangle=\mathbf{v}-2 \cdot \hat{\mathbf{n}} \cdot\langle\hat{\mathbf{n}}, \mathbf{v}\rangle \\
& =\mathbf{v}-2 \cdot \hat{\mathbf{n}} \cdot\left(\hat{\mathbf{n}}^{T} \cdot \mathbf{v}\right)=\mathbb{1} \cdot \mathbf{v}-2 \cdot\left(\hat{\mathbf{n}} \cdot \hat{\mathbf{n}}^{T}\right) \cdot \mathbf{v}=\left(\mathbb{1}-2 \cdot \hat{\mathbf{n}} \cdot \hat{\mathbf{n}}^{T}\right) \cdot \mathbf{v} . \tag{6.77}
\end{align*}


Dies impliziert (6.74) und wir haben den Satz bewiesen.

\section*{Bemerkungen:}
i) Es gilt $S(\hat{\mathbf{n}}) \in \mathrm{O}(n)$.\\
ii) In jedem Fall gilt


\begin{equation*}
S(-\hat{\mathbf{n}})=S(\hat{\mathbf{n}}) \tag{6.78}
\end{equation*}


d.h. es spielt keine Rolle auf welche Seite man $\hat{\mathbf{n}}$ wählt.


\end{document}