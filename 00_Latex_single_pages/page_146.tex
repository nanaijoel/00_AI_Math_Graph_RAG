\documentclass[10pt]{article}
\usepackage[ngerman]{babel}
\usepackage[utf8]{inputenc}
\usepackage[T1]{fontenc}
\usepackage{amsmath}
\usepackage{amsfonts}
\usepackage{amssymb}
\usepackage[version=4]{mhchem}
\usepackage{stmaryrd}
\usepackage{bbold}

\begin{document}
Bemerkungen:\\
i) Für alle Skalar-Produkte und alle $\mathbf{v}, \mathbf{w} \in V$ gilt die wichtige Konvention


\begin{equation*}
\mathbf{v} \perp \mathbf{w} \Leftrightarrow\langle\mathbf{v}, \mathbf{w}\rangle=0 \tag{7.91}
\end{equation*}


ii) Für positiv definite Skalar-Produkte kann die Formel (7.90) in jeden Fall angewendet werden, denn aus der Cauchy-Schwarz- Ungleichung folgt für alle $\mathbf{v}, \mathbf{w} \in V \backslash\{0\}$


\begin{equation*}
\frac{\langle\mathbf{v}, \mathbf{w}\rangle}{|\mathbf{v}| \cdot|\mathbf{w}|} \in[-1,1] . \tag{7.92}
\end{equation*}


iii) Für nicht positiv definite Skalar-Produkte lässt sich im allgemeinen keine universell gültige Formel für den Winkel zwischen zwei Vektoren finden.

\subsection*{7.3.2 Metrik}
Wir betrachten die folgende Definition.

\section*{Definition 7.15 Metrik}
Seien $(V, \mathbb{R},+, \cdot)$ ein reeller Vektorraum mit endlicher Dimension $n \in \mathbb{N}^{+}$und Skalar-Produkt $\langle.,$.$\rangle und B=\left\{\mathbf{e}_{1}, \ldots, \mathbf{e}_{n}\right\} \subset V$ eine Basis von $V$. Die Metrik $g \in \mathbb{M}(n, n, \mathbb{R})$ ist die GramMatrix der Basis-Vektoren in $B$, d.h.

\[
g=G\left(\mathbf{e}_{1} ; \ldots ; \mathbf{e}_{n}\right):=\left[\begin{array}{ccc}
\left\langle\mathbf{e}_{1}, \mathbf{e}_{1}\right\rangle & \ldots & \left\langle\mathbf{e}_{1}, \mathbf{e}_{n}\right\rangle  \tag{7.93}\\
\vdots & \vdots & \vdots \\
\left\langle\mathbf{e}_{n}, \mathbf{e}_{1}\right\rangle & \ldots & \left\langle\mathbf{e}_{n}, \mathbf{e}_{n}\right\rangle
\end{array}\right]
\]

Bemerkungen:\\
i) Die Komponenten der Metrik werden auch metrische Koeffizienten genannt. Für alle $i, j \in$ $\{1, \ldots, n\}$ gilt


\begin{equation*}
g_{i j}=\left\langle\mathbf{e}_{i}, \mathbf{e}_{j}\right\rangle \tag{7.94}
\end{equation*}


ii) Das Symbol $g$ wird sowohl für die Metrik als auch für deren Determinante verwendet. Welche Bedeutung gerade gemeint ist, muss aus dem Kontext abgelesen werden.\\
iii) Die Metrik ist in jedem Fall regulär, weil die Basis-Vektoren nach Definition linear unabhängig sein müssen.\\
iv) Die Metrik ist genau dann diagonal, wenn die Basis-Vektoren paarweise senkrecht aufeinander stehen.\\
v) In $\mathbb{R}^{n}$ mit Gram-Riemann-Skalar-Produkt haben die Standard-Einheitsvektoren $\left\{\hat{\mathbf{e}}_{1}, \ldots, \hat{\mathbf{e}}_{n}\right\}$ die Metrik\\
$\hat{g}=\mathbb{1} \quad$ bzw. $\quad g_{i j}=\delta_{i j}$.


\end{document}