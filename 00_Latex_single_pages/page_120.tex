\documentclass[10pt]{article}
\usepackage[ngerman]{babel}
\usepackage[utf8]{inputenc}
\usepackage[T1]{fontenc}
\usepackage{amsmath}
\usepackage{amsfonts}
\usepackage{amssymb}
\usepackage[version=4]{mhchem}
\usepackage{stmaryrd}
\usepackage{bbold}

\begin{document}
Die Determinante einer $3 \times 3$-Matrix ist folglich


\begin{align*}
\operatorname{det}(A)= & A_{1}^{1} \cdot A_{2}^{2} \cdot A_{3}^{3}+A_{1}^{2} \cdot A_{2}^{3} \cdot A_{3}^{1}+A_{1}^{3} \cdot A^{1}{ }_{2} \cdot A_{3}^{2} \\
& -A_{1}^{3} \cdot A_{2}^{2} \cdot A_{3}^{1}-A_{1}^{1} \cdot A^{3}{ }_{2} \cdot A_{3}^{2}-A_{1}^{2} \cdot A_{2}^{1} \cdot A_{3}^{3} \tag{6.112}
\end{align*}


iv) Offensichtlich gilt in allen Dimensionen


\begin{equation*}
\operatorname{det}(0)=0 \quad \text { und } \quad \operatorname{det}(\mathbb{1})=1 \tag{6.113}
\end{equation*}


v) Die Determinante einer diagonalen Matrix ist gerade das Produkt ihrer Eigenwerte. Es gilt also\\
$\operatorname{det}(D)=\lambda_{1} \cdot \lambda_{2} \cdot \ldots \cdot \lambda_{n}$.\\
vi) Die Determinante einer linken unteren oder rechten oberen Dreiecksmatrix reduziert sich auf das Produkt ihrer Diagonalenelemente. Es gilt also


\begin{align*}
& \operatorname{det}(L)=\operatorname{det}\left(\left[\begin{array}{ccccc}
L_{1}^{1} & 0 & 0 & \ldots & 0 \\
L_{1}^{2} & L^{2}{ }_{2} & 0 & \ldots & 0 \\
\vdots & \vdots & \vdots & \ldots & \vdots \\
L_{1}{ }_{1} & L^{n}{ }_{2} & L^{n}{ }_{3} & \ldots & L^{n}{ }_{n}
\end{array}\right]\right)=L_{1}^{1} \cdot L_{2}^{2} \cdot \ldots \cdot L_{n}^{n}  \tag{6.115}\\
& \operatorname{det}(R)=\operatorname{det}\left(\left[\begin{array}{ccccc}
R_{1}^{1} & R_{2}^{1} & R_{3}^{1} & \ldots & R_{n}^{1} \\
0 & R_{2}^{2} & R_{3}^{2} & \ldots & R_{n}^{2} \\
\vdots & \ldots & \vdots & \vdots & \vdots \\
0 & \ldots & 0 & 0 & R_{n}^{n}
\end{array}\right]\right)=R_{1}^{1} \cdot R_{2}^{2} \cdot \ldots \cdot R_{n}^{n} \tag{6.116}
\end{align*}


vii) Beispiel-Codes zur Berechnung der Determinante mit gängiger Software.

\begin{center}
\begin{tabular}{|l|l|}
\hline
MATLAB/Octave & $\mathrm{d}=\operatorname{det}(\mathrm{A})$ \\
\hline
Mathematica/WolframAlpha & $\mathrm{d}=$ Det [A] \\
\hline
Python/Numpy & \begin{tabular}{l}
import numpy as np; \\
d=np.linalg. $\operatorname{det}(\mathrm{A})$ \\
\end{tabular} \\
\hline
Python/Sympy & \begin{tabular}{l}
import sympy as sp; \\
$\mathrm{d}=\mathrm{sp} . \operatorname{det}(\mathrm{A})$ \\
\end{tabular} \\
\hline
\end{tabular}
\end{center}

Beispiele:

\begin{itemize}
  \item $\operatorname{det}\left(\left[\begin{array}{ll}1 & 2 \\ 3 & 4\end{array}\right]\right)=1 \cdot 4-3 \cdot 2=4-6=2$
  \item $\operatorname{det}\left(\left[\begin{array}{ll}2 & 4 \\ 3 & 6\end{array}\right]\right)=2 \cdot 6-3 \cdot 4=12-12=0$
  \item $\operatorname{det}\left(\left[\begin{array}{ll}2 & 0 \\ 0 & 3\end{array}\right]\right)=2 \cdot 3-0 \cdot 0=6-0=6$
\end{itemize}

\end{document}