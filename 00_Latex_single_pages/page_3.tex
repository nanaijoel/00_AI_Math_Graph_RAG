\documentclass[10pt]{article}
\usepackage[ngerman]{babel}
\usepackage[utf8]{inputenc}
\usepackage[T1]{fontenc}
\usepackage{graphicx}
\usepackage[export]{adjustbox}
\graphicspath{ {./images/} }
\usepackage{amsmath}
\usepackage{amsfonts}
\usepackage{amssymb}
\usepackage[version=4]{mhchem}
\usepackage{stmaryrd}
\usepackage{bbold}

\begin{document}
\subsection*{1.2 Archimedes-Cauchy-Riemann-Approximationsprozess}
\subsection*{1.2.1 Einleitung}
Einleitende Bemerkungen:\\
i) Die Abkürzung ACR-Prozess steht für Archimedes-Cauchy-Riemann- Approximationsprozess.\\
ii) Sinn und Zweck eines ACR-Prozess ist nicht die Berechnung von Integralen sondern das Auffinden von Integralen in der Praxis. Das Konzept ist im folgenden Schema dargestellt.\\
\includegraphics[max width=\textwidth, center]{2025_05_07_63370f1e1f15f16530dag-1}\\
iii) Aus den historischen Wurzeln des ACR-Prozesses ergibt sich auch die Erklärung für die Schreibweise mit dem "Integral-Haken".

\subsection*{1.2.2 Historische Methode zur Berechnung von Integralen}
Wir betrachten den Graphen einer stetigen Funktion $f: \mathbb{R} \rightarrow \mathbb{R}$ entlang eines reellen Intervalls $\left[x_{0}, x_{\mathrm{E}}\right]$. Die Situation ist im folgenden $x$ - $y$-Diagramm dargestellt.\\
\includegraphics[max width=\textwidth, center]{2025_05_07_63370f1e1f15f16530dag-1(1)}

Schon bevor die Newton-Leibniz-Formel bekannt war, gab es Methoden zur näherungsweise Berechnung der Fläche $A$ entlang des reellen Intervalls $\left[x_{0}, x_{\mathrm{E}}\right]$ zwischen der $x$-Achse und dem Graphen von $f$. Dabei wurde typischerweise nach den folgenden Schritten vorgegangen.

S1 Man wählt ein $N \in \mathbb{N}^{+}$und definiert sich die Werte


\begin{align*}
\delta x & :=\frac{x_{\mathrm{E}}-x_{0}}{N}  \tag{1.4}\\
x_{k} & :=x_{0}+k \cdot \delta x \quad \text { für } k \in\{1, \ldots, N\} . \tag{1.5}
\end{align*}


S2 Man berechnet für $k \in\{1, \ldots, N\}$ die Flächeninhalte der Rechtecke gemäss


\begin{equation*}
\delta A_{k}:=f\left(x_{k}\right) \cdot \delta x . \tag{1.6}
\end{equation*}



\end{document}