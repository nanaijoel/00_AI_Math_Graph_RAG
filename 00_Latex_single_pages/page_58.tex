\documentclass[10pt]{article}
\usepackage[ngerman]{babel}
\usepackage[utf8]{inputenc}
\usepackage[T1]{fontenc}
\usepackage{amsmath}
\usepackage{amsfonts}
\usepackage{amssymb}
\usepackage[version=4]{mhchem}
\usepackage{stmaryrd}
\usepackage{bbold}
\usepackage{graphicx}
\usepackage[export]{adjustbox}
\graphicspath{ {./images/} }

\begin{document}
S2 Für jede gefundene kritische Stelle $\left(x_{k} ; y_{k}\right)$ sind zu berechnen


\begin{align*}
z_{k} & =f\left(x_{k} ; y_{k}\right)  \tag{2.212}\\
H_{k} & =\nabla^{2} f\left(x_{k} ; y_{k}\right) . \tag{2.213}
\end{align*}


S3 Ergebnistabelle:

\begin{center}
\begin{tabular}{|c|c|c|c|c|c|c|c|l|}
\hline
$k$ & $x_{k}$ & $y_{k}$ & $z_{k}$ & $H_{k 11}$ & $H_{k 22}$ & $H_{k 12}$ & $\operatorname{det}\left(H_{k}\right)$ & Typ: \\
\hline\hline
1 & $x_{1}$ & $y_{1}$ & $z_{1}$ & $\ldots$ & $\ldots$ & $\ldots$ & $\ldots$ & $\ldots$ \\
$\vdots$ & $\vdots$ & $\vdots$ & $\vdots$ &  &  &  &  &  \\
$n$ & $x_{n}$ & $y_{n}$ & $z_{n}$ & $\ldots$ & $\ldots$ & $\ldots$ & $\ldots$ & $\ldots$ \\
\hline
\end{tabular}
\end{center}

\subsection*{2.7.4 Globale Extrema}
Wir betrachten ein Gebiet $G \subset \mathbb{R}^{2}$ und eine Funktion $f: \mathbb{R}^{2} \rightarrow \mathbb{R}$. Die Situation ist in der folgenden Skizze dargestellt.\\
\includegraphics[max width=\textwidth, center]{2025_05_07_c8fb93f1767599a79b2dg-1}

Um die globalen Extrema einer Funktion in 2D zu bestimmen, kann nach den folgenden Schritten vorgegangen werden.

S1 Inneres: Kritische Stellen von $f$ im Innern von $G$ bestimmen als Lösungen des Gleichungssystems


\begin{equation*}
\boldsymbol{\nabla} f(x ; y)=0 . \tag{2.215}
\end{equation*}


S2 Randstücke: Um die Kandidaten auf den Randstücken zu finden, kann nach den folgenden Schritten vorgegangen werden.

S 2.1: Die Randstücke $\Gamma_{k}$ als parametrisierte Kurven beschreiben.\\
S 2.2: Die Funkionen $u_{k}(\tau):=f(x(\tau) ; y(\tau))$ für $\tau \in\left[\tau_{0}, \tau_{\mathrm{E}}\right]$ berechnen.\\
S 2.3: Die Lösungen der Gleichungen $0=\dot{u}_{k}(\tau)$ sind Kandidaten für globale Extrema.\\
S3 Eckpunkte: Die Eckpunkte des Gebietes $G$ sind in jedem Fall Kandiaten für globale Extrema.


\end{document}