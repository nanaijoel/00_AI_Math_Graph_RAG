\documentclass[10pt]{article}
\usepackage[ngerman]{babel}
\usepackage[utf8]{inputenc}
\usepackage[T1]{fontenc}
\usepackage{amsmath}
\usepackage{amsfonts}
\usepackage{amssymb}
\usepackage[version=4]{mhchem}
\usepackage{stmaryrd}

\begin{document}
Bemerkungen:\\
i) Der Stokes-Integralsatz besagt die Gleichheit eines Linienintegrals mit einem Flussintegral. Es gilt


\begin{equation*}
\text { Zirkulation von } \mathbf{v} \equiv \text { Summe aller eingeschlossenen Wirbel von } \mathbf{v} \text {. } \tag{2.173}
\end{equation*}


ii) Der Stokes-Integralsatz kann allgemein in 2D und 3D formuliert werden.\\
iii) Der Stokes-Integralsatz ist eine Verallgemeinerung der Newton-Leibniz-Formel in 2D bzw. 3D.\\
iv) Der Stokes-Integralsatz lässt sich bezogen auf beide Seiten der Gleichung sinnvoll anwenden.\\
v) Der Stokes-Integralsatz etabliert die Interpretation der Rotation als Wirbeldichte eines Vektorfeldes.\\
vi) Gemäss Stokes-Integralsatz verschwindet offenbar die Zirkulation jedes wirbelfreien Vektorfeldes entlang einer beliebigen Randkurve $\partial G$. Aus $\operatorname{rot}(\mathbf{v})=0$ folgt


\begin{equation*}
\underline{\underline{\Upsilon_{\mathbf{v}}}}=\oint_{\partial G}\langle\mathbf{v}, \hat{\mathbf{e}}\rangle \mathrm{d} s=\int_{G}\langle\operatorname{rot}(\mathbf{v}), \hat{\mathbf{n}}\rangle \mathrm{d} A=\int_{G}\langle 0, \hat{\mathbf{n}}\rangle \mathrm{d} A=\int_{G} 0 \mathrm{~d} A=\underline{\underline{0}} \tag{2.174}
\end{equation*}


Anwendungen:

\begin{itemize}
  \item Elektrodynamik: Die Maxwell-Gleichungen für die Rotation des E-Feldes und B-Feldes lauten
\end{itemize}


\begin{align*}
\operatorname{rot}(\mathbf{E}) & =-\dot{\mathbf{B}}  \tag{2.175}\\
\operatorname{rot}(\mathbf{B}) & =\varepsilon_{0} \cdot \mu_{0} \cdot \dot{\mathbf{E}}+\mu_{0} \cdot \mathbf{J} . \tag{2.176}
\end{align*}


In einer statischen Situation, d.h. für $\dot{\mathbf{E}}=\dot{\mathbf{B}}=0$ vereinfachen sich diese Gleichungen zu


\begin{align*}
& \operatorname{rot}(\mathbf{E})=0  \tag{2.177}\\
& \operatorname{rot}(\mathbf{B})=\mu_{0} \cdot \mathbf{J} . \tag{2.178}
\end{align*}


Für die Zirkulation des E-Feldes und B-Feldes entlang einer beliebigen Randkurve $\partial G$ folgt aus dem Stoкes-Integralsatz


\begin{align*}
\underline{\underline{\Upsilon_{\mathbf{E}}}} & =\oint_{\partial G}\langle\mathbf{E}, \hat{\mathbf{e}}\rangle \mathrm{d} s=\int_{G}\langle\operatorname{rot}(\mathbf{E}), \hat{\mathbf{n}}\rangle \mathrm{d} A=\int_{G}\langle 0, \hat{\mathbf{n}}\rangle \mathrm{d} A=\int_{G} 0 \mathrm{~d} A=\underline{\underline{0}}  \tag{2.179}\\
\underline{\underline{\Upsilon_{\mathbf{B}}}} & =\oint_{\partial G}\langle\mathbf{B}, \hat{\mathbf{e}}\rangle \mathrm{d} s=\int_{G}\langle\operatorname{rot}(\mathbf{B}), \hat{\mathbf{n}}\rangle \mathrm{d} A=\int_{G}\left\langle\mu_{0} \cdot \mathbf{J}, \hat{\mathbf{n}}\right\rangle \mathrm{d} A=\mu_{0} \int_{G}\langle\mathbf{J}, \hat{\mathbf{n}}\rangle \mathrm{d} A \\
& =\underline{\underline{\mu_{0} \cdot I_{\mathrm{eg}}} .} \tag{2.180}
\end{align*}


\begin{itemize}
  \item Erhaltungssätze
  \item Flächen-Berechnungen
  \item Geometrische Analysis
\end{itemize}

\end{document}