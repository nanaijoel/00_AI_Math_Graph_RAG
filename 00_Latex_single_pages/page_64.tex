\documentclass[10pt]{article}
\usepackage[ngerman]{babel}
\usepackage[utf8]{inputenc}
\usepackage[T1]{fontenc}
\usepackage{amsmath}
\usepackage{amsfonts}
\usepackage{amssymb}
\usepackage[version=4]{mhchem}
\usepackage{stmaryrd}
\usepackage{bbold}

\begin{document}
\section*{Satz 3.2 Beschleunigungsarbeit}
Um einen Körper der Masse $m$ von der Anfangsgeschwindigkeit $v_{0}$ auf die Endgeschwindigkeit $v_{\mathrm{E}}$ zu beschleunigen, muss am Körper eine Beschleunigungsarbeit geleistet werden von


\begin{equation*}
\Delta W=\frac{m}{2} \cdot v_{\mathrm{E}}^{2}-\frac{m}{2} \cdot v_{0}^{2} . \tag{3.18}
\end{equation*}


Beweis: Wir zeigen mehrere Varianten um die Beschleunigungsarbeit zu berechnen.\\
Variante 1: Mit Hilfe des Arbeitsintegrals und mehrfacher Substitution erhalten wir


\begin{align*}
\underline{\underline{\Delta W}} & =\int_{s_{0}}^{s_{\mathrm{E}}} F(s) \mathrm{d} s=\int_{t_{0}}^{t_{\mathrm{E}}} F(s(t)) \cdot \dot{s}(t) \mathrm{d} t=\int_{t_{0}}^{t_{\mathrm{E}}} m \cdot a(t) \cdot v(t) \mathrm{d} t=m \int_{t_{0}}^{t_{\mathrm{E}}} v(t) \cdot \dot{v}(t) \mathrm{d} t \\
& =m \int_{v_{0}}^{v_{\mathrm{E}}} v \mathrm{~d} v=\left.m \cdot \frac{1}{2} \cdot\left[v^{2}\right]\right|_{v_{0}} ^{v_{\mathrm{E}}}=\frac{m}{2} \cdot\left(v_{\mathrm{E}}^{2}-v_{0}^{2}\right)=\underline{\underline{\frac{m}{2}} \cdot v_{\mathrm{E}}^{2}-\frac{m}{2} \cdot v_{0}^{2} .} \tag{3.19}
\end{align*}


Variante 2: Wir verwenden einen Archimedes-Cauchy-Riemann-Approximationsprozess. Dabei gehen wir nach folgenden Schritten vor.

S1 Lokal: Wir betrachten ein kleines Wegstück $\delta$ s. Die Beschleunigungsarbeit entlang $\delta s$ ist


\begin{equation*}
\underline{\delta W} \approx F(s) \cdot \delta s \approx m \cdot a(t) \cdot \frac{\delta s}{\delta t} \cdot \delta t \approx m \cdot \frac{\delta v}{\delta t} \cdot v \cdot \delta t=\underline{m \cdot v} \cdot \delta v . \tag{3.20}
\end{equation*}


S2 Global: Durch Integration über $v$ erhalten wir die Beschleunigungsarbeit


\begin{equation*}
\underline{\underline{\Delta W}}=m \int_{v_{0}}^{v_{\mathrm{E}}} v \mathrm{~d} v=\left.m \cdot \frac{1}{2} \cdot\left[v^{2}\right]\right|_{v_{0}} ^{v_{\mathrm{E}}}=\frac{m}{2} \cdot\left(v_{\mathrm{E}}^{2}-v_{0}^{2}\right)=\underline{\underline{\frac{m}{2}} \cdot v_{\mathrm{E}}^{2}-\frac{m}{2} \cdot v_{0}^{2}} . \tag{3.21}
\end{equation*}


Damit haben wir den Satz bewiesen.

\subsection*{3.1.2 Partielle Integration}
Wir betrachten den folgenden Satz.

\section*{Satz 3.3 Partielle Integration}
Seien $g, h: \mathbb{R} \rightarrow \mathbb{R}$ differentierbare und integrierbare Funktionen und $x_{0}, x_{\mathrm{E}} \in \mathbb{R}$ mit $x_{0}<x_{\mathrm{E}}$, dann gilt folgendes.\\
(a) $\begin{gathered}\downarrow \\ \int g^{\downarrow}(x) \cdot h^{\prime}(x) \mathrm{d} x=g(x) \cdot h(x)-\int g^{\prime}(x) \cdot h(x) \mathrm{d} x\end{gathered}$

(b) $\quad$\begin{tabular}{c}
$x_{x_{0}}^{x_{\mathrm{E}}} \stackrel{\downarrow}{g(x)} \cdot h^{\prime}(x) \mathrm{d} x=\left.[g(x) \cdot h(x)]\right|_{x_{0}} ^{x_{\mathrm{E}}}-\int_{x_{0}}^{x_{\mathrm{E}}} g^{\prime}(x) \cdot h(x) \mathrm{d} x$ \\
\hline
\end{tabular}

Beweis: Übung.


\end{document}