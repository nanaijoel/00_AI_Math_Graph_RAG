\documentclass[10pt]{article}
\usepackage[ngerman]{babel}
\usepackage[utf8]{inputenc}
\usepackage[T1]{fontenc}
\usepackage{amsmath}
\usepackage{amsfonts}
\usepackage{amssymb}
\usepackage[version=4]{mhchem}
\usepackage{stmaryrd}
\usepackage{bbold}

\begin{document}
\begin{itemize}
  \item $\operatorname{det}\left(\left[\begin{array}{rrr}3 & 0 & 0 \\ 0 & -2 & 0 \\ 0 & 0 & 4\end{array}\right]\right)=3 \cdot(-2) \cdot 4=-24$
  \item $\operatorname{det}\left(\left[\begin{array}{rrr}3 & 0 & 0 \\ 7 & -2 & 0 \\ 5 & -8 & 4\end{array}\right]\right)=3 \cdot(-2) \cdot 4=-24$
\end{itemize}

\subsection*{6.4.3.2 Eigenschaften}
Für die Determinante einer Matrix gelten einige einfache Rechenregeln.\\
Satz 6.15 Rechenregeln der Determinante\\
Seien $n \in \mathbb{N}^{+}, A, B \in \mathbb{M}(n, n, \mathbb{R})$ und $a \in \mathbb{R}$. Dann gelten folgende Rechenregeln.\\
(a) $\operatorname{det}\left(A^{T}\right)=\operatorname{det}(A)$\\
(c) $\operatorname{det}(A \cdot B)=\operatorname{det}(A) \cdot \operatorname{det}(B)$\\
(b) $\operatorname{det}(a \cdot A)=a^{n} \cdot \operatorname{det}(A)$\\
(d) $\operatorname{det}\left(A^{-1}\right)=\frac{1}{\operatorname{det}(A)}$ falls $A$ regulär

Beweis: Die Aussagen (a) und (b) sind klar, während (c) nur mit grossem Aufwand gezeigt werden kann. Falls $A$ regulär ist, dann hat sie eine Inverse $A^{-1} \in \mathbb{M}(n, n, \mathbb{R})$ und es gilt


\begin{align*}
A^{-1} \cdot A & =\mathbb{1} & & \operatorname{det}(\ldots)  \tag{6.117}\\
\Rightarrow & \operatorname{det}\left(A^{-1} \cdot A\right) & =\operatorname{det}(\mathbb{1}) &  \tag{6.118}\\
\Rightarrow & \operatorname{det}\left(A^{-1}\right) \cdot \operatorname{det}(A) & =1 & \mid: \operatorname{det}(A) . \tag{6.119}
\end{align*}


Daraus folgt $\operatorname{det}(A) \neq 0$ und nach der Division erhalten wir


\begin{equation*}
\underline{\underline{\operatorname{det}\left(A^{-1}\right)}=\frac{1}{\operatorname{det}(A)}} . \tag{6.120}
\end{equation*}


Damit haben wir die Aussage (d) bewiesen.\\
Bemerkungen:\\
i) Für die Determinante gibt es keine allgemeingültige Summen-Regel. Es gibt quadratische Matrizen $A$ und $B$, für welche gilt $\operatorname{det}(A+B)=\operatorname{det}(A)+\operatorname{det}(B)$ aber auch solche für die wir $\operatorname{det}(A+B) \neq \operatorname{det}(A)+\operatorname{det}(B)$ finden.\\
ii) Die Rechenregel (c) aus Satz 6.15 ist eine äusserst wichtige Eigenschaft der Determinante. Sie hat unter anderem zur Folge, dass gilt\\
$\operatorname{det}(B \cdot A)=\operatorname{det}(B) \cdot \operatorname{det}(A)=\operatorname{det}(A) \cdot \operatorname{det}(B)=\operatorname{det}(A \cdot B)$.\\
Das heisst, man darf innerhalb einer Determinante die Faktoren eines Matrix-Produkts vertauschen. Im Gegensatz zur Situation bei der Spur gilt dies auch bei mehr als zwei Faktoren für beliebige Änderungen der Reihenfolge der Faktoren. Für drei Faktoren erhalten wir


\begin{align*}
\operatorname{det}(A \cdot B \cdot C) & =\operatorname{det}(C \cdot A \cdot B)=\operatorname{det}(B \cdot C \cdot A)=\operatorname{det}(A \cdot C \cdot B)=\operatorname{det}(B \cdot A \cdot C) \\
& =\operatorname{det}(C \cdot B \cdot A)=\operatorname{det}(A) \cdot \operatorname{det}(B) \cdot \operatorname{det}(C) . \tag{6.122}
\end{align*}



\end{document}