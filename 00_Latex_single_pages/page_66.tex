\documentclass[10pt]{article}
\usepackage[ngerman]{babel}
\usepackage[utf8]{inputenc}
\usepackage[T1]{fontenc}
\usepackage{amsmath}
\usepackage{amsfonts}
\usepackage{amssymb}
\usepackage[version=4]{mhchem}
\usepackage{stmaryrd}

\begin{document}
v) Es ist üblich und sinnvoll, vor Anwendung der partiellen Integration die Rollen der Faktoren im Integranden mit Pfeilen zu markieren.\\
vi) Beim Aufleiten von $h^{\prime} \mapsto h$ darf eine beliebige Stammfunktion $h$ von $h^{\prime}$ verwendet werden. Es ist jedoch zwingend nötig, dass auf der rechten Seite in beiden Termen die gleiche Stammfunktion $h$ eingesetzt wird!\\
vii) Allgemein gilt

\[
\int f(x) \mathrm{d} x=\int \begin{gather*}
\downarrow  \tag{3.28}\\
f(x) \cdot \stackrel{\uparrow}{1} \mathrm{~d} x=f(x) \cdot x-\int f^{\prime}(x) \cdot x \mathrm{~d} x .
\end{gather*}
\]

Mit Hilfe dieses 1er-Tricks lassen sich u.a. die Stammfunktionen finden von\\
$f \in\left\{\ln , \log _{a}\right.$, arcsin, , arccos, arctan, arccot, arsinh, arcosh, artanh, arcoth $\}$.\\
Anwendungen:

\begin{itemize}
  \item Berechnung von Integralen
  \item Taylor-Entwicklungen
  \item Differentialgleichungen
  \item Variationsrechnung
  \item Fourier- und Laplace-Transformation
  \item FEM-Simulationen
\end{itemize}

\end{document}