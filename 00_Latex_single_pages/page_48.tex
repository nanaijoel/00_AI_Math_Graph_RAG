\documentclass[10pt]{article}
\usepackage[ngerman]{babel}
\usepackage[utf8]{inputenc}
\usepackage[T1]{fontenc}
\usepackage{amsmath}
\usepackage{amsfonts}
\usepackage{amssymb}
\usepackage[version=4]{mhchem}
\usepackage{stmaryrd}
\usepackage{bbold}
\usepackage{graphicx}
\usepackage[export]{adjustbox}
\graphicspath{ {./images/} }

\begin{document}
Anwendungen:

\begin{itemize}
  \item Strömungsdynamik: Aus der Inkompressibilität einer Flüssigkeit folgt die Quellenfreiheit des Geschwindigkeitsvektorfeldes v und somit das Verschwinden der Perforation von v durch eine beliebige Oberfäche $\partial V$.
  \item Elektrodynamik: Die Maxwell-Gleichungen für die Divergenz des E-Feldes und B-Feldes lauten
\end{itemize}


\begin{equation*}
\operatorname{div}(\mathbf{E})=\frac{1}{\varepsilon_{0}} \cdot \rho \tag{2.168}
\end{equation*}


$\operatorname{div}(\mathbf{B})=0$.\\
Für die Perforation des E-Feldes und B-Feldes durch eine beliebige Oberfläche $\partial V$ folgt aus dem Gauss-Integralsatz


\begin{align*}
& \underline{\underline{\Phi_{\mathbf{E}}}}=\oint_{\partial V}\langle\mathbf{E}, \hat{\mathbf{n}}\rangle \mathrm{d} A=\int_{V} \operatorname{div}(\mathbf{E}) \mathrm{d} V=\int_{V} \frac{1}{\varepsilon_{0}} \cdot \rho \mathrm{~d} V=\frac{1}{\varepsilon_{0}} \int_{V} \rho \mathrm{~d} V=\underline{\underline{\frac{1}{\varepsilon_{0}}} \cdot Q_{\mathrm{eg}}}  \tag{2.170}\\
& \underline{\underline{\Phi_{\mathbf{B}}}}=\oint_{\partial V}\langle\mathbf{B}, \hat{\mathbf{n}}\rangle \mathrm{d} A=\int_{V} \operatorname{div}(\mathbf{B}) \mathrm{d} V=\int_{V} 0 \mathrm{~d} V=\underline{\underline{0 .}} \tag{2.171}
\end{align*}


\begin{itemize}
  \item Erhaltungssätze
  \item Volumen-Berechnungen
  \item Geometrische Analysis
\end{itemize}

\subsection*{2.6.2 Stokes-Integralsatz}
Wir betrachten ein Gebiet $G \subset \mathbb{R}^{3}$ mit Randkurve $\partial G$, welche das Einheitsnormalen-Vektorfeld $\hat{\mathbf{n}}$ rechts umläuft im Bereich eines Vektorfeldes $\mathbf{v}: \mathbb{R}^{3} \rightarrow \mathbb{R}^{3}$. Die Situation ist in der folgenden Skizze dargestellt.\\
\includegraphics[max width=\textwidth, center]{2025_05_07_fe97686687f5b2916ce8g-1}

Wir betrachten den folgenden Satz.\\
Satz 2.20 Stokes-Integralsatz in 3D\\
Seien $G \subset \mathbb{R}^{3}$ ein Gebiet mit Randkurve $\partial G$, welche das Einheitsnormalen-Vektorfeld $\hat{\mathbf{n}}$ rechts umläuft im Bereich eines Vektorfeldes $\mathbf{v}: \mathbb{R}^{3} \rightarrow \mathbb{R}^{3}$, dann gilt


\begin{equation*}
\oint_{\partial G}\langle\mathbf{v}, \hat{\mathbf{e}}\rangle \mathrm{d} s=\Upsilon_{\mathbf{v}}=\int_{G}\langle\operatorname{rot}(\mathbf{v}), \hat{\mathbf{n}}\rangle \mathrm{d} A \tag{2.172}
\end{equation*}



\end{document}