\documentclass[10pt]{article}
\usepackage[ngerman]{babel}
\usepackage[utf8]{inputenc}
\usepackage[T1]{fontenc}
\usepackage{amsmath}
\usepackage{amsfonts}
\usepackage{amssymb}
\usepackage[version=4]{mhchem}
\usepackage{stmaryrd}
\usepackage{graphicx}
\usepackage[export]{adjustbox}
\graphicspath{ {./images/} }
\usepackage{bbold}

\begin{document}
iii) Werden die Definitionsmenge $A$ und die Zielmenge $B$ als Teilmengen eines geometrischen Raumes interpretiert, dann wird $\mathbf{v}$ als Vektorfeld auf $A$ bezeichnet. In diesem Fall wird an jedem Punkt $\left(x_{1} ; x_{2} ; \ldots ; x_{n}\right) \in A$ ein Vektor $\mathbf{v}\left(x_{1} ; x_{2} ; \ldots ; x_{n}\right)$ "angehängt".\\
\includegraphics[max width=\textwidth, center]{2025_05_07_d7c5040c5fd9570e2707g-1}\\
iv) Im allgemeinen können die reellen Variablen $x_{1}, x_{2}, \ldots, x_{n} \in \mathbb{R}$ und der Funktionswert $\mathbf{v}\left(x_{1} ; x_{2} ; \ldots ; x_{n}\right)$ bliebige und auch unterschiedliche Masseinheiten tragen.\\
v) In der Physik werden oft zeitabhängige Vektorfelder verwendet, welche zusätzlich von der Zeit als unabhängige Variable abhängen.

Anwendungen:

\begin{itemize}
  \item Strömungsdynamik: Die zentrale Grösse ist das Geschwindigkeitsvektorfeld\\
$\mathbf{v}(t ; x ; y ; z) \equiv$ Geschwindigkeit des Mediums zur Zeit $t$ am Ort $(x ; y ; z)$.
  \item Elektrodynamik: Um Dichte und Bewegung von elektrischen Ladungen und die Wechselwirkung mit elektrischen und magnetischen Feldern zu beschreiben, werden mehrere im allgemeinen zeitabhängige Skalarfelder und zeitabhängige Vektorfelder verwendet.
\end{itemize}


\begin{align*}
\rho(t ; x ; y ; z) & : \equiv \text { Elektrische Ladungsdichte }  \tag{2.22}\\
\mathbf{v}(t ; x ; y ; z) & : \equiv \text { Geschwindigkeitsfeld der elektrischen Ladungsträger }  \tag{2.23}\\
\mathbf{J}(t ; x ; y ; z) & :=\rho(t ; x ; y ; z) \cdot \mathbf{v}(t ; x ; y ; z) \equiv \text { Stromdichte-Vektorfeld }  \tag{2.24}\\
\mathbf{E}(t ; x ; y ; z) & : \equiv \text { Elektrisches Feld }  \tag{2.25}\\
\mathbf{B}(t ; x ; y ; z) & : \equiv \text { Magnetisches Feld } \equiv \text { Magnetische Flussdichte (veraltet) }  \tag{2.26}\\
\mathbf{P}(t ; x ; y ; z) & : \equiv \text { Polarisation }  \tag{2.27}\\
\mathbf{M}(t ; x ; y ; z) & : \equiv \text { Magnetisierung }  \tag{2.28}\\
\mathbf{D}(t ; x ; y ; z) & : \equiv \text { Dielektrische Verschiebung }  \tag{2.29}\\
\mathbf{H}(t ; x ; y ; z) & : \equiv \text { Magnetisches Hilfsfeld } \equiv \text { Magnetfeld (veraltet) }  \tag{2.30}\\
\mathbf{S}(t ; x ; y ; z) & : \equiv \text { PoYNTING- Vektorfeld }  \tag{2.31}\\
\phi(t ; x ; y ; z) & : \equiv \text { Skalarpotential }  \tag{2.32}\\
\mathbf{A}(t ; x ; y ; z) & : \equiv \text { Vektorpotential } \tag{2.33}
\end{align*}


\begin{itemize}
  \item Differentialgleichungen ersten Grades: Der Verlauf der verschiedenen Lösungen einer ODE ersten Grades kann durch das Richtungsvektorfeld $\hat{\mathbf{v}}(x ; y)$ beschrieben und visualisiert werden.
\end{itemize}

\end{document}