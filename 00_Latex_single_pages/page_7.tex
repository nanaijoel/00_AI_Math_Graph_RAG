\documentclass[10pt]{article}
\usepackage[ngerman]{babel}
\usepackage[utf8]{inputenc}
\usepackage[T1]{fontenc}
\usepackage{amsmath}
\usepackage{amsfonts}
\usepackage{amssymb}
\usepackage[version=4]{mhchem}
\usepackage{stmaryrd}
\usepackage{graphicx}
\usepackage[export]{adjustbox}
\graphicspath{ {./images/} }

\begin{document}
\subsection*{1.2.4.4 Volumen}
Wir betrachten einen Körper mit veränderlichem Querschnitt $A(x)$. Die Situation ist in der folgenden Sizze dargestellt.\\
\includegraphics[max width=\textwidth, center]{2025_05_07_15853b3735c17f857807g-1}

Um das Volumen des Körpers zu berechnen, verwenden wir einen Archimedes-Cauchy-Riemann-Approximationsprozess. Dabei gehen wir nach folgenden Schritten vor.

S1 Lokal: Wir betrachten eine kleine Teilstück des Körpers an der Position $x$ mit Querschnittfläche $A(x)$ und Länge $\delta x$. Das Volumen des Teilstücks beträgt


\begin{equation*}
\underline{\delta V} \approx \underline{A(x) \cdot \delta x} . \tag{1.18}
\end{equation*}


S2 Global: Durch Integration über $x$ können wir das gesamte Volumen des Körpers berechnen. Wir erhalten


\begin{equation*}
\underline{\underline{V}}=\int_{x_{0}}^{x_{\mathrm{E}}} A(x) \mathrm{d} x=\underline{\underline{\ldots}} \tag{1.19}
\end{equation*}



\end{document}