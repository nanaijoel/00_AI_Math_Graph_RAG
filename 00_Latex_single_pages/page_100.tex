\documentclass[10pt]{article}
\usepackage[ngerman]{babel}
\usepackage[utf8]{inputenc}
\usepackage[T1]{fontenc}
\usepackage{amsmath}
\usepackage{amsfonts}
\usepackage{amssymb}
\usepackage[version=4]{mhchem}
\usepackage{stmaryrd}
\usepackage{bbold}

\begin{document}
\begin{itemize}
  \item $\left[\begin{array}{rr}2 & 5 \\ -3 & -8\end{array}\right]^{-1}=\left[\begin{array}{rr}8 & 5 \\ -3 & -2\end{array}\right]$
\end{itemize}

Für invertierbare Matrizen gelten einige Rechenregeln.\\
Satz 6.2 Elementare Rechenregeln der Inversion\\
Es seien $A, B$ invertierbare Matrizen und $a \in \mathbb{R} \backslash\{0\}$. Dann gelten folgende Rechenregeln.\\
(a) $(a \cdot A)^{-1}=\frac{1}{a} \cdot A^{-1}$\\
(c) $\left(A^{T}\right)^{-1}=\left(A^{-1}\right)^{T}$\\
(b) $(A \cdot B)^{-1}=B^{-1} \cdot A^{-1}$\\
(d) $\left(A^{-1}\right)^{-1}=A$

\subsection*{6.1.3.6 Diagonale Matrix}
Mit quadratischen Matrizen, die nur auf der Hauptdiagonalen von Null verschiedene Komponenten haben, lässt sich besonders einfach rechnen.

Definition 6.12 Diagonal-Matrix\\
Sei $n \in \mathbb{N}^{+}$und $\lambda_{1}, \ldots, \lambda_{n} \in \mathbb{R}$. Eine Matrix $D \in \mathbb{M}(n, n, \mathbb{R})$ der Form

\[
D=\left[\begin{array}{cccc}
\lambda_{1} & 0 & \cdots & 0  \tag{6.29}\\
0 & \lambda_{2} & \ddots & \vdots \\
\vdots & \ddots & \ddots & 0 \\
0 & \cdots & 0 & \lambda_{n}
\end{array}\right]
\]

heisst diagonal.\\
Bemerkungen:\\
i) Die reellen Zahlen $\lambda_{1}, \ldots, \lambda_{n}$ auf der Hauptdiagonalen heissen Eigenwerte der Matrix D.\\
ii) Die Einheitsmatrix sowie jede quadratische Nullmatrix sind offensichtlich diagonal.\\
iii) Jede diagonale Matrix ist offensichtlich symmetrisch.\\
iv) Zwei beliebige diagonalen Matrizen kommutieren, d.h. es gilt


\begin{equation*}
[D, \tilde{D}]=0 . \tag{6.30}
\end{equation*}


Aber: Diagonale Matrizen kommutieren nicht mit allen Matrizen!\\
v) Sind alle Eigenwerte einer diagonalen Matrix von Null verschieden, dann ist die Matrix invertierbar. Die Inverse ist ebenfalls diagonal und es gilt

\[
D^{-1}=\left[\begin{array}{rrcr}
\frac{1}{\lambda_{1}} & 0 & \ldots & 0  \tag{6.31}\\
0 & \frac{1}{\lambda_{2}} & \ldots & 0 \\
\vdots & \vdots & \ddots & \vdots \\
0 & 0 & \ldots & \frac{1}{\lambda_{n}}
\end{array}\right] .
\]


\end{document}