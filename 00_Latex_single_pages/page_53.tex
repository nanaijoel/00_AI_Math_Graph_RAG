\documentclass[10pt]{article}
\usepackage[ngerman]{babel}
\usepackage[utf8]{inputenc}
\usepackage[T1]{fontenc}
\usepackage{amsmath}
\usepackage{amsfonts}
\usepackage{amssymb}
\usepackage[version=4]{mhchem}
\usepackage{stmaryrd}
\usepackage{bbold}
\usepackage{graphicx}
\usepackage[export]{adjustbox}
\graphicspath{ {./images/} }

\begin{document}
Wir betrachten $\Delta x, \Delta y \in \mathbb{R}$, so dass gilt\\
$\hat{\mathbf{e}}=\left[\begin{array}{l}\Delta x \\ \Delta y\end{array}\right]$.\\
Ferner seien $P:=(x ; y ; z)$ der Punkt oberhalb des Punktes $\left(x_{0}+\Delta x ; y_{0}+\Delta y\right)$ auf der Tangentialebene $E_{0}$ an den Graphen von $f$ im Punkt $P_{0}$ und\\
$\mathbf{w}:=\mathbf{P}-\mathbf{P}_{0}=\left[\begin{array}{c}x-x_{0} \\ y-y_{0} \\ z-f\left(x_{0} ; y_{0}\right)\end{array}\right]=\left[\begin{array}{c}x-x_{0} \\ y-y_{0} \\ z-z_{0}\end{array}\right]=\left[\begin{array}{c}\Delta x \\ \Delta y \\ \Delta z\end{array}\right]$.\\
Weil der Vektor w die Punkte $P$ und $P_{0}$ auf der Tangentialebene $E_{0}$ verbindet, liegt er selbst auf der Tangentialeben $E_{0}$ und es muss gelten

\[
0=\langle\mathbf{w}, \mathbf{n}\rangle=\left\langle\left[\begin{array}{c}\Delta x  \tag{2.195}\\ \Delta y \\ \Delta z\end{array}\right],\left[\begin{array}{c}-f_{, x} \\ -f_{, y} \\ 1\end{array}\right]\right\rangle=-\Delta x \cdot f_{, x}-\Delta y \cdot f_{, y}+\Delta z \cdot 1
\]

$=\Delta z-\left(\Delta x \cdot f_{, x}+\Delta y \cdot f_{, y}\right)=\Delta z-\left\langle\left[\begin{array}{c}\Delta x \\ \Delta y\end{array}\right],\left[\begin{array}{l}f_{, x} \\ f_{, y}\end{array}\right]\right\rangle=\Delta z-\langle\hat{\mathbf{e}}, \boldsymbol{\nabla} f\rangle$.\\
Daraus folgt


\begin{equation*}
\Delta z=\langle\hat{\mathbf{e}}, \boldsymbol{\nabla} f\rangle \tag{2.196}
\end{equation*}


und für die gesuchte Steigung erhalten wir\\
$\underline{\underline{m}}=\frac{\Delta z}{\Delta s}=\frac{\Delta z}{|\hat{\mathbf{e}}|}=\frac{\langle\hat{\mathbf{e}}, \boldsymbol{\nabla} f\rangle}{1}=\underline{\underline{\langle\hat{\mathbf{e}}, \nabla} \boldsymbol{\nabla}\rangle}$.\\
Damit haben wir den Satz bewiesen.\\
Wir betrachten die folgende Definition.\\
Definition 2.26 Richtungsableitung in nD\\
Seien $n \in \mathbb{N}^{+}, f: \mathbb{R}^{n} \rightarrow \mathbb{R}$ differentierbar und $\hat{\mathbf{e}} \in \mathbb{R}^{n}$ ein Einheitsvektor, dann ist die Richtungsableitung von $f$ in Richtung ê die reelle Zahl


\begin{equation*}
\nabla_{\hat{\mathbf{e}}} f:=\langle\hat{\mathbf{e}}, \boldsymbol{\nabla} f\rangle . \tag{2.198}
\end{equation*}


Wir betrachten die Situation in der $x$ - $y$-Ebene gemäss folgende Skizze.\\
\includegraphics[max width=\textwidth, center]{2025_05_07_d0bba98847a137d0005ag-1}


\end{document}