\documentclass[10pt]{article}
\usepackage[ngerman]{babel}
\usepackage[utf8]{inputenc}
\usepackage[T1]{fontenc}
\usepackage{amsmath}
\usepackage{amsfonts}
\usepackage{amssymb}
\usepackage[version=4]{mhchem}
\usepackage{stmaryrd}
\usepackage{bbold}

\begin{document}
Bemerkungen:\\
i) Für $z=0$ exisitiert in beiden Varianten kein eindeutiges Argument. Meistens wählt man $\arg (0)=0$.\\
ii) Für fast alle Anwendungen kann sowohl die Basler-Variante als auch die Zürcher-Variante verwendet werden.\\
iii) Die Formeln zur Berechnung von $\arg (z)$ aus Real- und Imaginärteil sind bei der ZürcherVariante etwas einfacher.\\
iv) Beispiel-Codes zur Berechnung des Argumentes mit gängiger Software.

\begin{center}
\begin{tabular}{|l|l|}
\hline
MATLAB/Octave & phi=angle (z) \\
\hline
Mathematica/WolframAlpha & phi=Arg [z] \\
\hline
Python/Numpy & import numpy as np; phi=np.angle (z) \\
\hline
Python/Sympy & import sympy as sp; phi=sp. $\arg (z)$ \\
\hline
\end{tabular}
\end{center}

Wir betrachten den folgenden Satz.\\
Satz 5.3 Trigonometrische Form\\
Seien $x, y \in \mathbb{R}, r \in \mathbb{R}^{+}$und $\varphi \in[0,2 \pi[$ oder $\varphi \in]-\pi, \pi]$ sowie


\begin{equation*}
\operatorname{cis}(\varphi):=\cos (\varphi)+\mathrm{i} \cdot \sin (\varphi) \tag{5.17}
\end{equation*}


dann gibt es ein eindeutiges $z \in \mathbb{C} \backslash\{0\}$ mit


\begin{equation*}
z=x+y \cdot \mathrm{i}=r \cdot \operatorname{cis}(\varphi) . \tag{5.18}
\end{equation*}


Ferner gelten die folgenden Umrechnungsformeln.\\
(a) $\quad x=r \cdot \cos (\varphi) \wedge y=r \cdot \sin (\varphi)$\\
(b) $r=|z|=\sqrt{x^{2}+y^{2}} \wedge \varphi=\arg (z)$

Beweis: Die Aussage folgt sofort durch Anwenden der Trigonometrie in der Gauss-Ebene.\\
Bemerkungen:\\
i) Jede komplexe Zahl $z \in \mathbb{C}$ lässt sich auf zwei Arten darstellen, nämlich


\begin{equation*}
z=\underbrace{x+y \cdot \mathrm{i}}_{\text {arithmetische Form }}=\underbrace{r \cdot \operatorname{cis}(\varphi)}_{\text {trigonometrische Form }} \tag{5.19}
\end{equation*}


ii) Man bezeichnet die Schreibweise $z=r \cdot \operatorname{cis}(\varphi)$ nur dann als trigonometrische Form, wenn gilt $r=|z| \geq 0$. Der Winkel $\varphi \in \mathbb{R}$ darf jedoch beliebig gewählt werden, d.h. $\varphi$ muss nicht unbedingt das Argument gemäss einer der beiden Varianten sein.


\end{document}