\documentclass[10pt]{article}
\usepackage[ngerman]{babel}
\usepackage[utf8]{inputenc}
\usepackage[T1]{fontenc}
\usepackage{amsmath}
\usepackage{amsfonts}
\usepackage{amssymb}
\usepackage[version=4]{mhchem}
\usepackage{stmaryrd}
\usepackage{bbold}
\usepackage{graphicx}
\usepackage[export]{adjustbox}
\graphicspath{ {./images/} }

\begin{document}
Wir betrachten den folgenden Satz.\\
Satz 2.25 Eigenschaften des Gradienten in 2D\\
Seien $f: \mathbb{R}^{2} \rightarrow \mathbb{R}$ differentierbar, $\hat{\mathbf{e}} \in \mathbb{R}^{2}$ ein Einheitsvektor und $\alpha=\measuredangle(\boldsymbol{\nabla} f ; \hat{\mathbf{e}})$, dann gilt folgendes.\\
(a) $\nabla_{\hat{\mathrm{e}}} f \in[-|\boldsymbol{\nabla} f|,+|\boldsymbol{\nabla} f|]$\\
(b) $\alpha=0 \Leftrightarrow \hat{\mathbf{e}} \|+\boldsymbol{\nabla} f \Leftrightarrow \nabla_{\hat{\mathbf{e}}} f=+|\boldsymbol{\nabla} f|$ maximal\\
(c) $\alpha=\pi \Leftrightarrow \hat{\mathbf{e}} \|-\nabla f \Leftrightarrow \nabla_{\hat{\mathbf{e}}} f=-|\boldsymbol{\nabla} f|$ minimal\\
(d) $\alpha=\frac{\pi}{2} \Leftrightarrow \hat{\mathbf{e}} \perp \boldsymbol{\nabla} f \Leftrightarrow \nabla_{\hat{\mathrm{e}}} f=0$\\
(e) $\boldsymbol{\nabla} f=0 \Leftrightarrow$ Tangentialebene an den Graphen von $f$ verläuft horizontal

Beweis: Für die Richtungsableitung von $f$ in Richtung ê erhalten wir


\begin{equation*}
\underline{\nabla_{\hat{\mathbf{e}}} f}=\langle\hat{\mathbf{e}}, \boldsymbol{\nabla} f\rangle=|\hat{\mathbf{e}}| \cdot|\nabla f| \cdot \cos (\alpha)=1 \cdot|\nabla f| \cdot \cos (\alpha)=\underline{\cos (\alpha) \cdot|\nabla f| .} \tag{2.199}
\end{equation*}


Daraus folgen unmittelbar alle Aussagen und wir haben den Satz bewiesen.\\
Bemerkungen:\\
i) Der Gradient $\boldsymbol{\nabla} f$ zeigt an jedem Punkt der $x-y$-Ebene gerade in die Richtung, in welcher die Steigung des Graphen von $f$ maximal ist.\\
ii) Die Länge $|\nabla f|$ des Gradienten ist an jedem Punkt der $x$ - $y$-Ebene gerade die maximale Steigung des Graphen von $f$.\\
iii) Der Gradient steht an jedem Punkt der $x$ - $y$-Ebene senkrecht auf der Level-Linie von $f$ durch diesen Punkt.\\
iv) Dort wo der Gradient verschwindet verläuft die Tangentialebene an den Graphen von $f$ horizontal. Entsprechend hat $f$ dort einen Hoch-Punkt, Tief-Punkt oder Sattel-Punkt.\\
v) Definition und Eigenschaften sowohl des Gradienten als auch der Richtungsableitung gelten ganz analog auch in nD.

Zeichnet man den Gradienten und die Level-Linien von $f$ in der $x$ - $y$-Ebene ein, dann ergibt sich ein Bild wie im folgenden Beispiel.\\
\includegraphics[max width=\textwidth, center]{2025_05_07_daf365c289cbfac44a30g-1}


\end{document}