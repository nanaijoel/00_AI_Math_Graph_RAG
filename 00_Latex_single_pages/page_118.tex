\documentclass[10pt]{article}
\usepackage[ngerman]{babel}
\usepackage[utf8]{inputenc}
\usepackage[T1]{fontenc}
\usepackage{amsmath}
\usepackage{amsfonts}
\usepackage{amssymb}
\usepackage[version=4]{mhchem}
\usepackage{stmaryrd}
\usepackage{bbold}

\begin{document}
\begin{itemize}
  \item $S_{2}=\left\{\left[\begin{array}{ll}1 & 2 \\ 1 & 2\end{array}\right],\left[\begin{array}{ll}1 & 2 \\ 2 & 1\end{array}\right]\right\}$
  \item $S_{3}=\left\{\left[\begin{array}{lll}1 & 2 & 3 \\ 1 & 2 & 3\end{array}\right],\left[\begin{array}{lll}1 & 2 & 3 \\ 2 & 3 & 1\end{array}\right],\left[\begin{array}{lll}1 & 2 & 3 \\ 3 & 1 & 2\end{array}\right],\left[\begin{array}{lll}1 & 2 & 3 \\ 3 & 2 & 1\end{array}\right],\left[\begin{array}{lll}1 & 2 & 3 \\ 1 & 3 & 2\end{array}\right],\left[\begin{array}{lll}1 & 2 & 3 \\ 2 & 1 & 3\end{array}\right]\right\}$
\end{itemize}

Es stellt sich natürlich die Frage, wie man die Kardinalitäten der symmetrischen Gruppen $S_{n}$ möglichst einfach aus $n \in \mathbb{N}^{+}$berechnen kann.

Satz 6.14 Kardinalität der symmetrischen Gruppen\\
Sei $n \in \mathbb{N}^{+}$. Es gilt


\begin{equation*}
\# S_{n}=n!=1 \cdot 2 \cdot \ldots \cdot n \tag{6.102}
\end{equation*}


Beweis: Um ein $p \in S_{n}$ zu definieren, wählen wir für jedes $k \in\{1, \ldots, n\}$ einen Funktionswert $p(k) \in\{1, \ldots, n\}$ aus. Weil $p$ bijektiv ist, müssen wir dabei jedes Element von $\{1, \ldots, n\}$ genau einmal verwenden. Für $p(1)$ haben wir also $n$ Werte zur Auswahl, für $p(2)$ dann nur noch $(n-1)$, für $p(3)$ nur noch $(n-2)$ und immer so weiter, bis am Schluss für $p(n)$ nur noch ein einziger Wert übrig ist. Durch Kombination erhalten wir insgesamt die Anzahl Möglichkeiten von


\begin{equation*}
N=n \cdot(n-1) \cdot(n-2) \cdot \ldots \cdot 2 \cdot 1=n!. \tag{6.103}
\end{equation*}


Weil jede dieser $N=n$ ! Möglichkeiten genau eine der Permutationen aus $S_{n}$ definiert, haben wir damit den Satz bewiesen.

Manche Eigenschaften einer Permutation lassen sich am Wert von zwei Zahlen ablesen. Dazu machen wir folgende Definitionen.

Definition 6.20 Inversionszahl \& Vorzeichen\\
Seien $n \in \mathbb{N}^{+}$und $p \in S_{n}$.\\
(a) Die Inversionszahl von $p$ ist


\begin{equation*}
\rho(p):=\#\left\{(i ; j) \in\{1, \ldots, n\}^{2} \mid i<j \wedge p(i)>p(j)\right\} . \tag{6.104}
\end{equation*}


(b) Das Vorzeichen von $p$ ist


\begin{equation*}
\sigma(p):=(-1)^{\rho(p)} . \tag{6.105}
\end{equation*}


Bemerkungen:\\
i) Die Inversionszahl einer Permutation ist also die Anzahl der Pärchen aus Zahlen in $\{1, \ldots, n\}$ für welche die Permutation die Reihenfolge umdreht.\\
ii) Die Inversionszahl wird auch Fehlstandszahl genannt und für das Vorzeichen wird auch das lateinische Wort Signum verwendet.\\
iii) In jedem Fall gilt


\begin{equation*}
\rho(p) \in\left\{0,1,2, \ldots, \frac{n \cdot(n-1)}{2}\right\} . \tag{6.106}
\end{equation*}



\end{document}