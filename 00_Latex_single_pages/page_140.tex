\documentclass[10pt]{article}
\usepackage[ngerman]{babel}
\usepackage[utf8]{inputenc}
\usepackage[T1]{fontenc}
\usepackage{amsmath}
\usepackage{amsfonts}
\usepackage{amssymb}
\usepackage[version=4]{mhchem}
\usepackage{stmaryrd}
\usepackage{bbold}

\begin{document}
\subsection*{7.2.3 Bild \& Kern}
Für eine lineare Abbildung lassen sich zwei charakteristische Mengen definieren.

\section*{Definition 7.9 Bild \& Kern}
Seien $(V, \mathbb{K},+, \cdot)$ und ( $W, \mathbb{K},+, \cdot)$ zwei Vektorräume über dem gleichen Zahlenkörper $\mathbb{K}$ und $a: V \rightarrow W$ eine lineare Abbildung.\\
(a) Das Bild von $a$ ist die Menge


\begin{equation*}
\operatorname{img}(a):=a(V):=\{\mathbf{w} \in W \mid \text { Es gibt ein } \mathbf{v} \in V \text { mit } a(\mathbf{v})=\mathbf{w} .\} . \tag{7.43}
\end{equation*}


(b) Der Kern von $a$ ist die Menge


\begin{equation*}
\operatorname{ker}(a):=\{\mathbf{v} \in V \mid a(\mathbf{v})=0\} . \tag{7.44}
\end{equation*}


Bemerkungen:\\
i) Der Begriff des Bildes in der linearen Algebra stimmt überein mit dem entsprechenden Begriff aus der allgemeinen Theorie der Funktionen.\\
ii) Der Kern einer linearen Abbildung besteht gerade aus jenen Vektoren, die auf 0 abgebildet werden.\\
iii) Um Verwechslungen vorzubeugen, sei nochmals betont, dass gilt


\begin{equation*}
\operatorname{img}(a) \subseteq W \quad \text { aber } \quad \operatorname{ker}(a) \subseteq V \tag{7.45}
\end{equation*}


iv) In jedem Fall gilt $0 \in \operatorname{ker}(a)$.\\
v) Gilt $\operatorname{ker}(a)=\{0\}$, dann sagt man, $a$ hat einen trivialen Kern.

Charakteristisch für lineare Abbildungen ist der folgende Satz.\\
Satz 7.7 Dimensionssatz\\
Seien $(V, \mathbb{K},+, \cdot)$ und ( $W, \mathbb{K},+, \cdot)$ zwei Vektorräume über dem gleichen Zahlenkörper $\mathbb{K}$ und $a: V \rightarrow W$ eine lineare Abbildung. Dann gilt $\operatorname{img}(a) \leq W$ und $\operatorname{ker}(a) \leq V$ sowie


\begin{equation*}
\operatorname{dim}(\operatorname{img}(a))+\operatorname{dim}(\operatorname{ker}(a))=\operatorname{dim}(V) . \tag{7.46}
\end{equation*}


Bemerkungen:\\
i) Bild und Kern sind demnach nicht nur Teilmengen sondern Unterräume von $W$ bzw. $V$.\\
ii) Der Dimensionssatz ist eine Art "Erhaltungssatz" für Dimensionen unter der Wirkung einer linearen Abbildung. Die Dimension von $V$ wird aufgeteilt auf Bild und Kern.

Satz 7.8 Umkehrbarkeitssatz\\
Seien $(V, \mathbb{K},+, \cdot)$ ein Vektorraum und $a: V \rightarrow V$ eine lineare Abbildung. Dann gilt


\begin{equation*}
\text { a bijektiv } \Leftrightarrow \operatorname{ker}(a)=\{0\} . \tag{7.47}
\end{equation*}



\end{document}