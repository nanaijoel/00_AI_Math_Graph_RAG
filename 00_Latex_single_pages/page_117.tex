\documentclass[10pt]{article}
\usepackage[ngerman]{babel}
\usepackage[utf8]{inputenc}
\usepackage[T1]{fontenc}
\usepackage{amsmath}
\usepackage{amsfonts}
\usepackage{amssymb}
\usepackage[version=4]{mhchem}
\usepackage{stmaryrd}
\usepackage{bbold}

\begin{document}
Ferner berechnen wir\\
$\underline{\underline{\operatorname{tr}(B \cdot A)}}=\sum_{s=1}^{n} \sum_{r=1}^{n} B_{r}^{s} \cdot A^{r}{ }_{s}=\sum_{s=1}^{n} \sum_{r=1}^{n} A_{s}^{r} \cdot B_{r}^{s}=\sum_{r=1}^{n} \sum_{s=1}^{n} A_{s}^{r} \cdot B_{r}^{s}=\underline{\underline{\operatorname{tr}(A \cdot B)}}$.\\
Damit haben wir alle Aussagen und den Satz bewiesen.\\
Bemerkungen:\\
i) Die Spur kann selbst als lineare Abbildung des $\operatorname{Typs} \operatorname{tr}: \mathbb{M}(n, n, \mathbb{R}) \rightarrow \mathbb{R}$ aufgefasst werden.\\
ii) Bei der Anwendung der Regel (d) aus Satz 6.13 auf Produkte von mehr als zwei Matrizen ist Vorsicht geboten. Für drei Matrizen gilt


\begin{align*}
& \operatorname{tr}(A \cdot B \cdot C)=\operatorname{tr}(C \cdot A \cdot B)=\operatorname{tr}(B \cdot C \cdot A)  \tag{6.98}\\
& \operatorname{tr}(A \cdot C \cdot B)=\operatorname{tr}(B \cdot A \cdot C)=\operatorname{tr}(C \cdot B \cdot A) . \tag{6.99}
\end{align*}


Die Gleichheit der Werte der beiden Zeilen muss jedoch nicht gelten. Die Spur bleibt im allgemeinen nur erhalten, wenn man die Faktoren eines Matrix-Produkts zyklisch vertauscht. Bei beliebigen Änderungen der Reihenfolge kann der Wert der Spur sich auch ändern.

\subsection*{6.4.3 Determinante}
\subsection*{6.4.3.1 Definition}
Die Definition der Determinante nach der bekannten Leibniz-Formel basiert auf dem Begriff der Permutationen. Dies benötigt ein paar Vorbereitungen.

Definition 6.19 Permutation \& symmetrische Gruppe\\
Sei $n \in \mathbb{N}^{+}$. Die symmetrische Gruppe vom Grad $n$ ist die Menge der $n$-stelligen Permutationen, d.h.


\begin{equation*}
S_{n}:=\{p:\{1, \ldots, n\} \rightarrow\{1, \ldots, n\} \mid p \text { ist bijektiv }\} \tag{6.100}
\end{equation*}


Bemerkungen:\\
i) Eine Permutation $p \in S_{n}$ kann interpretiert werden als Umordnung bzw. Umsortierung der natürlichen Zahlen in $\{1, \ldots, n\}$.\\
ii) Permutationen haben ihre Hauptanwendung in der Kombinatorik, wenn es darum geht $n$ unterscheidbare Objekte auf $n$ unterscheidbare Plätze zu verteilen. $S_{6}$ entspricht dabei z.B. der Menge aller Möglichkeiten um 6 Autos auf 6 Parkplätze zu verteilen.\\
iii) Üblicherweise wird eine Permutation $p \in S_{n}$ als $2 \times n$-Matrix dargestellt gemäss

\[
\left[\begin{array}{cccc}
1 & 2 & \ldots & n  \tag{6.101}\\
p(1) & p(2) & \ldots & p(n)
\end{array}\right] .
\]

Beispiele:

\begin{itemize}
  \item $S_{1}=\left\{\left[\begin{array}{l}1 \\ 1\end{array}\right]\right\}$
\end{itemize}

\end{document}