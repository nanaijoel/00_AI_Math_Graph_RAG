\documentclass[10pt]{article}
\usepackage[ngerman]{babel}
\usepackage[utf8]{inputenc}
\usepackage[T1]{fontenc}
\usepackage{amsmath}
\usepackage{amsfonts}
\usepackage{amssymb}
\usepackage[version=4]{mhchem}
\usepackage{stmaryrd}
\usepackage{bbold}

\begin{document}
\subsection*{6.4.4 Mass-Formeln}
Mit Hilfe der Determinante lassen sich Begriffe wie Länge, Fläche, Volumen und entsprechende Verallgemeinerungen in beliebig dimensionalen Euklid-Räumen einführen und berechnen. Dazu machen wir folgende Definition.

\section*{Definition 6.22 Gram-Matrix}
Seien $m, n \in \mathbb{N}^{+}$und $\mathbf{v}_{1}, \ldots, \mathbf{v}_{m} \in \mathbb{R}^{n}$. Die Gram-Matrix dieser Vektoren ist

\[
G\left(\mathbf{v}_{1} ; \ldots ; \mathbf{v}_{m}\right):=\left[\begin{array}{cccc}
\left\langle\mathbf{v}_{1}, \mathbf{v}_{1}\right\rangle & \left\langle\mathbf{v}_{1}, \mathbf{v}_{2}\right\rangle & \ldots & \left\langle\mathbf{v}_{1}, \mathbf{v}_{m}\right\rangle  \tag{6.134}\\
\left\langle\mathbf{v}_{2}, \mathbf{v}_{1}\right\rangle & \left\langle\mathbf{v}_{2}, \mathbf{v}_{2}\right\rangle & \ldots & \left\langle\mathbf{v}_{2}, \mathbf{v}_{m}\right\rangle \\
\vdots & \vdots & \vdots & \vdots \\
\left\langle\mathbf{v}_{m}, \mathbf{v}_{1}\right\rangle & \left\langle\mathbf{v}_{m}, \mathbf{v}_{2}\right\rangle & \ldots & \left\langle\mathbf{v}_{m}, \mathbf{v}_{m}\right\rangle
\end{array}\right] .
\]

Bemerkungen:\\
i) Die Komponenten der Gram-Matrix sind gerade alle möglichen Skalar-Produkte, die sich aus den Vektoren $\mathbf{v}_{1}, \ldots, \mathbf{v}_{m}$ bilden lassen. Davon gibt es insgesamt $m^{2}$ und konsequenterweise gilt $G \in \mathbb{M}(m, m, \mathbb{R})$.\\
ii) Wegen der Symmetrie des Skalar-Produkts muss gelten


\begin{equation*}
G^{T}=G, \tag{6.135}
\end{equation*}


d.h. $G$ ist symmetrisch.\\
iii) Wegen der positiven Definitheit des Skalar-Produkts muss gelten


\begin{equation*}
\operatorname{det}(G) \geq 0 . \tag{6.136}
\end{equation*}


iv) Für die Standard-Einheitsvektoren gilt


\begin{equation*}
G\left(\hat{\mathbf{e}}_{1} ; \ldots ; \hat{\mathbf{e}}_{n}\right)=\mathbb{1} . \tag{6.137}
\end{equation*}


v) Es sei $A \in \mathbb{M}(n, m, \mathbb{R})$ die Matrix, deren Spalten gerade die Vektoren $\mathbf{v}_{1}, \ldots, \mathbf{v}_{m}$ sind, d.h.

\[
A=\left[\begin{array}{llll}
\mathbf{v}_{1} & \mathbf{v}_{2} & \ldots & \mathbf{v}_{m} \tag{6.138}
\end{array}\right] .
\]

Dann gilt


\begin{align*}
\underline{\underline{A^{T} \cdot A}} & =\left[\begin{array}{c}
\mathbf{v}_{1}^{T} \\
\mathbf{v}_{2}^{T} \\
\vdots \\
\mathbf{v}_{m}^{T}
\end{array}\right] \cdot\left[\begin{array}{llll}
\mathbf{v}_{1} & \mathbf{v}_{2} & \ldots & \mathbf{v}_{m}
\end{array}\right]=\left[\begin{array}{ccccc}
\mathbf{v}_{1}^{T} \cdot \mathbf{v}_{1} & \mathbf{v}_{1}^{T} \cdot \mathbf{v}_{2} & \ldots & \mathbf{v}_{1}^{T} \cdot \mathbf{v}_{m} \\
\mathbf{v}_{2}^{T} \cdot \mathbf{v}_{1} & \mathbf{v}_{2}^{T} \cdot \mathbf{v}_{2} & \ldots & \mathbf{v}_{2}^{T} \cdot \mathbf{v}_{m} \\
\vdots & \vdots & \vdots & \vdots \\
\mathbf{v}_{m}^{T} \cdot \mathbf{v}_{1} & \mathbf{v}_{m}^{T} \cdot \mathbf{v}_{2} & \ldots & \mathbf{v}_{m}^{T} \cdot \mathbf{v}_{m}
\end{array}\right] \\
& =\left[\begin{array}{cccc}
\left\langle\mathbf{v}_{1}, \mathbf{v}_{1}\right\rangle & \left\langle\mathbf{v}_{1}, \mathbf{v}_{2}\right\rangle & \ldots & \left\langle\mathbf{v}_{1}, \mathbf{v}_{m}\right\rangle \\
\left\langle\mathbf{v}_{2}, \mathbf{v}_{1}\right\rangle & \left\langle\mathbf{v}_{2}, \mathbf{v}_{2}\right\rangle & \ldots & \left\langle\mathbf{v}_{2}, \mathbf{v}_{m}\right\rangle \\
\vdots & \vdots & \vdots & \vdots \\
\left\langle\mathbf{v}_{m}, \mathbf{v}_{1}\right\rangle & \left\langle\mathbf{v}_{m}, \mathbf{v}_{2}\right\rangle & \ldots & \left\langle\mathbf{v}_{m}, \mathbf{v}_{m}\right\rangle
\end{array}\right]=\underline{\underline{G}} . \tag{6.139}
\end{align*}



\end{document}