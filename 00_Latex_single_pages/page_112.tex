\documentclass[10pt]{article}
\usepackage[ngerman]{babel}
\usepackage[utf8]{inputenc}
\usepackage[T1]{fontenc}
\usepackage{amsmath}
\usepackage{amsfonts}
\usepackage{amssymb}
\usepackage[version=4]{mhchem}
\usepackage{stmaryrd}
\usepackage{bbold}
\usepackage{graphicx}
\usepackage[export]{adjustbox}
\graphicspath{ {./images/} }

\begin{document}
iii) Die Spiegelungsmatrix $S(\hat{\mathbf{n}})$ ist symmetrisch und es folgt\\
$\underline{\underline{S^{-1}(\hat{\mathbf{n}})}=S^{T}(\hat{\mathbf{n}})=S(\hat{\mathbf{n}})}$.\\
Wie geometrisch offensichtlich, ist die Spiegelungsmatrix ihre eigene Inverse.\\
iv) Betrachtet man einen Normalen-Vektor $\mathbf{n} \in \mathbb{R}^{n} \backslash\{0\}$ der noch nicht normiert ist, dann kann man diese Normierung auch direkt in die Householder-Formel (6.74) einbauen. Es gilt\\
$\underline{\underline{S(\mathbf{n})}}=\mathbb{1}-2 \cdot \hat{\mathbf{n}} \cdot \hat{\mathbf{n}}^{T}=\mathbb{1}-2 \cdot \frac{\mathbf{n}}{|\mathbf{n}|} \cdot \frac{\mathbf{n}^{T}}{|\mathbf{n}|}=\mathbb{1}-2 \cdot \frac{\mathbf{n} \cdot \mathbf{n}^{T}}{|\mathbf{n}|^{2}}=\underline{\underline{\mathbb{1}}-2 \cdot \frac{\mathbf{n} \cdot \mathbf{n}^{T}}{\langle\mathbf{n}, \mathbf{n}\rangle}}$.\\
v) Beispiel-Codes zum Erzeugen von Householder-Spiegelungsmatrizen gemäss (6.80) mit gängiger Software.

\begin{center}
\begin{tabular}{|l|l|}
\hline
MATLAB/Octave & \includegraphics[max width=\textwidth]{2025_05_07_16a88d88102cc8e7d192g-1}
 \\
\hline
Python/Numpy & \begin{tabular}{l}
import numpy as np; \\
def $S(n)$ : \\
$\mathrm{nn}=\mathrm{n} / \mathrm{np}$.linalg.norm(n); \\
M=np.eye(n.shape[0])-2*nn@nn.T; \\
return M; \\
\end{tabular} \\
\hline
\end{tabular}
\end{center}

\subsection*{6.3.3.2 Rotationen in 3D}
Wir betrachten die Rotation in $\mathbb{R}^{3}$ um den Winkel $\varphi \in \mathbb{R}$ rechtshändig um die Drehachse in Richtung $\hat{\boldsymbol{\varphi}} \in \mathbb{R}^{3}$. Die Situation ist in der folgenden Skizze dargestellt.\\
\includegraphics[max width=\textwidth, center]{2025_05_07_16a88d88102cc8e7d192g-1(1)}

Zunächst bemerken wir, dass sich aus den Bestimmungsstücken $\varphi$ und $\hat{\boldsymbol{\varphi}}$ der Rotation auf kanonische Weise ein Vektor bilden lässt gemäss


\begin{equation*}
\varphi=\varphi \cdot \hat{\varphi} . \tag{6.81}
\end{equation*}


Dieser Vektor zeigt demnach in Richtung der Drehachse und hat gerade die Länge des Drehwinkels. Als nächstes betrachten wir die sogenannte Generator-Matrix.


\end{document}