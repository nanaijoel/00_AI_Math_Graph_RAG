\documentclass[10pt]{article}
\usepackage[ngerman]{babel}
\usepackage[utf8]{inputenc}
\usepackage[T1]{fontenc}
\usepackage{amsmath}
\usepackage{amsfonts}
\usepackage{amssymb}
\usepackage[version=4]{mhchem}
\usepackage{stmaryrd}
\usepackage{bbold}

\begin{document}
\subsection*{7.2 Lineare Abbildungen}
\subsection*{7.2.1 Definition}
Der Begriff lineare Abbildung lässt sich zwischen allgemeinen Vektorräumen definieren.

\section*{Definition 7.7 Lineare Abbildung}
Seien $(V, \mathbb{K},+, \cdot)$ und $(W, \mathbb{K},+, \cdot)$ zwei Vektorräume über dem gleichen Zahlenkörper $\mathbb{K}$. Eine Abbildung der Form


\begin{equation*}
a: V \rightarrow W \tag{7.34}
\end{equation*}


heisst lineare Abbildung, falls für alle $\mathbf{u}, \mathbf{v} \in V$ und $x, y \in \mathbb{K}$ gilt


\begin{equation*}
a(x \cdot \mathbf{u}+y \cdot \mathbf{v})=x \cdot a(\mathbf{u})+y \cdot a(\mathbf{v}) \tag{7.35}
\end{equation*}


Bemerkungen:\\
i) Für diese Definition ist es sehr wichtig, dass die Vektorräume $V$ und $W$ über dem gleichen Zahlenkörper $\mathbb{K}$ definiert sind.\\
ii) Diese Definition entspricht der Definition 6.14 aus Abschnitt 6.2.1.\\
iii) Eine lineare Abbildung erkennt man daran, dass sie die Struktur einer Linearkombination respektiert.\\
iv) Für alle linearen Abbildungen gilt offensichtlich


\begin{equation*}
\underline{\underline{a(0)}}=a(0 \cdot 0)=0 \cdot a(0)=\underline{\underline{0}} . \tag{7.36}
\end{equation*}


Beispiele:

\begin{itemize}
  \item Die bereits bekannten geometrisch definierten linearen Abbildungen in $\mathbb{R}^{n}$ wie Spiegelungen, Drehungen, Projektionen etc...
  \item Die Ableitung $d: \mathcal{P}_{n}(\mathbb{R}) \rightarrow \mathcal{P}_{n}(\mathbb{R})$ oder $d: \mathcal{P}_{n}(\mathbb{R}) \rightarrow \mathcal{P}_{n-1}(\mathbb{R})$.
  \item Die Orthogonal-Projektion in $\mathcal{L}^{2}(\mathbb{R})$.
\end{itemize}

\subsection*{7.2.2 Matrix-Darstellung}
Betrachtet man eine lineare Abbildung zwischen zwei endlich dimensionalen Vektorräumen und wählt in beiden jeweils eine Basis, dann kann man die lineare Abbildung durch eine Abbildungsmatrix darstellen.

\section*{Definition 7.8 Abbildungsmatrix}
Seien $(V, \mathbb{K},+, \cdot)$ und $(W, \mathbb{K},+, \cdot)$ zwei Vektorräume über dem gleichen Zahlenkörper $\mathbb{K}$ mit den endlichen Dimensionen $\operatorname{dim}(V)=n \in \mathbb{N}^{+}$bzw. $\operatorname{dim}(W)=m \in \mathbb{N}^{+}$und Basen $\left\{\mathbf{e}_{1}, \ldots, \mathbf{e}_{n}\right\} \subseteq$ $V$ bzw. $\left\{\mathbf{E}_{1}, \ldots, \mathbf{E}_{m}\right\} \subseteq W$ sowie $a: V \rightarrow W$ eine lineare Abbildung. Die Abbildungsmatrix von $a$ bezüglich der gewählten Basen ist die Matrix $A \in \mathbb{M}(m, n, \mathbb{K})$ mit den Komponenten $A^{i}{ }_{j} \in \mathbb{K}$, so dass für alle $j \in\{1, \ldots, n\}$ gilt


\begin{equation*}
a\left(\mathbf{e}_{j}\right)=\sum_{i=1}^{m} A_{j}^{i} \cdot \mathbf{E}_{i} . \tag{7.37}
\end{equation*}



\end{document}