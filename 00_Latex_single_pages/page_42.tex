\documentclass[10pt]{article}
\usepackage[ngerman]{babel}
\usepackage[utf8]{inputenc}
\usepackage[T1]{fontenc}
\usepackage{amsmath}
\usepackage{amsfonts}
\usepackage{amssymb}
\usepackage[version=4]{mhchem}
\usepackage{stmaryrd}
\usepackage{bbold}

\begin{document}
\subsection*{2.5.3 Hesse-Matrix}
Wir betrachten die folgende Definition.

\section*{Definition 2.20 Hesse-Matrix}
Seien $n \in \mathbb{N}^{+}$und $f: \mathbb{R}^{n} \rightarrow \mathbb{R}$ eine zweifach differentierbare reellwertige Funktion. Die HesseMatrix von $f$ ist das Vektorfeld

\[
\boldsymbol{\nabla}^{2} f:=\left[\begin{array}{cccc}
f_{1,1} & f_{1,2} & \ldots & f_{1, n}  \tag{2.142}\\
f_{2,1} & f_{2,2} & \ldots & f_{2, n} \\
\vdots & \vdots & \vdots & \vdots \\
f_{, n, 1} & f_{, n, 2} & \ldots & f_{, n, n}
\end{array}\right] .
\]

Beispiele:

\begin{itemize}
  \item Wir betrachten\\
$f(x ; y):=x^{2} \cdot y^{2}$.\\
Der Gradient von $f$ ist
\end{itemize}

\[
\boldsymbol{\nabla} f=\left[\begin{array}{l}
f_{, 1}  \tag{2.144}\\
f_{, 2}
\end{array}\right]=\left[\begin{array}{l}
2 x \cdot y^{2} \\
x^{2} \cdot 2 y
\end{array}\right]=\left[\begin{array}{l}
2 x y^{2} \\
2 x^{2} y
\end{array}\right] .
\]

Die Hesse-Matrix von $f$ ist

Wir betrachten den folgenden Satz.

\section*{Satz 2.14 Schwarz-Clairaut-Young-Satz}
Seien $n \in \mathbb{N}^{+}$und $f: \mathbb{R}^{n} \rightarrow \mathbb{R}$ eine zweifach differentierbare reellwertige Funktion mit HesseMatrix $H \in \mathbb{M}(n, n, \mathbb{R})$, dann gilt


\begin{equation*}
H^{T}=H . \tag{2.146}
\end{equation*}


Bemerkungen:\\
i) Die Symmetrie der Hesse-Matrix gemäss Schwarz-Clairaut-Young-Satz ist äquivalent zur Tatsache, dass die partiellen Ableitungen vertauscht werden dürfen, d.h. für alle $\mu, \nu \in\{1, \ldots, n\}$ gilt


\begin{equation*}
f_{\nu, \mu}=f_{, \mu, \nu} . \tag{2.147}
\end{equation*}


ii) Weil die Hesse-Matrix symmetrisch ist, ist sie diagonalisierbar, d.h. ähnlich zu einer diagonalen Matrix.

Wir betrachten die folgende Definition.

\section*{Definition 2.21 Laplace-Ableitung}
Seien $n \in \mathbb{N}^{+}$und $f: \mathbb{R}^{n} \rightarrow \mathbb{R}$ eine zweifach differentierbare reellwertige Funktion. Die LA-PLACE-Ableitung von $f$ ist


\begin{equation*}
\Delta f:=\operatorname{tr}\left(\nabla^{2} f\right)=f_{, 1,1}+f_{, 2,2}+\ldots+f_{, n, n} . \tag{2.148}
\end{equation*}



\end{document}