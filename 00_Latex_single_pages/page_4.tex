\documentclass[10pt]{article}
\usepackage[ngerman]{babel}
\usepackage[utf8]{inputenc}
\usepackage[T1]{fontenc}
\usepackage{amsmath}
\usepackage{amsfonts}
\usepackage{amssymb}
\usepackage[version=4]{mhchem}
\usepackage{stmaryrd}
\usepackage{bbold}
\usepackage{graphicx}
\usepackage[export]{adjustbox}
\graphicspath{ {./images/} }

\begin{document}
S3 Für $N$ gross genug erhält man in guter Näherung


\begin{equation*}
A \approx \sum_{k=1}^{N} \delta A_{k}=\sum_{k=1}^{N} f\left(x_{k}\right) \cdot \delta x . \tag{1.7}
\end{equation*}


Unter Verwendung des modernen Cauchy-Grenzwertbegriffs erhält man\\
$A=\int_{x_{0}}^{x_{\mathrm{E}}} f(x) \mathrm{d} x=\lim _{N \rightarrow \infty} \sum_{k=1}^{N} f\left(x_{k}\right) \cdot \delta x \approx \sum_{k=1}^{N} f\left(x_{k}\right) \cdot \delta x$.\\
Bemerkungen:\\
i) Aus (1.8) lässt sich die Herkunft des "Integral-Hakens" erkennen: Es ist ein grosses "S" für Summe. Analog wurde aus dem Faktor $\delta x$ das Differentialsymbol dx.\\
ii) Früher berechnete man Integrale im Sinne von (1.8) näherungsweise durch Summen .\\
iii) Heute berechnet man Summen im Sinne von (1.8) durch Integrale mit Hilfe der Newton-Leibniz-Formel.

\subsection*{1.2.3 Konzeption am Urbeispiel}
Wir betrachten den Graphen einer stetigen Funktion $f: \mathbb{R} \rightarrow \mathbb{R}$ entlang eines reellen Intervalls $\left[x_{0}, x_{\mathrm{E}}\right]$. Die Situation ist im folgenden $x-y$-Diagramm dargestellt.\\
\includegraphics[max width=\textwidth, center]{2025_05_07_eb908acbc8be7f6e4794g-1}

Um die Fläche zwischen der $x$-Achse oberhalb des Intervalles $\left[x_{0}, x_{\mathrm{E}}\right]$ und dem Graphen von $f$ zu berechnen, verwenden wir einen Archimedes-Cauchy-Riemann-Approximationsprozess. Dabei gehen wir nach folgenden Schritten vor.

S1 Lokal: Wir betrachten einen kleinen Streifen an der Position $x$ der Breite $\delta x$. Die Fläche dieses Streifens beträgt\\
$\underline{\delta A} \approx \underline{f(x) \cdot \delta x}$.\\
S2 Global: Durch Integration über $x$ können wir die gesamte Fläche $A$ berechnen. Wir erhalten $\underline{\underline{A}}=\int_{x_{0}}^{x_{\mathrm{E}}} f(x) \mathrm{d} x=\left.[F(x)]\right|_{x_{0}} ^{x_{\mathrm{E}}}=\underline{\underline{F\left(x_{\mathrm{E}}\right)-F\left(x_{0}\right) .}}$


\end{document}