\documentclass[10pt]{article}
\usepackage[ngerman]{babel}
\usepackage[utf8]{inputenc}
\usepackage[T1]{fontenc}
\usepackage{amsmath}
\usepackage{amsfonts}
\usepackage{amssymb}
\usepackage[version=4]{mhchem}
\usepackage{stmaryrd}
\usepackage{bbold}

\begin{document}
Beispiele:

\begin{itemize}
  \item Für jedes $n \in \mathbb{N}^{+}$bilden die Standard-Einheitsvektoren
\end{itemize}

\[
\left\{\hat{\mathbf{e}}_{1}=\left[\begin{array}{c}
1  \tag{7.23}\\
0 \\
0 \\
\vdots \\
0
\end{array}\right], \hat{\mathbf{e}}_{2}=\left[\begin{array}{c}
0 \\
1 \\
0 \\
\vdots \\
0
\end{array}\right], \ldots, \hat{\mathbf{e}}_{n}=\left[\begin{array}{c}
0 \\
0 \\
\vdots \\
0 \\
1
\end{array}\right]\right\}
\]

eine Basis von $\mathbb{K}^{n}$, die sogenannte Standard-Basis. Somit folgt


\begin{equation*}
\operatorname{dim}\left(\mathbb{K}^{n}\right)=n . \tag{7.24}
\end{equation*}


\begin{itemize}
  \item Für jedes $n \in \mathbb{N}^{+}$bilden die Monome
\end{itemize}


\begin{equation*}
\left\{1, x, x^{2}, x^{3}, \ldots, x^{n}\right\} \tag{7.25}
\end{equation*}


eine Basis des Polynomraums $\mathcal{P}_{n}(\mathbb{R})$. Somit folgt


\begin{equation*}
\operatorname{dim}\left(\mathcal{P}_{n}(\mathbb{R})\right)=n+1 \tag{7.26}
\end{equation*}


\begin{itemize}
  \item Die Funktionenräume $C(\mathbb{R})$ und $\mathcal{L}^{2}(\mathbb{R})$ können nicht endlich erzeugt werden und haben daher unendliche Dimension.
\end{itemize}

Ist in einem Vektorraum ( $V, \mathbb{K},+, \cdot)$ eine Basis $B=\left\{\mathbf{e}_{1}, \ldots, \mathbf{e}_{n}\right\}$ gewählt, dann lässt sich jeder Vektor eindeutig Linearkombination des Basis-Vektoren schreiben gemäss


\begin{equation*}
\mathbf{v}=\sum_{k=1}^{n} v^{k} \cdot \mathbf{e}_{k} \tag{7.27}
\end{equation*}


Die Koeffizienten $v^{1}, \ldots, v^{n} \in \mathbb{K}$ können wie im Falle der Euklid-Räume $\mathbb{R}^{n}$ in eine SpaltenMatrix geschrieben werden, d.h. man identifiziert

\[
\mathbf{v}=\left[\begin{array}{c}
v^{1}  \tag{7.28}\\
\vdots \\
v^{n}
\end{array}\right]
\]

Die Operationen + und $\cdot$ im Vektorraum lassen sich dann wieder durch die gewohnten Operationen in den Komponenten realisieren.

Beispiel:

\begin{itemize}
  \item Wir betrachten den Vektorraum $\mathcal{P}_{2}(\mathbb{R})$ der quadratischen Funktionen mit reellen Koeffizienten und wählen die Basis $\left\{1, x, x^{2}\right\}$. Dann identifizieren wir
\end{itemize}

\[
f(x)=3 x^{2}+5 x+7 \mapsto \mathbf{f}=\left[\begin{array}{l}
7  \tag{7.29}\\
5 \\
3
\end{array}\right]
\]


\end{document}