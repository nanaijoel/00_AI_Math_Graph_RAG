\documentclass[10pt]{article}
\usepackage[ngerman]{babel}
\usepackage[utf8]{inputenc}
\usepackage[T1]{fontenc}
\usepackage{amsmath}
\usepackage{amsfonts}
\usepackage{amssymb}
\usepackage[version=4]{mhchem}
\usepackage{stmaryrd}
\usepackage{bbold}

\begin{document}
\subsection*{2.2 Parametrisierte Kurven \& Linienintegrale}
\subsection*{2.2.1 Parametrisierte Kurven}
\subsection*{2.2.1.1 Definition}
Wir betrachten die folgende Definition.\\
Definition 2.6 Parametrisierte Kurve\\
Seien $n \in \mathbb{N}^{+}$und $\tau_{0}, \tau_{\mathrm{E}} \in \mathbb{R}$ mit $\tau_{0}<\tau_{\mathrm{E}}$. Eine parametrisierte Kurve ist eine differentierbare Funktion der Form


\begin{align*}
& \mathbf{s}:\left[\tau_{0}, \tau_{\mathrm{E}}\right] \rightarrow \mathbb{R}^{n} \\
& \tau \mapsto \mathbf{s}(\tau):=\left[\begin{array}{c}
s_{1}(\tau) \\
s_{2}(\tau) \\
\vdots \\
s_{n}(\tau)
\end{array}\right] . \tag{2.36}
\end{align*}


Analog zur Vektor-Kinematik in der Physik definiert man weitere Begriffe, um eine parametrisierte Kurve zu beschreiben. Wir betrachten dazu die folgende Definition.

Definition 2.7 Weitere Begriffe\\
Seien $n \in \mathbb{N}^{+}, \tau_{0}, \tau_{\mathrm{E}} \in \mathbb{R}$ mit $\tau_{0}<\tau_{\mathrm{E}}$ und $\mathbf{s}:\left[\tau_{0}, \tau_{\mathrm{E}}\right] \rightarrow \mathbb{R}^{n}$ eine parametrisierte Kurve.\\
(a) Geschwindigkeitsvektor:\\
(d) Beschleunigungsvektor:

$$
\mathbf{v}(\tau):=\dot{\mathbf{s}}(\tau)
$$

$$
\mathbf{a}(\tau):=\dot{\mathbf{v}}(\tau)
$$

(b) Bahngeschwindigkeit:

$$
v(\tau):=|\mathbf{v}(\tau)|
$$

(c) Bahnvektor für $v(\tau) \neq 0$ :

$$
\hat{\mathbf{e}}(\tau):=\hat{\mathbf{v}}(\tau)
$$

(e) Bahnbeschleunigung:

$$
a_{\mathrm{B}}(\tau):=\langle\mathbf{a}(\tau), \hat{\mathbf{e}}(\tau)\rangle
$$

(f) Bahn:

$$
B=\mathbf{s}\left(\left[\tau_{0}, \tau_{\mathrm{E}}\right]\right)
$$

Bemerkungen:\\
i) Der Ortsvektor $\mathbf{s}(\tau)$ zeigt für jedes $\tau$ vom Ursprung auf den entsprechenden Punkt der Bahn der parametrisierten Kurve in $\mathbb{R}^{n}$.\\
ii) Für die Masseinheiten erhalten wir


\begin{align*}
& {[\mathbf{v}]=[v]=\frac{[\mathbf{s}]}{[\tau]}}  \tag{2.37}\\
& {[\mathbf{a}]=\left[a_{\mathrm{B}}\right]=\frac{[\mathbf{s}]}{[\tau]^{2}} .} \tag{2.38}
\end{align*}



\end{document}