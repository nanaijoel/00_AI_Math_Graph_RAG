\documentclass[10pt]{article}
\usepackage[ngerman]{babel}
\usepackage[utf8]{inputenc}
\usepackage[T1]{fontenc}
\usepackage{amsmath}
\usepackage{amsfonts}
\usepackage{amssymb}
\usepackage[version=4]{mhchem}
\usepackage{stmaryrd}
\usepackage{bbold}

\begin{document}

\begin{align*}
& =\mathbf{v}+(1-\cos (\varphi)) \cdot J(\hat{\boldsymbol{\varphi}}) \cdot(J(\hat{\boldsymbol{\varphi}}) \cdot \mathbf{v})+\sin (\varphi) \cdot J(\hat{\boldsymbol{\varphi}}) \cdot \mathbf{v} \\
& =\mathbf{v}+(1-\cos (\varphi)) \cdot(J(\hat{\boldsymbol{\varphi}}) \cdot J(\hat{\boldsymbol{\varphi}})) \cdot \mathbf{v}+\sin (\varphi) \cdot J(\hat{\boldsymbol{\varphi}}) \cdot \mathbf{v} \\
& =\mathbb{1} \cdot \mathbf{v}+(1-\cos (\varphi)) \cdot J^{2}(\hat{\boldsymbol{\varphi}}) \cdot \mathbf{v}+\sin (\varphi) \cdot J(\hat{\boldsymbol{\varphi}}) \cdot \mathbf{v} \\
& =\underline{\underline{\left(\mathbb{1}+(1-\cos (\varphi)) \cdot J^{2}(\hat{\boldsymbol{\varphi}})+\sin (\varphi) \cdot J(\hat{\boldsymbol{\varphi}})\right) \cdot \mathbf{v}}} \tag{6.87}
\end{align*}


Dies impliziert (6.84) und wir haben den Satz bewiesen.\\
Bemerkungen:\\
i) Es gilt $R(\boldsymbol{\varphi}) \in \mathrm{O}(n)$.\\
ii) In jedem Fall gilt


\begin{equation*}
\underline{\underline{R^{-1}}(\varphi)=R^{T}(\varphi)=R(-\varphi) .} \tag{6.88}
\end{equation*}


Wie geometrisch offensichtlich, ist die Rotation um den Winkel - $\varphi$ die Inverse der Rotation um den Winkel $\varphi$.\\
iii) In der Rodrigues-Formel (6.84) steht in beiden Termen jeweils $J(\hat{\boldsymbol{\varphi}})$ und nicht $J(\boldsymbol{\varphi})$. Die Verwechslung der beiden ist eine berüchtigte Fehlerquelle.\\
iv) Beispiel-Codes zum Erzeugen von Rodrigues-Rotationsmatrizen gemäss (6.84) mit gängiger Software.

\begin{center}
\begin{tabular}{|l|l|}
\hline
MATLAB/Octave & $\mathrm{J}=$ @ (w) $[0,-\mathrm{w}(3), \mathrm{w}(2) ; \mathrm{w}(3), 0,-\mathrm{w}(1) ;-\mathrm{w}(2), \mathrm{w}(1), 0]$; R=@(n, phi)eye(3)+(1-cos(phi))*J(n/norm(n)) ${ }^{-2}$ $+\sin (\mathrm{phi}) * J(\mathrm{n} / \mathrm{norm}(\mathrm{n}))$; \\
\hline
Python/Numpy & \begin{tabular}{l}
import numpy as np; \\
def J(w): \\
M=np.array ([[0, -w[2],w[1]],[w[2],0,-w[0]], [-w[1],w[0] ,0]]); \\
return M; \\
def R(phi,n): \\
nn=n/np.linalg.norm(n); \\
$M=n p . e y e(3)+(1-n p . \cos (p h i)) * J(n n) @ J(n n)$ \\
$+\mathrm{np} . \sin (\mathrm{phi}) * J(\mathrm{nn})$; \\
return M; \\
\end{tabular} \\
\hline
\end{tabular}
\end{center}


\end{document}