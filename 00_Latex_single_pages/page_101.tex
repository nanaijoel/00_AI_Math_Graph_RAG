\documentclass[10pt]{article}
\usepackage[ngerman]{babel}
\usepackage[utf8]{inputenc}
\usepackage[T1]{fontenc}
\usepackage{amsmath}
\usepackage{amsfonts}
\usepackage{amssymb}
\usepackage[version=4]{mhchem}
\usepackage{stmaryrd}

\begin{document}
vi) Beispiel-Codes zum Erzeugen von diagonalen Matrizen mit gängiger Software.

\begin{center}
\begin{tabular}{|l|l|}
\hline
MATLAB/Octave & $\mathrm{M}=\operatorname{diag}([1,2,3])$ \\
\hline
Mathematica/WolframAlpha & M=DiagonalMatrix [\{1, 2, 3\}] \\
\hline
Python/Numpy & \begin{tabular}{l}
import numpy as np; \\
M=np.diag([1, 2, 3]) \\
\end{tabular} \\
\hline
Python/Sympy & \begin{tabular}{l}
import sympy as sp; \\
M=sp.diag(1,2,3) \\
\end{tabular} \\
\hline
\end{tabular}
\end{center}

Beispiele:

\begin{itemize}
  \item \hspace{0pt} [3]
  \item $\left[\begin{array}{ll}1 & 0 \\ 0 & 2\end{array}\right]$
  \item $\left[\begin{array}{rrr}-7 & 0 & 0 \\ 0 & 3 & 0 \\ 0 & 0 & -1\end{array}\right]$
\end{itemize}

\end{document}