\documentclass[10pt]{article}
\usepackage[ngerman]{babel}
\usepackage[utf8]{inputenc}
\usepackage[T1]{fontenc}
\usepackage{amsmath}
\usepackage{amsfonts}
\usepackage{amssymb}
\usepackage[version=4]{mhchem}
\usepackage{stmaryrd}

\begin{document}
Fall 2: $M$ ist eine Fläche mit geschlossener Randkurve $\gamma$. Dann gilt


\begin{align*}
\Phi_{\mathbf{J}} & =I \equiv \text { Stromstärke durch die Fläche }[A]  \tag{2.131}\\
\Phi_{\mathbf{S}} & =P \equiv \text { Leistung durch die Fläche }[W]  \tag{2.132}\\
U_{\text {ind }} & =\Upsilon_{\mathbf{E}}=\oint_{\gamma}\langle\mathbf{E}, \hat{\mathbf{e}}\rangle \mathrm{d} s=-\left(\int_{M}\langle\mathbf{B}, \hat{\mathbf{n}}\rangle \mathrm{d} A\right)^{\bullet}=-\dot{\Phi}_{\mathbf{B}} \tag{2.133}
\end{align*}



\end{document}