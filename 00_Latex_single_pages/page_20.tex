\documentclass[10pt]{article}
\usepackage[ngerman]{babel}
\usepackage[utf8]{inputenc}
\usepackage[T1]{fontenc}
\usepackage{amsmath}
\usepackage{amsfonts}
\usepackage{amssymb}
\usepackage[version=4]{mhchem}
\usepackage{stmaryrd}
\usepackage{graphicx}
\usepackage[export]{adjustbox}
\graphicspath{ {./images/} }

\begin{document}
iii) Man kann zeigen, dass das Linienintegral I bis auf das Vorzeichen nicht von der Wahl der Parametrisierung sondern nur von der Bahn abhängt (sofern mehrfache Durchläufe auch mehrfach gerechnet werden).\\
iv) Wählt man für den Kurvenparameter die geometrische Weglänge $\tau=s$ entlang der Bahn ("Kilometrierung"), dann gilt\\
$\mathbf{v}(s)=v(s) \cdot \hat{\mathbf{e}}(s)=1 \cdot \hat{\mathbf{e}}(s)=\hat{\mathbf{e}}(s)$.\\
Die Situation ist in der folgenden Skizze dargestellt.\\
\includegraphics[max width=\textwidth, center]{2025_05_07_92f2456d31e5eb52c995g-1}

Somit folgt\\
$\underline{\underline{I}}=\int_{\tau_{0}}^{\tau_{\mathrm{E}}}\langle\mathbf{w}, \mathbf{v}\rangle \mathrm{d} \tau=\underline{\underline{\int_{s_{0}}}\langle\mathbf{w}, \hat{\mathbf{e}}\rangle \mathrm{d} s}$.\\
v) Um ein Linienintegral auszurechnen, müssen die Koordinaten der Punkte entlang der parametrisierten Kurve im Vektorfeld eingesetzt werden. Vollständig ausgeschrieben mit allen Abhängigkeiten ergibt dies\\
$I=\int_{\tau_{0}}^{\tau_{\mathrm{E}}}\langle\mathbf{w}(\mathbf{s}(\tau)), \mathbf{v}(\tau)\rangle \mathrm{d} \tau=\int_{s_{0}}^{s_{\mathrm{E}}}\langle\mathbf{w}(\mathbf{s}(s)), \hat{\mathbf{e}}(s)\rangle \mathrm{d} s$.\\
vi) In der Literatur findet man für Linienintegrale entlang einer Kurve $\gamma$ die Schreibweisen $I=\int_{\tau_{0}}^{\tau_{\mathrm{E}}}\langle\mathbf{w}, \mathbf{v}\rangle \mathrm{d} \tau=\int_{s_{0}}^{s_{\mathrm{E}}}\langle\mathbf{w}, \hat{\mathbf{e}}\rangle \mathrm{d} s=\int_{\gamma}\langle\mathbf{w}, \hat{\mathbf{e}}\rangle \mathrm{d} s=\int_{\gamma} \mathbf{w} \cdot \mathrm{d} \mathbf{s}$.\\
vii) Ein Linienintegral über eine geschlossene Kurve heisst Zirkulation des Vektorfeldes entlang der betreffenden Kurve.\\
\includegraphics[max width=\textwidth, center]{2025_05_07_92f2456d31e5eb52c995g-1(1)}

In diesem Fall verwendet man das Ring-Integralzeichen, d.h.\\
$\Upsilon=\oint_{\tau_{0}}^{\tau_{\mathrm{E}}}\langle\mathbf{w}, \mathbf{v}\rangle \mathrm{d} \tau=\oint_{s_{0}}^{s_{\mathrm{E}}}\langle\mathbf{w}, \hat{\mathbf{e}}\rangle \mathrm{d} s=\oint_{\gamma} \mathbf{w} \cdot \mathrm{d} \mathbf{s}$.\\
viii) Das Linienintegral eines Vektorfeldes $\mathbf{w}$ entlang einer Kurve ist das Integral des Anteils von w, welcher entlang ê und somit entlang der Kurve zeigt. Der zu ê bzw. zur Kurve senkrechte Anteil von $\mathbf{w}$ ist für das Linienintegral irrelevant.


\end{document}