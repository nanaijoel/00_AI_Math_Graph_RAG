\documentclass[10pt]{article}
\usepackage[ngerman]{babel}
\usepackage[utf8]{inputenc}
\usepackage[T1]{fontenc}
\usepackage{amsmath}
\usepackage{amsfonts}
\usepackage{amssymb}
\usepackage[version=4]{mhchem}
\usepackage{stmaryrd}

\begin{document}
Variante 1: Durch strukturelle Ergänzung erhalten wir


\begin{align*}
\underline{\underline{F(x)}} & =\int x \cdot \cos \left(x^{2}\right) \mathrm{d} x=\int \frac{1}{2} \cdot 2 x \cdot \cos \left(x^{2}\right) \mathrm{d} x=\frac{1}{2} \int 2 x \cdot \cos \left(x^{2}\right) \mathrm{d} x \\
& =\frac{1}{2} \int u^{\prime}(x) \cdot \cos (u(x)) \mathrm{d} x=\frac{1}{2} \int \cos (u) \mathrm{d} u=\frac{1}{2} \cdot \sin (u)+c=\underline{\underline{\frac{1}{2}} \cdot \sin \left(x^{2}\right)+c} \tag{3.4}
\end{align*}


Variante 2: Durch Kalkulieren mit den Differentialsymbolen erhalten wir


\begin{equation*}
\frac{\mathrm{d} u}{\mathrm{~d} x}=2 x \Leftrightarrow \mathrm{~d} u=2 x \mathrm{~d} x \Leftrightarrow \mathrm{~d} x=\frac{1}{2 x} \mathrm{~d} u \tag{3.5}
\end{equation*}


und somit


\begin{align*}
\underline{\underline{F(x)}} & =\int x \cdot \cos \left(x^{2}\right) \mathrm{d} x=\int x \cdot \cos (u) \cdot \frac{1}{2 x} \mathrm{~d} u=\frac{1}{2} \int \cos (u) \mathrm{d} u=\frac{1}{2} \cdot \sin (u)+c \\
& =\underline{\underline{\frac{1}{2}} \cdot \sin \left(x^{2}\right)+c} \tag{3.6}
\end{align*}


\begin{itemize}
  \item Wir betrachten das unbestimmte Integral
\end{itemize}


\begin{equation*}
F(x)=\int \tan (x) \mathrm{d} x=\int \frac{\sin (x)}{\cos (x)} \mathrm{d} x \tag{3.7}
\end{equation*}


Als Substitution wählen wir


\begin{equation*}
u(x):=\cos (x) \Rightarrow u^{\prime}(x)=-\sin (x) . \tag{3.8}
\end{equation*}


Durch Kalkulieren mit den Differentialsymbolen erhalten wir


\begin{equation*}
\frac{\mathrm{d} u}{\mathrm{~d} x}=-\sin (x) \Leftrightarrow \mathrm{d} u=-\sin (x) \mathrm{d} x \Leftrightarrow \mathrm{~d} x=-\frac{1}{\sin (x)} \mathrm{d} u \tag{3.9}
\end{equation*}


und somit


\begin{align*}
\underline{\underline{F(x)}} & =\int \tan (x) \mathrm{d} x=\int \frac{\sin (x)}{\cos (x)} \cdot(-1) \cdot \frac{1}{\sin (x)} \mathrm{d} u=-\int \frac{1}{u} \mathrm{~d} u=-\ln (|u|)+c \\
& =\underline{-\ln (|\cos (x)|)+c .} \tag{3.10}
\end{align*}


\begin{itemize}
  \item Wir betrachten das bestimmte Integral
\end{itemize}


\begin{equation*}
I=\int_{0}^{1} \sqrt{1-x^{2}} \mathrm{~d} x=\int_{0}^{1} \sqrt{1-u^{2}} \mathrm{~d} u \tag{3.11}
\end{equation*}


Als Substitution wählen wir


\begin{equation*}
u(\varphi):=\sin (\varphi) \Rightarrow u^{\prime}(\varphi)=\cos (\varphi) . \tag{3.12}
\end{equation*}


Durch Kalkulieren mit den Differentialsymbolen erhalten wir


\begin{equation*}
\frac{\mathrm{d} u}{\mathrm{~d} \varphi}=\cos (\varphi) \Leftrightarrow \mathrm{d} u=\cos (\varphi) \mathrm{d} \varphi \tag{3.13}
\end{equation*}



\end{document}