\documentclass[10pt]{article}
\usepackage[ngerman]{babel}
\usepackage[utf8]{inputenc}
\usepackage[T1]{fontenc}
\usepackage{amsmath}
\usepackage{amsfonts}
\usepackage{amssymb}
\usepackage[version=4]{mhchem}
\usepackage{stmaryrd}

\begin{document}
iii) Durch Normierung erhält man aus dem Normalen-Vektor den Einheitsnormalen-Vektor


\begin{equation*}
\hat{\mathbf{n}}:= \pm \frac{1}{|\mathbf{n}|} \cdot \mathbf{n} . \tag{2.103}
\end{equation*}


iv) Für geschlossene Flächen $M$ wird das Vorzeichen des Einheitsnormalen-Vektors so gewählt, dass $\hat{\mathbf{n}}$ überall nach aussen zeigt.

\subsection*{2.4.1.4 Metrik}
Wir betrachten folgende Definition.

\section*{Definition 2.14 Metrik}
Die Metrik einer parametrisierten Fläche ist die Gram-Matrix der Koordinatenbasis-Vektorfelder, d.h.

\[
G:=\left[\begin{array}{ll}
g_{11} & g_{12}  \tag{2.104}\\
g_{21} & g_{22}
\end{array}\right]:=\left[\begin{array}{ll}
\left\langle\mathbf{e}_{1}, \mathbf{e}_{1}\right\rangle & \left\langle\mathbf{e}_{1}, \mathbf{e}_{2}\right\rangle \\
\left\langle\mathbf{e}_{2}, \mathbf{e}_{1}\right\rangle & \left\langle\mathbf{e}_{2}, \mathbf{e}_{2}\right\rangle
\end{array}\right] .
\]

Bemerkungen:\\
i) Weil das Skalar-Produkt symmetrisch ist, gilt dies auch für $G$, d.h.


\begin{equation*}
G^{T}=G \tag{2.105}
\end{equation*}


ii) Weil das Skalar-Produkt positiv definit ist, gilt dies auch für $G$, d.h.


\begin{equation*}
g:=\operatorname{det}(G)=g_{11} \cdot g_{22}-g_{21} \cdot g_{12}>0 . \tag{2.106}
\end{equation*}


iii) In vielen Anwendungen stehen die Koordinatenbasis-Vektorfelder senkrecht aufeinander. Allgemein gilt


\begin{equation*}
\mathbf{e}_{1} \perp \mathbf{e}_{2} \Leftrightarrow G \text { ist diagonal. } \tag{2.107}
\end{equation*}


\subsection*{2.4.1.5 Mass-Funktion}
Wir betrachten folgende Definition.

\section*{Definition 2.15 Mass-Funktion}
Seien $G$ die Metrik einer parametrisierten Fläche und


\begin{equation*}
g:=\operatorname{det}(G), \tag{2.108}
\end{equation*}


dann ist die Mass-Funktion die Wurzel $\sqrt{g}$.\\
Wir betrachten folgenden Satz.

\section*{Satz 2.8 Normalen-Vektor \& Mass-Funktion}
Seien $\mathbf{n}$ der Normalen-Vektor und $\sqrt{g}$ die Mass-Funktion einer parametrisierten Fläche, dann gilt

\[
\begin{array}{|l|}
\hline g  \tag{2.109}\\
\hline
\end{array}
\]


\end{document}