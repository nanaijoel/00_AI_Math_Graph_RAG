\documentclass[10pt]{article}
\usepackage[ngerman]{babel}
\usepackage[utf8]{inputenc}
\usepackage[T1]{fontenc}
\usepackage{amsmath}
\usepackage{amsfonts}
\usepackage{amssymb}
\usepackage[version=4]{mhchem}
\usepackage{stmaryrd}
\usepackage{bbold}
\usepackage{graphicx}
\usepackage[export]{adjustbox}
\graphicspath{ {./images/} }

\begin{document}
\subsection*{2.3.1.4 Integration über allgemeine Gebiete}
Mit Hilfe des Fubini-Satzes können auch Integrale über allgemeinere Gebiete $G \subset \mathbb{R}$ berechnet. Wir zeigen im folgenden eine Auswahl von besonders häufig auftretenden Fällen.

\begin{enumerate}
  \item Wir betrachten ein in $y$-Richtung durch Graphen begrenztes Gebiet, wie im folgendem $x-y$-Diagramm dargestellt.\\
\includegraphics[max width=\textwidth, center]{2025_05_07_bfcec06cb7f780c5d25bg-1}
\end{enumerate}

Gemäss Fubini-Satz erhalten wir in diesem Fall\\
$\int_{G} f \mathrm{~d} A=\int_{x_{0}}^{x_{\mathrm{E}}} \int_{u(x)}^{v(x)} f(x ; y) \mathrm{d} y \mathrm{~d} x$.\\
2. Wir betrachten ein in $x$-Richtung durch Graphen begrenztes Gebiet, wie im folgendem $x$ - $y$-Diagramm dargestellt.\\
\includegraphics[max width=\textwidth, center]{2025_05_07_bfcec06cb7f780c5d25bg-1(1)}

Gemäss Fubini-Satz erhalten wir in diesem Fall $\int_{G} f \mathrm{~d} A=\int_{y_{0}}^{y_{\mathrm{E}}} \int_{u(y)}^{v(y)} f(x ; y) \mathrm{d} x \mathrm{~d} y$.


\end{document}