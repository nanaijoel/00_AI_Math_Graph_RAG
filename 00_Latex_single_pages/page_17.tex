\documentclass[10pt]{article}
\usepackage[ngerman]{babel}
\usepackage[utf8]{inputenc}
\usepackage[T1]{fontenc}
\usepackage{amsmath}
\usepackage{amsfonts}
\usepackage{amssymb}
\usepackage[version=4]{mhchem}
\usepackage{stmaryrd}
\usepackage{graphicx}
\usepackage[export]{adjustbox}
\graphicspath{ {./images/} }

\begin{document}
iii) Der Kurvenparameter $\tau$ kann je nach parametrisierter Kurve ganz unterschiedliche geometrische oder physikalische Bedeutungen haben.

\begin{itemize}
  \item physikalische Zeit (Kinematik)
  \item Länge
  \item Winkel
  \item eine beliebige "Absteckung" der Bahn\\
iv) Die Bahn ist die Menge aller von der parametrisierten Kurve durchlaufenen Punkte, d.h. sie ist die Kurve im klassischen geometrischen Sinn.\\
v) Der Bahnvektor ist der Richtungsvektor (Einheitsvektor) entlang der Bahn.\\
vi) Die Parametrisierung $\mathbf{s}(\tau)$ muss nicht injektiv sein, aber soll so gewählt werden, dass gilt $v(\tau)>0$. Ausnahmen an einzelnen Punkten sind jedoch möglich. An diesen muss der Bahnvektor ê von Hand definiert werden.\\
vii) Mehrere unterschiedliche parametrisierte Kurven können die gleiche Bahn haben.
\end{itemize}

Beispiele:

\begin{itemize}
  \item Wir betrachten die parametrisierte Kurve\\
$\mathbf{s}(\tau)=\left[\begin{array}{c}3 \cdot \cos (\tau) \\ 3 \cdot \sin (\tau)\end{array}\right] \quad$ für $\quad \tau \in[0, \pi]$.\\
Die Bahn dieser parametrisierten Kurve ist ein Halbkreis mit Mittelpunkt am Ursprung und Radius $r=3$.\\
\includegraphics[max width=\textwidth, center]{2025_05_07_5a3a52fc98004389f015g-1}
  \item Wir betrachten die parametrisierte Kurve\\
$\mathbf{s}(\tau)=\left[\begin{array}{c}3 \cdot \cos (\pi \cdot \tau) \\ 3 \cdot \sin (\pi \cdot \tau)\end{array}\right] \quad$ für $\quad \tau \in[0,1]$.\\
Die Bahn ist die gleiche wie beim ersten Beispiel.
  \item Wir betrachten die parametrisierte Kurve\\
$\mathbf{s}(\tau)=\left[\begin{array}{c}3 \cdot \cos \left(\pi \cdot \tau^{2}\right) \\ 3 \cdot \sin \left(\pi \cdot \tau^{2}\right)\end{array}\right] \quad$ für $\quad \tau \in[0,1]$.\\
Die Bahn ist die gleiche wie beim ersten Beispiel.
  \item Wir betrachten die parametrisierte Kurve\\
$\mathbf{s}(\tau)=\left[\begin{array}{l}3 \\ 2\end{array}\right]+\left[\begin{array}{r}-2 \\ 1\end{array}\right] \cdot \tau \quad$ für $\quad \tau \in[0,4]$.\\
Die Bahn ist eine gerade Strecke zwischen den Punkten $(3 ; 2)$ und $(-5 ; 6)$.
\end{itemize}

\end{document}