\documentclass[10pt]{article}
\usepackage[ngerman]{babel}
\usepackage[utf8]{inputenc}
\usepackage[T1]{fontenc}
\usepackage{amsmath}
\usepackage{amsfonts}
\usepackage{amssymb}
\usepackage[version=4]{mhchem}
\usepackage{stmaryrd}
\usepackage{bbold}

\begin{document}
\subsection*{6.1.3.5 Invertierbare Matrix}
Einige quadratische Matrizen haben so etwas wie einen Kehrwert.

\section*{Definition 6.11 Invertierbare Matrix}
Sei $n \in \mathbb{N}^{+}$. Eine quadratische Matrix $A \in \mathbb{M}(n, n, \mathbb{R})$ heisst invertierbar, falls es eine quadratische Matrix $A^{-1} \in \mathbb{M}(n, n, \mathbb{R})$ gibt, so dass


\begin{equation*}
A^{-1} \cdot A=\mathbb{1} . \tag{6.24}
\end{equation*}


Bemerkungen:\\
i) Invertierbare Matrizen werden auch regulär genannt, während nicht invertierbare Matri$z e n$ als singulär bezeichnet werden.\\
ii) Die quadratische Matrix $A^{-1}$, falls es die denn gibt, wird Inverse Matrix von $A$ genannt.\\
iii) Obwohl die Matrix-Multiplikation im allgemeinen nicht kommutativ ist, lässt sich die Reihenfolge in (6.24) immer vertauschen. Ist $A$ invertierbar, dann gilt


\begin{equation*}
A^{-1} \cdot A=A \cdot A^{-1}=\mathbb{1} . \tag{6.25}
\end{equation*}


Daraus erhält man den Kommutator


\begin{equation*}
\left[A, A^{-1}\right]=A \cdot A^{-1}-A^{-1} \cdot A=\mathbb{1}-\mathbb{1}=0 . \tag{6.26}
\end{equation*}


iv) Die Einheitsmatrix ist offensichtlich invertierbar und ihre eigene Inverse, es gilt also


\begin{equation*}
\mathbb{1}^{-1}=\mathbb{1} . \tag{6.27}
\end{equation*}


v) Die Nullmatrix ist offensichtlich singulär, d.h. nicht invertierbar.\\
vi) Die Inversion einer reellen Matrix ist offensichtlich eine Involution. Für jede invertierbare Matrix gilt


\begin{equation*}
\left(A^{-1}\right)^{-1}=A . \tag{6.28}
\end{equation*}


vii) Beispiel-Codes zur Berechnung von inversen Matrizen mit gängiger Software.

\begin{center}
\begin{tabular}{|l|l|}
\hline
MATLAB/Octave & M=inv(A) \\
\hline
Mathematica/WolframAlpha & M=Inverse [A] \\
\hline
Python/Numpy & \begin{tabular}{l}
import numpy as np; \\
$M=n p . l i n a l g . i n v ~(A) ~$ \\
\end{tabular} \\
\hline
Python/Sympy & $M=A . \operatorname{inv}()$ \\
\hline
\end{tabular}
\end{center}

Beispiele:

\begin{itemize}
  \item $[2]^{-1}=\left[\frac{1}{2}\right]$
  \item $\left[\begin{array}{ll}2 & 1 \\ 5 & 3\end{array}\right]^{-1}=\left[\begin{array}{rr}3 & -1 \\ -5 & 2\end{array}\right]$
\end{itemize}

\end{document}