\documentclass[10pt]{article}
\usepackage[ngerman]{babel}
\usepackage[utf8]{inputenc}
\usepackage[T1]{fontenc}
\usepackage{amsmath}
\usepackage{amsfonts}
\usepackage{amssymb}
\usepackage[version=4]{mhchem}
\usepackage{stmaryrd}
\usepackage{graphicx}
\usepackage[export]{adjustbox}
\graphicspath{ {./images/} }

\begin{document}
Beweis: Um das Volumen zwischen dem Reckteck $G$ und dem Graphen von $f$ zu berechnen verwenden wir einen Archimedes-Cauchy-Riemann-Approximationsprozess. Dabei gehen wir nach folgenden Schritten vor.

S1 Lokal: Wir betrachten einen kleinen Streifen des Rechtecks $G$ mit Breite $\delta x>0$, wie im folgenden $x-y$ - $z$-Diagramm dargestellt.\\
\includegraphics[max width=\textwidth, center]{2025_05_07_5faa0fe9ee4abe22ae77g-1}

Die Querschnittsfläche $A_{\mathrm{Q}}(x)$ können wir berechnen durch das bestimmte Integral\\
$A_{\mathrm{Q}}(x)=\int_{y_{0}}^{y_{\mathrm{E}}} f(x ; y) \mathrm{d} y$.\\
Das Volumen zwischen dem kleinen Streifen und dem Graphen von $f$ ist\\
$\underline{\delta I} \approx \underline{A_{Q}(x) \cdot \delta x}$.\\
S2 Global: Durch Integration über $x$ können wir das gesamte Volumen zwischen dem Rechteck $G$ und dem Graphen von $f$ berechnen. Wir erhalten\\
$\underline{\underline{I}}=\int_{x_{0}}^{x_{\mathrm{E}}} A_{\mathrm{Q}}(x) \mathrm{d} x=\underline{\underline{\int_{x_{0}}} x_{y_{0}}^{x_{\mathrm{E}}} y_{\mathrm{E}}} f(x ; y) \mathrm{d} y \mathrm{~d} x$.

Durch Vertauschen der Rollen von $x$ und $y$ erhalten wir auf analoge Weise die zweite Version. Damit haben wir den Satz bewiesen.

Bemerkungen:\\
i) Durch den Fubini-Satz kann ein Integral über ein Gebiet in 2D auf die Berechnung von 2 verschachtelten Integralen in 1D zurückgeführt werden und umgekehrt.\\
ii) Bei der Anwendung des Fubini-Satzes werden die Zweifach-Integrale in der Reihenfolge von innen nach aussen berechnet.\\
iii) Bei der Integration über ein Rechteck kann die Integrationsreihenfolge vertauscht werden.\\
iv) Der Fubini-Satz ist ein allgemein gültiges Prinzip, dessen Aussage sowohl auf allgemeinere Gebiete als auch auf Gebiete in nD erweitert werden kann.


\end{document}