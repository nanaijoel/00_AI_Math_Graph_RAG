\documentclass[10pt]{article}
\usepackage[ngerman]{babel}
\usepackage[utf8]{inputenc}
\usepackage[T1]{fontenc}
\usepackage{amsmath}
\usepackage{amsfonts}
\usepackage{amssymb}
\usepackage[version=4]{mhchem}
\usepackage{stmaryrd}
\usepackage{bbold}

\begin{document}
\subsection*{6.1.3.4 Einheitsmatrix}
Eine ganz spezielle Rolle spielen quadratische Matrizen, die auf der Hauptdiagonalen nur Einsen und sonst überall nur Nullen haben.

Definition 6.10 Einheitsmatrix\\
Sei $n \in \mathbb{N}^{+}$. Die Matrix $\mathbb{1} \in \mathbb{M}(n, n, \mathbb{R})$ mit

\[
\mathbb{1}=\left[\begin{array}{cccc}
1 & 0 & \ldots & 0  \tag{6.20}\\
0 & 1 & \ldots & 0 \\
\vdots & \vdots & \ddots & \vdots \\
0 & 0 & \ldots & 1
\end{array}\right]
\]

heisst Einheitsmatrix.

Bemerkungen:\\
i) Alle Einheitsmatrizen werden unabhängig von ihrer Dimension identifiziert und mit $\mathbb{1}$ bezeichnet. Es gilt also

\[
\mathbb{1}=[1]=\left[\begin{array}{ll}
1 & 0  \tag{6.21}\\
0 & 1
\end{array}\right]=\left[\begin{array}{lll}
1 & 0 & 0 \\
0 & 1 & 0 \\
0 & 0 & 1
\end{array}\right]=\left[\begin{array}{llll}
1 & 0 & 0 & 0 \\
0 & 1 & 0 & 0 \\
0 & 0 & 1 & 0 \\
0 & 0 & 0 & 1
\end{array}\right]=\ldots .
\]

ii) Die Einheitsmatrix hat die gleichen algebraischen Eigenschaften wie die Zahl Eins. Für jede Matrix $A$ gilt


\begin{equation*}
\mathbb{1} \cdot A=A \cdot \mathbb{1}=A . \tag{6.22}
\end{equation*}


iii) Ist $A$ eine quadratische Matrix, dann folgt der Kommutator


\begin{equation*}
[A, \mathbb{1}]=A \cdot \mathbb{1}-\mathbb{1} \cdot A=A-A=0 . \tag{6.23}
\end{equation*}


Das heisst, die Einheitsmatrix kommutiert mit jeder quadratischen Matrix.\\
iv) Die Einheitsmatrix ist offensichtlich symmetrisch.\\
v) Beispiel-Codes zum Erzeugen von Einheitsmatrizen mit gängiger Software.

\begin{center}
\begin{tabular}{|l|l|}
\hline
MATLAB/Octave & \begin{tabular}{l}
M=eye (3) \\
M=eye $(2,3)$ \\
\end{tabular} \\
\hline
Mathematica/WolframAlpha & M=IdentityMatrix [3] \\
\hline
Python/Numpy & \begin{tabular}{l}
import numpy as np; \\
$\mathrm{M}=\mathrm{np}$.eye (3) \\
$M=n p$.eye $(2,3)$ \\
\end{tabular} \\
\hline
Python/Sympy & \begin{tabular}{l}
import sympy as sp; \\
M=sp.eye(3) \\
M=sp.eye $(2,3)$ \\
\end{tabular} \\
\hline
\end{tabular}
\end{center}


\end{document}