\documentclass[10pt]{article}
\usepackage[ngerman]{babel}
\usepackage[utf8]{inputenc}
\usepackage[T1]{fontenc}
\usepackage{amsmath}
\usepackage{amsfonts}
\usepackage{amssymb}
\usepackage[version=4]{mhchem}
\usepackage{stmaryrd}
\usepackage{bbold}
\usepackage{graphicx}
\usepackage[export]{adjustbox}
\graphicspath{ {./images/} }

\begin{document}
\subsection*{3.2 Uneigentliche Integrale}
\subsection*{3.2.1 Integration über unendliche Intervalle}
\subsection*{3.2.1.1 Integration über ein einseitig unendliches Intervall}
Eine Fläche kann durch eine unendlich lange Randkurve begrenzt sein und trotzdem einen endlichen Flächeninhalt haben. Wir betrachten den Graphen einer Funktion $f: \mathbb{R} \rightarrow \mathbb{R}$, welche sich für $x \rightarrow \infty$ asymptotisch der $x$-Achse annähert. Eine solche Situation ist im folgenden $x$ - $y$-Diagramm dargestellt.\\
\includegraphics[max width=\textwidth, center]{2025_05_07_2bba3dc5ed5033204f9eg-1}

Wir betrachten die folgende Definition.\\
Definition 3.1 Uneigentliches Integral mit einer unendlichen Grenze.\\
Seien $x_{0}, x_{\mathrm{E}} \in \mathbb{R}$ und $f: \mathbb{R} \rightarrow \mathbb{R}$ eine integrierbare Funktion. Die uneigentlichen Integrale von $f$ Richtung $\pm \infty$ sind\\
(a) $\int_{x_{0}}^{\infty} f(x) \mathrm{d} x:=\lim _{s \rightarrow \infty} \int_{x_{0}}^{s} f(x) \mathrm{d} x$,\\
(b) $\int_{-\infty}^{x_{\mathrm{E}}} f(x) \mathrm{d} x:=\lim _{s \rightarrow \infty} \int_{-s}^{x_{\mathrm{E}}} f(x) \mathrm{d} x$,\\
falls die betreffenden Grenzwerte jeweils konvergieren.

Beispiele:

\begin{itemize}
  \item Wir betrachten das uneigentliche Integral
\end{itemize}


\begin{equation*}
\underline{\underline{I}}=\int_{2}^{\infty} \frac{1}{x^{2}} \mathrm{~d} x=\lim _{s \rightarrow \infty} \int_{2}^{s} \frac{1}{x^{2}} \mathrm{~d} x=\left.\lim _{s \rightarrow \infty}\left[-\frac{1}{x}\right]\right|_{2} ^{s}=\lim _{s \rightarrow \infty}\left(-\frac{1}{s}+\frac{1}{2}\right)=0+\frac{1}{2}=\underline{\underline{\frac{1}{2}}} . \tag{3.30}
\end{equation*}


\begin{itemize}
  \item Wir betrachten das uneigentliche Integral
\end{itemize}


\begin{equation*}
I=\int_{2}^{\infty} \frac{1}{x} \mathrm{~d} x=\lim _{s \rightarrow \infty} \int_{2}^{s} \frac{1}{x} \mathrm{~d} x=\lim _{s \rightarrow \infty} \ln \left(\frac{s}{2}\right)=\infty . \tag{3.31}
\end{equation*}


Dieses uneigentliche Integral ist divergent und existiert daher nicht.

\begin{itemize}
  \item Wir betrachten das uneigentliche Integral
\end{itemize}


\begin{align*}
\underline{\underline{I}} & =\int_{\ln (2)}^{\infty} \mathrm{e}^{-x} \mathrm{~d} x=\lim _{s \rightarrow \infty} \int_{\ln (2)}^{s} \mathrm{e}^{-x} \mathrm{~d} x=\left.\lim _{s \rightarrow \infty}\left[-\mathrm{e}^{-x}\right]\right|_{\ln (2)} ^{s}=\lim _{s \rightarrow \infty}\left(-\mathrm{e}^{-s}+\mathrm{e}^{-\ln (2)}\right)=0+\mathrm{e}^{-\ln (2)} \\
& =\mathrm{e}^{-\ln (2)}=\frac{1}{\mathrm{e}^{\ln (2)}}=\underline{\underline{1}} . \tag{3.32}
\end{align*}



\end{document}