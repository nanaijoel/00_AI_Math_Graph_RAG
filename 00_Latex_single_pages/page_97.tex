\documentclass[10pt]{article}
\usepackage[ngerman]{babel}
\usepackage[utf8]{inputenc}
\usepackage[T1]{fontenc}
\usepackage{amsmath}
\usepackage{amsfonts}
\usepackage{amssymb}
\usepackage[version=4]{mhchem}
\usepackage{stmaryrd}
\usepackage{bbold}

\begin{document}
\subsection*{6.1.3.3 Nullmatrix}
Eine reelle Matrix, deren Komponenten alle verschwinden, heisst Nullmatrix.\\
Definition 6.9 Nullmatrix\\
Seien $m, n \in \mathbb{N}^{+}$. Die Matrix $0 \in \mathbb{M}(m, n, \mathbb{R})$ mit

\[
0=\left[\begin{array}{cccc}
0 & 0 & \ldots & 0  \tag{6.17}\\
0 & 0 & \ldots & 0 \\
\vdots & \vdots & \ddots & \vdots \\
0 & 0 & \ldots & 0
\end{array}\right]
\]

heisst Nullmatrix.

Bemerkungen:\\
i) Alle Nullmatrizen werden unabhängig von ihren Dimensionen identifiziert und mit 0 bezeichnet. Es gilt also

\[
0=[0]=\left[\begin{array}{ll}
0 & 0
\end{array}\right]=\left[\begin{array}{l}
0  \tag{6.18}\\
0
\end{array}\right]=\left[\begin{array}{ll}
0 & 0 \\
0 & 0
\end{array}\right]=\left[\begin{array}{lll}
0 & 0 & 0 \\
0 & 0 & 0
\end{array}\right]=\left[\begin{array}{ll}
0 & 0 \\
0 & 0 \\
0 & 0
\end{array}\right]=\ldots .
\]

ii) Die Nullmatrix hat die gleichen algebraischen Eigenschaften wie die Zahl Null. Für jede Matrix A gilt


\begin{equation*}
A+0=A \quad \text { und } \quad 0 \cdot A=0 . \tag{6.19}
\end{equation*}


iii) Die quadratischen Nullmatrizen sind die einzigen Matrizen die sowohl symmetrisch als auch schiefsymmetrisch sind.\\
iv) Beispiel-Codes zum Erzeugen von Nullmatrizen mit gängiger Software.

\begin{center}
\begin{tabular}{|l|l|}
\hline
MATLAB/Octave & \begin{tabular}{l}
M=zeros (3) \\
M=zeros $(2,3)$ \\
\end{tabular} \\
\hline
Mathematica/WolframAlpha & \begin{tabular}{l}
M=ConstantArray $[0,\{3,3\}]$ \\
M=ConstantArray [0, $\{2,3\}]$ \\
\end{tabular} \\
\hline
Python/Numpy & \begin{tabular}{l}
import numpy as np; \\
$M=n p . z e r o s((3,3))$ \\
M=np.zeros $((2,3))$ \\
\end{tabular} \\
\hline
Python/Sympy & \begin{tabular}{l}
import sympy as sp; \\
M=sp.zeros (3) \\
M=sp.zeros $(2,3)$ \\
\end{tabular} \\
\hline
\end{tabular}
\end{center}


\end{document}