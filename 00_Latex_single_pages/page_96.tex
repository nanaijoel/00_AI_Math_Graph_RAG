\documentclass[10pt]{article}
\usepackage[ngerman]{babel}
\usepackage[utf8]{inputenc}
\usepackage[T1]{fontenc}
\usepackage{amsmath}
\usepackage{amsfonts}
\usepackage{amssymb}
\usepackage[version=4]{mhchem}
\usepackage{stmaryrd}
\usepackage{bbold}

\begin{document}
\subsection*{6.1.3.2 Symmetrische \& schiefsymmetrische Matrix}
Spezielle Bedeutung haben quadratische Matrizen, die bei Transposition in sich selbst oder in ihr Negatives übergehen.

Definition 6.8 Symmetrische \& schiefsymmetrische Matrix\\
Seien $n \in \mathbb{N}^{+}$und $A \in \mathbb{M}(n, n, \mathbb{R})$.\\
(a) $A$ ist symmetrisch, genau falls $A^{T}=A$.\\
(b) $A$ ist schiefsymmetrisch, genau falls $A^{T}=-A$.

Bemerkungen:\\
i) Die Komponenten auf der Hauptdiagonalen einer schiefsymmetrischen Matrix müssen zwingend verschwinden.\\
ii) Eine symmetrische Matrix $A \in \mathbb{M}(n, n, \mathbb{R})$ hat die Anzahl unabhängiger Komponenten von


\begin{equation*}
n_{+}=\frac{n \cdot(n+1)}{2} \tag{6.15}
\end{equation*}


iii) Eine schiefsymmetrische Matrix $A \in \mathbb{M}(n, n, \mathbb{R})$ hat die Anzahl unabhängiger Komponenten von


\begin{equation*}
n_{-}=\frac{n \cdot(n-1)}{2} \tag{6.16}
\end{equation*}


Beispiele:

\begin{itemize}
  \item $\left[\begin{array}{ll}1 & 2 \\ 2 & 3\end{array}\right]$ ist symmetrisch
  \item $\left[\begin{array}{rr}0 & -2 \\ 2 & 0\end{array}\right]$ ist schiefsymmetrisch
  \item $\left[\begin{array}{rrr}0 & -1 & 2 \\ 1 & 0 & -3 \\ -2 & 3 & 0\end{array}\right]$ ist schiefsymmetrisch
\end{itemize}

\end{document}