\documentclass[10pt]{article}
\usepackage[ngerman]{babel}
\usepackage[utf8]{inputenc}
\usepackage[T1]{fontenc}
\usepackage{amsmath}
\usepackage{amsfonts}
\usepackage{amssymb}
\usepackage[version=4]{mhchem}
\usepackage{stmaryrd}
\usepackage{bbold}

\begin{document}
\begin{itemize}
  \item Trägheitsmoment: Wir betrachten einen Körpers $K$ mit Dichte $\rho: \mathbb{R}^{3} \rightarrow \mathbb{R}$. Der Abstand eines Punktes von einer vorgegebenen Drehachse sei $r: \mathbb{R}^{3} \rightarrow \mathbb{R}$.
\end{itemize}

S1 Lokal: Das Trägheitsmoment eines kleinen Volumenstücks $\delta V$ im $x-y$ - $z$-Raum ist


\begin{equation*}
\underline{\delta I} \approx r^{2} \cdot \delta m \approx r^{2} \cdot \rho \cdot \delta V=\underline{r^{2}(x ; y ; z) \cdot \rho(x ; y ; z) \cdot \delta V} . \tag{2.96}
\end{equation*}


S2 Global: Durch Integration über den Körper $K$ können wir sein gesamtes Trägheitsmoment berechnen. Wir erhalten


\begin{equation*}
I=\int_{K} r^{2} \cdot \rho \mathrm{~d} V . \tag{2.97}
\end{equation*}



\end{document}