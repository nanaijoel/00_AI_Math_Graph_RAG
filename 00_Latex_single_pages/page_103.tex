\documentclass[10pt]{article}
\usepackage[ngerman]{babel}
\usepackage[utf8]{inputenc}
\usepackage[T1]{fontenc}
\usepackage{amsmath}
\usepackage{amsfonts}
\usepackage{amssymb}
\usepackage[version=4]{mhchem}
\usepackage{stmaryrd}
\usepackage{bbold}

\begin{document}
In der Analysis heisst eine Funktion genau dann linear, wenn ihr Graph eine Gerade ist. Somit ist im Funktionsterm nebst dem Term $m \cdot x$ auch noch eine Konstante $q \neq 0$ zulässig. In der linearen Algebra würde solch ein $q$ jedoch die Gültigkeit der Haupteigenschaft (6.33) verhindern.

\subsection*{6.2.2 Eigenschaften}
Die Abbildungsmatrix der Verknüpfung von zwei linearen Abbildungen lässt sich auf einfache Weise aus den Abbildungsmatrizen der beiden verknüpften linearen Abbildungen berechnen.

\section*{Satz 6.3 Verknüpfungssatz}
Seien $m, n, l \in \mathbb{N}^{+}$und $a: \mathbb{R}^{m} \rightarrow \mathbb{R}^{n}$ sowie $b: \mathbb{R}^{n} \rightarrow \mathbb{R}^{l}$ zwei lineare Abbildungen mit Abbildungsmatrizen $A \in \mathbb{M}(n, m, \mathbb{R})$ bzw. $B \in \mathbb{M}(l, n, \mathbb{R})$. Dann ist die Verknüpfung


\begin{align*}
c: \mathbb{R}^{m} & \rightarrow \mathbb{R}^{l} \\
\mathbf{x} & \mapsto c(\mathbf{x}):=b(a(\mathbf{x})) \tag{6.37}
\end{align*}


ebenfalls eine lineare Abbildung mit Abbildungsmatrix


\begin{equation*}
C=B \cdot A . \tag{6.38}
\end{equation*}


Beweis: Wegen der Assoziativität des Matrix-Produkts finden wir für alle $\mathbf{x} \in \mathbb{R}^{m}$


\begin{equation*}
c(\mathbf{x})=b(a(\mathbf{x}))=b(A \cdot \mathbf{x})=B \cdot(A \cdot \mathbf{x})=\underline{(B \cdot A) \cdot \mathbf{x}=: C \cdot \mathbf{x},} \tag{6.39}
\end{equation*}


wobei gelten muss


\begin{equation*}
\underline{C=B \cdot A .} \tag{6.40}
\end{equation*}


Damit haben wir den Satz bewiesen.\\
Ist eine lineare Abbildung bijektiv, dann ist auch die Umkehrabbildung wieder eine lineare Abbildung, deren Abbildungsmatrix gerade die Inverse der ursprünglichen Abbildungsmatrix ist.

Satz 6.4 Inversionssatz\\
Seien $m, n \in \mathbb{N}^{+}$und $a: \mathbb{R}^{m} \rightarrow \mathbb{R}^{n}$ eine lineare Abbildung mit Abbildungsmatrix $A \in \mathbb{M}(n, m, \mathbb{R})$. Dann gilt folgendes.\\
(a) a bijektiv $\Rightarrow n=m$\\
(b) a bijektiv $\Leftrightarrow A$ regulär\\
(c) $a$ bijektiv $\Rightarrow a^{-1}(\mathbf{y})=A^{-1} \cdot \mathbf{y}$.

Bemerkungen:\\
i) Die Verknüpfung von zwei linearen Abbildungen geschieht durch Matrix-Multiplikation der Abbildungsmatrizen. Dabei muss die Reihenfolge beachtet werden, d.h.


\begin{equation*}
a(\mathbf{x}):=a_{N}\left(\ldots a_{2}\left(a_{1}(\mathbf{x})\right)\right) \Rightarrow A=A_{N} \cdot \ldots \cdot A_{2} \cdot A_{1} . \tag{6.41}
\end{equation*}


ii) Die Umkehrung einer bijektiven linearen Abbildung geschieht durch Inversion der Abbildungsmatrix.


\end{document}