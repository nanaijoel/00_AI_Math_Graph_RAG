\documentclass[10pt]{article}
\usepackage[ngerman]{babel}
\usepackage[utf8]{inputenc}
\usepackage[T1]{fontenc}
\usepackage{amsmath}
\usepackage{amsfonts}
\usepackage{amssymb}
\usepackage[version=4]{mhchem}
\usepackage{stmaryrd}
\usepackage{bbold}

\begin{document}

\begin{align*}
& =\frac{p}{n!} \int_{0}^{x} \mathrm{e}^{s} \cdot|s-x|^{n} \mathrm{~d} s \leq \frac{p}{n!} \int_{0}^{x} \mathrm{e}^{|x|} \cdot|0-x|^{n} \mathrm{~d} s=\frac{p}{n!} \int_{0}^{x} \mathrm{e}^{|x|} \cdot|x|^{n} \mathrm{~d} s \\
& =\mathrm{e}^{|x|} \cdot|x|^{n} \cdot \frac{p}{n!} \int_{0}^{x} 1 \mathrm{~d} s=\left.\mathrm{e}^{|x|} \cdot \frac{|x|^{n}}{n!} \cdot p \cdot[s]\right|_{0} ^{x}=\mathrm{e}^{|x|} \cdot \frac{|x|^{n}}{n!} \cdot p \cdot(x-0) \\
& =\mathrm{e}^{|x|} \cdot \frac{|x|^{n}}{n!} \cdot p \cdot x=\mathrm{e}^{|x|} \cdot|x| \cdot \frac{|x|^{n}}{n!} \xrightarrow{n \rightarrow \infty} \mathrm{e}^{|x|} \cdot|x| \cdot 0=\underline{0} \tag{4.14}
\end{align*}


Somit ist exp auf ganz $\mathbb{R}$ analytisch und kann dargestellt werden durch die Maclaurin-Reihe


\begin{equation*}
\underline{\underline{\exp (x)=\sum_{k=0}^{\infty} \frac{x^{k}}{k!}=1+x+\frac{x^{2}}{2!}+\frac{x^{3}}{3!}+\frac{x^{4}}{4!}+\frac{x^{5}}{5!}+\ldots .}} \tag{4.15}
\end{equation*}


Siehe Übungen für die andern Funktionen.\\
Bemerkungen:\\
i) Es ist heute allgemein üblich, diese Elementarfunktionen durch ihre MACLAURIN-Reihen zu definieren.\\
ii) Die Maclaurin-Entwicklungen dieser Elementarfunktionen sind in modernen Taschenrechnern, PCs und auch Grossrechnern hardwareseitig implementiert.\\
iii) Die Maclaurin-Reihe der natürlichen Exponentialfunktion wird Exponentialreihe genannt. Durch Einsetzen von $x=1$ erhält man für die Euler-Zahl die bekannte Reihendarstellung\\
$\mathrm{e}=\exp (1)=\sum_{k=0}^{\infty} \frac{1}{k!}=1+1+\frac{1}{2!}+\frac{1}{3!}+\frac{1}{4!}+\frac{1}{5!}+\ldots$.\\
iv) Die Maclaurin-Entwicklungen dürfen Termweise abgeleitet werden, wordurch die bekannten Ableitungsregeln offensichtlich werden.\\
v) Die Maclaurin-Entwicklungen zeigen deutlich die Verwandtschaft zwischen trigonometrischen und hyperbolischen Funktionen.

\subsection*{4.1.3 Anwendungen}
Maclaurin-Entwicklungen haben zahlreiche Anwendungen.

\begin{itemize}
  \item Definition von Funktionen
  \item Numerische Berechnung von Funktionswerten
  \item Numerische Näherungen von Funktionswerten
  \item Vereinfachungen von Funktionen in der Nähe von 0 (z.B. Kleinwinkelnäherungen)
  \item Linearisierungen von Differentialgleichungen (z.B. in der Regelungstechnik)
  \item Untersuchung einer Funktion in der Nähe von 0
  \item Untersuchung von Beziehungen zwischen Funktionen (z.B. Euler-Formel)
  \item Übergänge zwischen Modellen (z.B. Relativitätstheorie und NEWTON-Mechanik)
  \item Informationstheorie
\end{itemize}

\end{document}